\documentclass[11pt]{article}
\usepackage[letterpaper,top=2cm,bottom=2cm,left=2cm,right=2cm,marginparwidth=1.75cm]{geometry}

 % Useful packages 
\usepackage{hyperref}
\usepackage{biblatex}
\usepackage{braket}
\usepackage{mathtools}
\DeclarePairedDelimiterXPP\BigOSI[2]%
  {\mathcal{O}}{(}{)}{}%
  {\SI{#1}{#2}}
\usepackage{amsmath}
\usepackage{amssymb}
\usepackage{empheq}
\usepackage[most]{tcolorbox}
\usepackage{amsmath}
\usepackage{mathrsfs}
\usepackage[utf8]{inputenc}
\usepackage{graphicx}
\usepackage{float}
\usepackage{parskip}
\usepackage{comment}
\usepackage{mhchem}
 \usepackage{tabularx}
 \usepackage{titling}
  \usepackage[explicit]{titlesec}
\usepackage{fancyhdr}
\usepackage{derivative}

\setlength{\droptitle}{3em} 
\title{Classical Field theory}
\author{Thomas Brosnan}
\date{Notes taken in Professor Andrei Parnachev's class, Michaelmas Term 2023}

  \newenvironment{bux}
    {
    \empheq[box=\tcbhighmath]{align}
   }{
    \endempheq
    }
    \newenvironment{bux*}
    {
    \empheq[box=\tcbhighmath]{align*}
   }{
    \endempheq
    }
\newcommand{\hsp}{\hspace{8pt}}

\numberwithin{equation}{section}

\newcommand*{\sectionFont}{%
  \LARGE\bfseries
}
\newtcbox{\mymath}[1][]{%
    nobeforeafter, math upper, tcbox raise base,
    enhanced, colframe=blue!30!black,
    colback=blue!30, boxrule=1pt,
    #1}


\makeatletter
\let\Title\@title % Copy the title to a new command
\makeatother

%change this RGB value to change the section background colour 
\definecolor{mycolor1}{RGB}{74, 174, 241}
\colorlet{SectionColour}{mycolor1}
%subsection background colour 
\definecolor{mycolor2}{gray}{0.8}
\colorlet{subSectionColour}{mycolor2}
%subsubsection background colour 
\definecolor{mycolor3}{RGB}{255,255,255}
\colorlet{subsubSectionColour}{mycolor3}
    
\begin{document}

\maketitle

\newpage
\vspace*{\fill}
\begin{center}
    \Large 
    "If you make an even number of mistakes you get the right answer."  

 - Andrei Parnachev
\end{center}
\vspace*{\fill}
\newpage

\tableofcontents
% For \section
 \titleformat{\section}[block]{\sectionFont}{}{0pt}{%
 \fcolorbox{black}{SectionColour}{\noindent\begin{minipage}{\dimexpr\textwidth-2\fboxsep-2\fboxrule\relax}\thesection  \hsp #1 {\strut} \end{minipage}}}
% For \subsection
 \titleformat{\subsection}[block]{\bfseries}{}{0pt}{%
 \fcolorbox{black}{subSectionColour}{\noindent\begin{minipage}{\dimexpr\textwidth-2\fboxsep-2\fboxrule\relax}\thesubsection  \hsp #1 {\strut} \end{minipage}}}
% For \section*
 \titleformat{name=\section, numberless}[block]{\sectionFont}{}{0pt}{%
 \fcolorbox{black}{SectionColour}{\noindent\begin{minipage}{\dimexpr\textwidth-2\fboxsep-2\fboxrule\relax} #1 {\strut} \end{minipage}}}
  % For \subsection*
 \titleformat{name=\subsection, numberless}[block]{\bfseries}{}{0pt}{%
 \fcolorbox{black}{subSectionColour}{\noindent\begin{minipage}{\dimexpr\textwidth-2\fboxsep-2\fboxrule\relax} #1 {\strut} \end{minipage}}}
 % For \subsubsection
 \titleformat{\subsubsection}[block]{\bfseries}{}{0pt}{%
 \fcolorbox{black}{subsubSectionColour}{\noindent\begin{minipage}{15cm}\thesubsubsection \hsp #1 {\strut} \end{minipage}}}
  % For \subsubsection*
 \titleformat{name=\subsubsection, numberless}[block]{\bfseries}{}{0pt}{%
 \fcolorbox{black}{subsubSectionColour}{\noindent\begin{minipage}{15cm} #1 {\strut} \end{minipage}}}
\newpage
%header 
\pagestyle{fancy}
\fancyhf{} % Clear all header and footer fields
\fancyhead[L]{\Title}
\fancyhead[R]{\nouppercase{\leftmark}}
\fancyfoot[C]{-~\thepage~-}
\renewcommand{\headrulewidth}{1pt}


\normalsize
\newpage
\tcbset{highlight math style={boxsep=5mm,colback=green!0!red!0!blue!0}}

\newpage
\section{Electrostatics}

\subsection{Coulombs Law:} 
\begin{itemize}
    \item Coulombs Law in SI units for two point charges $q_1$ and $q_2$ is given by:
\begin{empheq}[box=\tcbhighmath]{equation}
\begin{split}
    \textbf{F} = \frac{1}{4 \pi \epsilon_0}\frac{q_1q_2}{r^2}\hat{\textbf{n}}
\end{split}
\end{empheq}
This leads to the defining of the electric field $\textbf{E}$ generated by a single point charge, through the fact that the force a particle of charge $q_2$ experiences due to the presence of another charge $q_1$ is $\textbf{F} = q_2E_1$. In vector notation the electric field of the charge $q_1$ is :
\begin{empheq}[box=\tcbhighmath]{equation}
\begin{split}
\label{eqn:1.2}
    \textbf{E}_1(\textbf{x}) = \frac{q_1}{4 \pi \epsilon_0}\frac{\textbf{x}-\textbf{x}_1}{|\textbf{x}-\textbf{x}_1|^3}
\end{split}
\end{empheq}
This then generalises for many point charges to:
\begin{empheq}[box=\tcbhighmath]{equation}
\begin{split}
   \textbf{E}_1(\textbf{x}) = \frac{1}{4 \pi \epsilon_0}\sum_iq_i\frac{\textbf{x}-\textbf{x}_i}{|\textbf{x}-\textbf{x}_i|^3}
\end{split}
\end{empheq}
\end{itemize}
\subsection{Continuous distribution}
\begin{itemize}
    \item For a continuous distribution of charges we need to change to a charge density $\rho(\textbf{x})$. The charge of a small element is then given by $\Delta q = \rho(\textbf{x})\Delta x \Delta y \Delta z$. Putting this into our equation for many charges and taking the limit as $\Delta V=\Delta x \Delta y \Delta z \rightarrow 0$  results in:
\begin{empheq}[box=\tcbhighmath]{equation}
\begin{split}
\label{eqn:1.4}
\textbf{E}(\textbf{x}) = \frac{1}{4 \pi \epsilon_0}\int \rho(\textbf{x})\frac{\textbf{x}-\textbf{x}'}{|\textbf{x}-\textbf{x}'|^3}d^3x'
\end{split}
\end{empheq}
\end{itemize}

\subsection{Gauss' Law}
\begin{itemize}
    \item The above equation is not always the most suitable form to calculate $\textbf{E}(\textbf{x})$.  Another way is to consider a point $q$ and a closed surface S. If $r$ is the distance from the particle to a given point on the surface and $\hat{\textbf{n}}$ is a unit vector normal to the surface at this point. The vector $E$ dotted with $\hat{\textbf{n}}$, according to equation \ref{eqn:1.2} is:
\begin{empheq}[box=\tcbhighmath]{equation}
\begin{split}
   \textbf{E}\cdot \hat{\textbf{n}} = \frac{q}{4\pi \epsilon_0} \frac{\cos \theta}{r^2}
\end{split}
\end{empheq}
Here $\theta$ is just the angle between $\textbf{E}$ and $\hat{\textbf{n}}$. 

We can then multiply both sides by the infinitesimal surface area at this point $da$ and notice that since the RHS can be simplified with the solid angle $d\Omega = \frac{da \cos \theta}{r^2}$. Then integrating over the whole surface results in:
\begin{empheq}[box=\tcbhighmath]{equation}
\begin{split}
   \oint_S \textbf{E}\cdot \hat{\textbf{n}}da = \begin{cases}
       \frac{q}{\epsilon_0} ~~ \text{if q lies inside S} \\
       0   ~~\text{if q lies outside S}
   \end{cases}
\end{split}
\end{empheq}
The full form of Gauss' law generalizes this to a continuous distribution of charge with charge density $\rho$:
\begin{empheq}[box=\tcbhighmath]{equation}
\begin{split}
   \oint_S \textbf{E}\cdot \hat{\textbf{n}}da =\frac{1}{\epsilon_0}\int_V\rho(\textbf{x})d^3x
\end{split}
\end{empheq}


\end{itemize}
\subsection{Scalar potential }
\begin{itemize}
    \item To solve equation \ref{eqn:1.4} it helps us to notice the following derivative:
\begin{empheq}[box=\tcbhighmath]{equation}
\begin{split}
   \nabla \frac{1}{|\textbf{x} - \textbf{x}'|} = -\frac{\textbf{x}- \textbf{x}'}{|\textbf{x}-\textbf{x}'|^3}
\end{split}
\end{empheq}
Where $\nabla$ is wrt to $x$. This allows us to sub into equation \ref{eqn:1.4} and pull out the nabla as it is wrt $x$ and not $x'$. This leads to the new definition of the electric field as the derivative of a scalar potential: 
\begin{empheq}[box=\tcbhighmath]{equation}
\begin{split}
\label{eqn:1.9}
\textbf{E}(\textbf{x}) = -  \nabla \Phi
\end{split}
\end{empheq}
Where:
\begin{empheq}[box=\tcbhighmath]{equation}
\begin{split}
   \Phi = \frac{1}{4 \pi \epsilon_0} \int \frac{\rho(\textbf{x}')}{|\textbf{x}-\textbf{x}'|}d^3x'
\end{split}
\end{empheq}
For example if we consider a point charge $\rho(\textbf{x}) = q\delta(\textbf{x})$ and thus the integral is trivial and $\Phi = \frac{q}{4 \pi \epsilon_0}\frac{1}{|\textbf{x}|}$,the electric field can also be simply recovered from this $\Phi$ through equation \ref{eqn:1.9}

The equation \ref{eqn:1.9} also tells us that $\nabla \times E =0$ as $\nabla \times (\nabla \Phi) = 0$

\end{itemize}
\subsection{Work to move a charge}
\begin{itemize}
    \item The work done moving a charge from point $a$ to point $b$ is defined as:
\begin{empheq}[box=\tcbhighmath]{equation}
\begin{split}
   W = - \int _a^b \textbf{F} \cdot d \textbf{l} = -q\int_a^b \textbf{E} d \textbf{l}
\end{split}
\end{empheq}
Then subbing in \ref{eqn:1.9} we get:
\begin{empheq}[box=\tcbhighmath]{equation}
\begin{split}
   q\left[\Phi(b) - \Phi(a)\right]
\end{split}
\end{empheq}
So the work to go from $r= \infty$ to $r=r$ for a point charge is $q \Phi(r)$.
\end{itemize}
\subsection{Charged plate}
\begin{itemize}
    \item If we have a charged conducting plate, the component of the electric field that lies tangent to the plate will all ways be 0, other wise the charge would flow to make it so this was true. 
    \item However for the component that is normal to the plate there is a discontinuity as one moves from one side of the plate to the other. We can apply Gauss' law to see that:
\begin{empheq}[box=\tcbhighmath]{equation}
\begin{split}
   \oint_S \textbf{E}\cdot \hat{\textbf{n}}da =\frac{1}{\epsilon_0}\int\sigma da \\
\implies (E_{out}-E_{in}) =\frac{\sigma}{\epsilon_0}
\end{split}
\end{empheq}

\end{itemize}
\subsection{Poisson's equation}
\begin{itemize}
    \item We can apply the Divergence theorem to Gauss' law, meaning:
\begin{empheq}[box=\tcbhighmath]{equation}
\begin{split}
   \frac{1}{\epsilon_0}\int_V\rho(\textbf{x})d^3x =\oint_S \textbf{E}\cdot \hat{\textbf{n}}da = \int_V \nabla \cdot \textbf{E} d^3x 
\end{split}
\end{empheq}
And then since we know $E = - \nabla \Phi$:
\begin{empheq}[box=\tcbhighmath]{equation}
\begin{split}
\label{eqn:1.14}
\nabla \cdot \textbf{E} = \frac{\rho(\textbf{x})}{\epsilon_0} \\
\implies \nabla^2 \Phi = -\frac{\rho}{\epsilon_0}
\end{split}
\end{empheq}
In empty space $\rho=0$ so $ \nabla^2 \Phi =0$, this is known as the Laplace equation. The first equation here is Gauss' law and is one of Maxwell's equations. 

\item Applying this to the above potential for a point charge, $\Phi = \frac{q}{4 \pi \epsilon_0}\frac{1}{|\textbf{x}|}$, we see that:
\begin{empheq}[box=\tcbhighmath]{equation}
\label{eqn:1.15}
\begin{split}
\frac{q}{4 \pi \epsilon_0} \nabla^2(\frac{1}{|\textbf{x}|}) = - \frac{q}{\epsilon_0}\delta(\textbf{x}) \\
\nabla^2(\frac{1}{|\textbf{x}|}) = -4 \pi \delta(\textbf{x})
\end{split}
\end{empheq}
A known result that will be use full later.


\end{itemize}
\subsection{Greens theorems}
\begin{itemize}
    \item If we start with the divergence theorem:
\begin{empheq}[box=\tcbhighmath]{equation}
\begin{split}
  \int_V \nabla \cdot \textbf{A} d^3x = \oint_{\partial V}\textbf{A} \cdot \hat{\textbf{n}} da 
\end{split}
\end{empheq}
Then letting $A = \phi \nabla \psi $, we get Greens first identity:
\begin{empheq}[box=\tcbhighmath]{equation}
\begin{split}
\label{eqn:1.18}
\int_V(\nabla\phi \nabla \psi+ \phi \nabla^2 \psi)dV = \oint_{\partial V} \phi \nabla \psi \cdot \hat{\textbf{n}}da =\oint_{\partial V} \psi \frac{\partial \psi }{\partial \hat{\textbf{n}}} da
\end{split}
\end{empheq}
\item Greens second theorem is then found by interchanging $\phi$ for $\psi$ and then taking that quantity away from the original unchanged version, this takes the form:
\begin{empheq}[box=\tcbhighmath]{equation}
\begin{split}
   \int_V(\phi \nabla ^2 \psi - \psi \nabla^2 \phi)dV = \int_{\partial V} (\phi\frac{\partial \psi}{\partial \hat{\textbf{n}}} - \psi \frac{\partial \phi}{\partial \hat{\textbf{n}}}) da
\end{split}
\end{empheq}
\item This can then be used to find a formula for $\Phi$. Letting $\phi = \Phi $ and $\psi = \frac{1}{|\textbf{x} - \textbf{x}'|}$, Then plugging in a using relations \ref{eqn:1.14} and \ref{eqn:1.15}, results in:
\begin{empheq}[box=\tcbhighmath]{equation}
\begin{split}
   -4\pi \Phi (\textbf{x}) + \frac{1}{\epsilon_0} \int \frac{\rho(\textbf{x}')}{|\textbf{x} - \textbf{x}'|} = \oint ( \frac{1}{|\textbf{x} - \textbf{x}'|} \frac{\partial \Phi}{\partial \hat{\textbf{n}'}} - \Phi \frac{\partial }{\partial \hat{\textbf{n}'}}( \frac{1}{|\textbf{x} - \textbf{x}'|}))da'
\end{split}
\end{empheq}
And thus $\Phi $ is restrained by the master formula!:
\begin{empheq}[box=\tcbhighmath]{equation}
\label{eqn:1.21}
\begin{split}
\Phi (\textbf{x}) = \frac{1}{4\pi \epsilon_0} \int \frac{\rho(\textbf{x}')}{|\textbf{x} - \textbf{x}'|} d^3x' + \frac{1}{4\pi }\oint ( \frac{1}{|\textbf{x} - \textbf{x}'|} \frac{\partial \Phi}{\partial \hat{\textbf{n}'}} - \Phi \frac{\partial }{\partial \hat{\textbf{n}'}}( \frac{1}{|\textbf{x} - \textbf{x}'|}))da'
\end{split}
\end{empheq}
The only assumption made here is that the Volume we integrate over includes the charge otherwise the integral of the delta function would have been 0. 




\end{itemize}
\subsection{Uniqueness of solution with boundary conditions}
\begin{itemize}
    \item Suppose there are two solutions to the potential $\Phi_1$ and $\Phi_2$. We then set $U= \Phi_2-\Phi_1$. Then looking at Green's first identity \ref{eqn:1.18} we see that for $\phi = \psi =U$:
\begin{empheq}[box=\tcbhighmath]{equation}
\begin{split}
   \int_V (|\nabla U|^2 - U\nabla^2U)d^3x = \oint_S U \frac{\partial U}{\partial n}da 
\end{split}
\end{empheq}
\item It is here that we consider the boundary conditions. There are two types of boundary conditions that will lead to unique solution for $\phi$. 

\item The first is Dirichlet boundary conditions (DBC), this is essentially idea that anything connected by a conductor should have the same potential so $U=0$ on the closed surface $S$. 

\item The second is the Neumann boundary condition (NBC), which is the idea that the perpendicular component of the electric field should be the same everywhere on a surface (perhaps due to some charge density) should be the same. Then seeing as this component of the electric field is just the normal derivative of the potential; we have that $\frac{\partial U}{\partial n} = 0$. 

One can easily see that plugging either of these conditions into the above equation, along with realising that $\nabla^2 U = \frac{\rho - \rho}{\epsilon_0}$ , results in:
\begin{empheq}[box=\tcbhighmath]{equation}
\begin{split}
   \int_V (|\nabla U|^2 d^3x = 0
\end{split}
\end{empheq}
This then implies that $U$ is a constant everywhere.  Thus the two solutions may differ by a constant. Usually we set the constant such that $\Phi(\textbf{x}=\infty)=0$, but this is not all ways the case.  




\end{itemize}


\subsection{Greens functions}
\begin{itemize}
    \item The green function for the Poisson equation is such that:
\begin{empheq}[box=\tcbhighmath]{equation}
\begin{split}
   \nabla'^2 G(\textbf{x},\textbf{x}') = -4\pi\delta(\textbf{x}-\textbf{x}')
\end{split}
\end{empheq}
We saw already that a function that satisfies this is $G = \frac{1}{|\textbf{x}-\textbf{x}'|}$, so any Green function for the Poisson equation should take the form:
\begin{empheq}[box=\tcbhighmath]{equation}
\begin{split}
 G(\textbf{x},\textbf{x}') = \frac{1}{|\textbf{x}-\textbf{x}'|} + F(\textbf{x},\textbf{x}')
\end{split}
\end{empheq}
Where $F$ satisfies $\nabla'^2F=0$. 

\item The physical meaning of  $G$ is the potential $\Phi$ which a unit point charge $\textbf{x}'$ creates subject to boundary conditions.  

This function $G$ can be put instead of $\frac{1}{|\textbf{x}-\textbf{x}'|}$ in equation \ref{eqn:1.21}. Then if we impose $DBC$ the equation shortens as $G$ must be 0 on $S$: 
\begin{empheq}[box=\tcbhighmath]{equation}
\begin{split}
\Phi (\textbf{x}) = \frac{1}{4\pi \epsilon_0} \int \rho(\textbf{x}')G(\textbf{x},\textbf{x}') d^3x' - \frac{1}{4\pi }\oint\Phi \frac{\partial }{\partial \hat{\textbf{n}'}}G(\textbf{x},\textbf{x}')da'
\end{split}
\end{empheq}
Naively if we try to say the boundary condition NBC implies $\frac{\partial G}{\partial \textbf{n}'} = 0 $ on $S$. However this is not going to work as:
\begin{empheq}[box=\tcbhighmath]{equation}
\begin{split}
\oint \frac{\partial G}{\partial \textbf{n}'} da'  = \oint \nabla' G d\textbf{A}'
\end{split}
\end{empheq}
Which by divergence theorem is:
\begin{bux}
    \begin{split}
       \int \nabla'^2  G dV' = -4\pi 
    \end{split}
\end{bux}
It can also be noted that $G(\textbf{x},\textbf{x}') = G(\textbf{x}',\textbf{x})$.


\end{itemize}
\subsection{Energy Density} 
\begin{itemize}
    \item For a multi charge system the energy (work) is given by:
\begin{empheq}[box=\tcbhighmath]{equation}
\begin{split}
W = \sum_i W_i =\frac{1}{8 \pi \epsilon_0} \sum_{i \neq j}^n\frac{q_iq_j}{|\textbf{r}_i-\textbf{r}_j|}
\end{split}
\end{empheq}
\item Then for a continuous distribution:
\begin{empheq}[box=\tcbhighmath]{equation}
\begin{split}
W = \frac{1}{8 \pi \epsilon_0} \int \int \frac{\rho(\textbf{x})\rho(\textbf{x}')}{|\textbf{x}_i-\textbf{x}_j|} = \frac{1}{2}\int \rho(\textbf{x})\Phi(\textbf{x})d^3x = -\frac{\epsilon_0}{2}\int (\nabla^2 \Phi)\Phi(\textbf{x})d^3x
\end{split}
\end{empheq}
Integrating by parts we get:
\begin{empheq}[box=\tcbhighmath]{equation}
\begin{split}
W =  \frac{\epsilon_0}{2}\int |\nabla \Phi |^2d^3x =  \frac{\epsilon_0}{2}\int \textbf{E}^2d^3x
\end{split}
\end{empheq}


\end{itemize}
\subsection{Mirror image problem}
\begin{itemize}
    \item Due to how powerful the uniqueness theorem is, it does not matter how we come across our solution for the electric potential. As long as it satisfies the Poisson equation and has the correct values at the boundaries, that's it, it must be correct. 

This usefulness is clear in the method of images. Here we are usually dealing with the problem of a point charge and some boundary conducting surface that is in some way grounded. The problem then with finding the electric field is that the charge also induces a bit of opposite charge on this surface. 

To solve for the potential one introduces another so called mirror charge on the other side of the boundary. This then makes the problem much easier to solve and if the charge is placed in such a way that the boundary conditions on the surface are still met then the solution of $\Phi$ must be correct (and unique!) on the side of the boundary that has the original charge. 
\end{itemize}

\begin{itemize}
    \item If we have a charge $q$ at a distance $d$ from the origin, inside a conducting sphere of radius $a$ then the image charge needed to be placed outside this sphere as an image charge must have a charge $q'$ at a distance $d'$ from the origin. The case when these charges are swapped results in the same equations.
\begin{bux}
    \begin{split}
        q' = -q \frac{a}{d},~~~d' = \frac{a^2}{d}
    \end{split}
\end{bux}
\end{itemize}

    \subsection{Separation of variables}
\begin{itemize}
    \item The Laplace equation in rectangular coordinates is:
\begin{bux}
    \begin{split}
        \frac{\partial^2\Phi}{\partial x^2}+ \frac{\partial^2\Phi}{\partial y^2}+ \frac{\partial^2\Phi}{\partial z^2} =0
    \end{split}
\end{bux}
A solution can be found by assuming the solution is separable, that is:
\begin{bux}
    \begin{split}
        \Phi(x,y,z) = X(x)Y(y)Z(z)
    \end{split}
\end{bux}
Thus subbing in and dividing across by $\Phi$:
\begin{bux}
    \begin{split}
         \frac{1}{X}\frac{\partial^2X}{\partial x^2}+ \frac{1}{Y}\frac{\partial^2Y}{\partial y^2}+ \frac{1}{Z}\frac{\partial^2Z}{\partial z^2} =0
    \end{split}
\end{bux}
We can vary any of $x,y$ or $z$ and this equation should still be $0$ so each of these terms must be constants, so we write:
\begin{bux}
    \begin{split}
      &   \frac{1}{X}\frac{\partial^2X}{\partial x^2} = -\alpha^2 \\
&\frac{1}{Y}\frac{\partial^2Y}{\partial y^2} = -\beta^2 \\
& \frac{1}{Z}\frac{\partial^2Z}{\partial z^2} =\gamma^2
    \end{split}
\end{bux}
So $\alpha^2+\alpha^2 = \gamma^2$. From these equations the general solution is:
\begin{bux}
    \begin{split}
        \Phi = (A_1e^{ i \alpha x}+A_2e^{ -i \alpha x})(B_1e^{ i \beta y}+B_2e^{- i \beta y})(C_1e^{ \gamma z}+C_2e^{ -\gamma z})
    \end{split}
\end{bux}
Boundary terms are need to refine this large class of solutions any further.  For example consider a conducting cuboid if we have boundary conditions such as $\Phi=0$ when $x,y,z=0$, as well as $x=a$ and $y=b$, then the constants can take many values depending on $n,m \in \mathbb{N}$:
\begin{bux}
    \begin{split}
        & \alpha_n = \frac{n \pi }{a}  \\
& \beta_m = \frac{m \pi }{b} \\
& \gamma_{n,m} = \pi \sqrt{\frac{n^2}{a^2}+\frac{m^2}{b^2}}
    \end{split}
\end{bux}

Since Laplace's equation is linear any linear combination these solutions is also a solution. Exploiting this fact we can patch together these solution's to fit any potential $\Phi(z=c)=V(x,y)$ on the final boundary.  The solution then looks like: 
\begin{bux}
    \begin{split}
        \Phi(x,y,z) = \sum_{n,m=1}^{\infty}A_{nm}\sin(  \frac{n \pi }{a} x)\sin(\frac{m \pi }{b} y)\sinh(\pi \sqrt{\frac{n^2}{a^2}+\frac{m^2}{b^2}} z)
    \end{split}
\end{bux} 
The coefficients $A_{nm}$ can be solved for using the orthogonality of sin. Multiplying both sides of the above equation by $\sin(\frac{n'\pi}{a}x)\sin(\frac{m'\pi}{b}y)dxdy$ and then integrating from $0$ to $a$ and from $0$ to $b$ across the surface $z=c$.  The LHS is:
\begin{bux}
    \begin{split}
     &  \int_0^a\int_0^b \Phi(x,y,c)\sin(\frac{n'\pi}{a}x)\sin(\frac{m'\pi}{b}y)dxdy \\  = &\int_0^a\int_0^b V(x,y)\sin(\frac{n'\pi}{a}x)\sin(\frac{m'\pi}{b}y)dxdy 
    \end{split}
\end{bux} 
And the RHS is:
\end{itemize}
\begin{bux}
    \begin{split}
    &\int_0^a\int_0^b  \sum_{n,m=1}^{\infty}A_{nm}\sin(  \frac{n \pi }{a} x)\sin(\frac{m \pi }{b} y)\sinh(\pi \sqrt{\frac{n^2}{a^2}+\frac{m^2}{b^2}} c)\sin(\frac{n'\pi}{a}x)\sin(\frac{m'\pi}{b}y)dxdy \\
&  = A_{n'm'} \frac{ab}{4}\sinh(\pi \sqrt{\frac{n'^2}{a^2}+\frac{m'^2}{b^2}})
    \end{split}
\end{bux} 
\begin{itemize}
\item Where here we have used the fact that: 
\begin{bux}
    \begin{split}
        \int_0^a\sin(\frac{n\pi}{a}x)\sin(\frac{n\pi}{a}x)dx = \frac{a}{2}\delta_{nn'}
    \end{split}
\end{bux}
Thus we can say to solve for the coefficients we simply need to compute: 
\end{itemize}
\begin{bux}
    \begin{split}
        A_{nm} = \frac{4}{ab\sinh(\pi \sqrt{\frac{n^2}{a^2}+\frac{m^2}{b^2}})}\int_0^a\int_0^b V(x,y)\sin(\frac{n\pi}{a}x)\sin(\frac{m\pi}{b}y)dxdy
    \end{split}
\end{bux}
\subsection{Spherical coordinates }
\begin{itemize}
    \item The Laplacian in spherical coordinates can be written as follows: 
\begin{bux}
    \begin{split}
        \nabla^2 \Phi = \frac{1}{r}\frac{\partial^2}{\partial r^2}(r\Phi) + \frac{1}{r^2\sin(\theta)} \frac{\partial}{\partial \theta}(\sin\theta \frac{\partial \Phi}{\partial \theta}) + \frac{1}{r^2\sin^2(\theta)}\frac{\partial ^2 \Phi}{\partial \phi^2}=0
    \end{split}
\end{bux}
So we can once again assume the coordinates are separable that is:
\begin{bux}
    \begin{split}
        \Phi= \frac{U(r)}{r}P(\theta)Q(\phi)
    \end{split}
\end{bux}
The $\frac{U(r)}{r}$ is simply so that this subs into the Laplacian nicely.  Subbing in a multiplying across by $\frac{r^2\sin^2\theta}{\Phi}$: 
\begin{bux}
    \begin{split}
        r^2\sin^2\theta \left[\frac{1}{U}\frac{d^2 U}{d r^2}+ \frac{1}{Pr^2\sin\theta} \frac{d}{d \theta}(\sin \theta \frac{dP}{d\theta})\right] + \frac{1}{Q}\frac{d^2Q}{d \phi^2} = 0
    \end{split}
\end{bux}
Having separated the $\phi$ terms we can set $ \frac{1}{Q}\frac{d^2Q}{d \phi^2} = -m^2$ a constant. Thus $Q$ has the solutions $Q = e^{\pm im\phi}$, with $m \in \mathbb{Z}$ as we want $Q$ to be single valued, so $\Phi(\phi+ 2\pi) = \Phi(\phi)$.   Similarly we can write:
\begin{bux}
    \begin{split}
        r^2\frac{1}{U}\frac{d^2 U}{d r^2}+ \frac{1}{P\sin\theta} \frac{d}{d \theta}(\sin \theta \frac{dP}{d\theta}) + \frac{1}{Q\sin^2\theta}\frac{d^2Q}{d \phi^2} = 0
    \end{split}
\end{bux}
So we can set $ r^2\frac{1}{U}\frac{d^2 U}{d r^2} =l(l+1)$, a constant. This weird choice of $l(l+1) \in \mathbb{R}$  is just so that the solution to this differential has nice powers, as it can easily seen that the solution to this differential is just: 
\begin{bux}
\label{eqn:1.48}
    \begin{split}
       &\frac{d^2 U}{d r^2} =\frac{l(l+1)U}{r^2}, \\ 
 \implies& U = Ar^{l+1} + B r^{-l} \\
 \implies &\frac{U}{r} = Ar^{l} + B \frac{1}{r^{l+1}}
    \end{split}
\end{bux}
Finally for $\theta$ we can write the Laplacian as: 
\begin{bux}
    \begin{split}
        \frac{1}{\sin\theta} \frac{d}{d \theta}(\sin \theta \frac{dP}{d\theta})  + \left[ l(l+1)  - \frac{m^2}{\sin ^2 \theta }\right]P = 0
    \end{split}
\end{bux} 
Which if we let $x= \cos \theta$ takes the following form:
\begin{bux}
    \begin{split}
         \frac{d}{dx}\left[(1-x^2) \frac{dP}{dx}\right]  + \left[ l(l+1)  - \frac{m^2}{1-x^2}\right]P = 0
    \end{split}
\end{bux}
This equation is called the generalised Legendre equation.


\end{itemize}

\subsection{Azimuthal symmetry} 
\begin{itemize}
    \item If there is azimuthal symmetry (i.e. the solution $\Phi$ is independent of $\phi$) then $m=0$ and the Legendre equation reduces to: 
\begin{bux}
    \begin{split}
         \frac{d}{dx}\left[(1-x^2) \frac{dP}{dx}\right]  + l(l+1)P = 0
    \end{split}
\end{bux}
This equation can be solved by method of series solution (particularly method of Frobenius ) with the only convergent solution being valid in the range $-1< x < 1$. This can be extended to $-1\leq x \leq 1$ if the series solution terminates after some point.  From the recursion relation for the series solution this can be shown to be the case only if $l\in \mathbb{N}^{\ast}$. The solutions found that satisfy these conditions are the Legendre polynomials.  
\end{itemize}
\subsubsection{Legendre Polynomials}

\begin{itemize}
\item These polynomials can be found from Rodrigues' formula:
\begin{bux}
    \begin{split}
        P_l(x) = \frac{1}{2^ll!}\frac{d^l}{dx^l}(x^2-1)^l
    \end{split}
\end{bux}
They also have the useful properties : 
\begin{bux}
    \begin{split}
     &  P_l(x) = \frac{1}{2l+1}\frac{d}{dx}(P_{l+1}(x) - P_{l-1}(x)) \\
& P_l(-x) = (-1)^lP_l(x) \\
&P_l(1) =1
    \end{split}
\end{bux}
\item The Legendre polynomials are also orthogonal allowing us to expand $\Phi$ over these polynomials and solve for the coefficients by integrating just like we did for Cartesian co-ords. The polynomials are orthogonal so:
\begin{bux}
    \begin{split}
        \int_{-1}^1P_l(x)P_{l'}(x)dx = \frac{2}{2l+1}\delta_{ll'}
    \end{split}
\end{bux}
We can also write regular polynomials in terms of the Legendre polynomials. This fact becomes particularly use full when dealing with problems where the potential on a sphere is held to be in terms of powers of $\cos \theta$ as then the substitution $x=\cos \theta$ leads to powers of $x$ which can be turned into Legendre polynomials and make the integrals found below in \ref{eqn:1.56} and \ref{eqn:1.57} easy to solve. 
\item A list of the Legendre polynomials and other polynomials in terms of them are shown below
\end{itemize}

\begin{table}[H]
    \centering
    \begin{tabular}{|c|c|c|c|} \hline 
         $n$&  $P_n(x)$&  $Q(x)$& $Q(P_n)$\\ \hline 
         0& $1$& 1&$P_0$\\\hline
         1&  $x$&  $x$& $P_1$\\ \hline 
         2&  $\frac{1}{2}(3x^2-1)$&  $x^2$& $\frac{1}{3}(P_0+2P_2)$\\ \hline 
         3&  $\frac{1}{2}(5x^3-3x)$&  $x^3$& $\frac{1}{5}(3P_1+2P_3)$\\ \hline 
         4&  $\frac{1}{8}(35x^4-30x^2+3)$&  $x^4$& $\frac{1}{35}(7P_0+20P_2+8P_4)$\\\hline
    \end{tabular}
    \caption{Legendre polynomials}
    \label{tab:1}
\end{table}


\subsubsection{Solutions for the potential }
\begin{itemize}
\item The full form of the solution in spherical co-ords can be written as: 
\begin{bux}
    \begin{split}
        \Phi(r,\theta) = \sum_{l=0}^{\infty}[A_lr^{l} + B_l \frac{1}{r^{l+1}}]P_l(\cos \theta)
    \end{split}
\end{bux}
If we have conducting hollow sphere of radius $R$, with a potential $V(\theta)$, we can solve for the coefficients. Inside the sphere in order for the solution to not blow up at $r=0$, $B_l=0,~~\forall~l$.  So by similar method to above we multiply by $P_{l'}\sin \theta d \theta$ and integrate from $0$ to $\pi$, from which we find: 
\begin{bux}
    \begin{split}
\label{eqn:1.56}
        A_l = \frac{2l+1}{2R^{l}} \int_0^{\pi} V(\theta)P_l(\cos \theta) \sin \theta d\theta
    \end{split}
\end{bux}
Similarly outside the sphere $A_l=0,~~\forall~l$ so that the solution does not blow up as $r \rightarrow \infty$.  Then the coefficients are: 
\begin{bux}
    \begin{split}
\label{eqn:1.57}
        B_l = R^{l+1}\frac{2l+1}{2} \int_0^{\pi} V(\theta)P_l(\cos \theta) \sin \theta d\theta
    \end{split}
\end{bux}
\end{itemize}
 
\subsection{Point charge with spherical co-ords}
\begin{itemize}
    \item If we have a unit point charge at $\textbf{x}'$ then the potential at $\textbf{x}$ is $\Phi(\textbf{x})$, where:
\begin{bux}
    \begin{split}
        4 \pi \epsilon_0 \Phi =  \frac{1}{|\textbf{x}-\textbf{x}'|} =  \sum_{l=0}^{\infty}[A_lr^{l} + B_l \frac{1}{r^{l+1}}]P_l(\cos \gamma)
    \end{split}
\end{bux}
Where here $\gamma$ is the angle between the two vectors $\textbf{x}$ and $\textbf{x}'$. Then denoting $r_> \equiv max(r,r')$ and $r_< \equiv min(r,r')$, where $r =|\textbf{x}|$ and  $r' =|\textbf{x}'|$.  If we then rotate the $z$ axis so that $\textbf{x}'$ lies along it, then $\cos \gamma =1$ and:
\begin{bux}
    \begin{split}
         \frac{1}{|\textbf{x}-\textbf{x}'|} = \frac{1}{|r-r'|} = \frac{1}{r_>-r_<} = \frac{1}{r_>}\frac{1}{1- \frac{r_<}{r_>}} = \frac{1}{r_>}\sum_{l=0}^{\infty} \left(\frac{r_<}{r_>}\right)^l
    \end{split}
\end{bux}
Then by our ansatz of separability of co-ords, to put back in the dependence on $\gamma$ we can just multiply each term by $P_l(\cos \gamma)$.  Thus the full solution is:
\begin{bux}
    \begin{split}
    \Phi= \frac{1}{ 4 \pi \epsilon_0 } \frac{1}{|\textbf{x}-\textbf{x}'|} =  \frac{1}{ 4 \pi \epsilon_0 } \frac{1}{r_>}\sum_{l=0}^{\infty} \left(\frac{r_<}{r_>}\right)^lP_l(\cos \gamma)
    \end{split}
\end{bux}
\end{itemize}


\subsection{No Azimuthal symmetry}
\begin{itemize}
    \item If we add back in $\phi$ dependence, it can be shown in a similar manner to the case when $m$ was $0$, that in the range $-1 \leq x \leq 1$ , $l$ must be a non-negative integer and $m$ must take only the values $ -l , -(l-1), ...,0,...,l-1,l$.  The solutions to the generalised Legendre equation are called the associated Legendre functions $P_l^m(x)$. For positive $m$ they are defined by the formula:
\begin{bux}
    \begin{split}
        P_l^m(x) = (-1)^m(1-x^2)^{\frac{m}{2}}\frac{d^m}{dx^m}P_l(x)
    \end{split}
\end{bux}
Thus Rodrigues' formula for these functions is:
\begin{bux}
    \begin{split}
          P_l^m(x) = \frac{(-1)^m}{2^ll!}(1-x^2)^{\frac{m}{2}}\frac{d^{m+l}}{dx^{m+l}}(x^2-1)^l
    \end{split}
\end{bux}
From this we can then see that we can get the negative $m$ functions from the relation:
\begin{bux}
    \begin{split}
          P_l^{-m}(x) = (-1)^m\frac{(l-m)!}{(l+m)!}P_l^m(x)
    \end{split}
\end{bux} 
These functions are also orthogonal: 
\begin{bux}
    \begin{split}
         \int_{-1}^1P_l^m(x)P_{l'}^m(x)dx = \frac{2}{2l+1}\frac{(l+m)!}{(l-m)!}\delta_{ll'}
    \end{split}
\end{bux}
It is then convenient to combine both of the angular factors and construct orthonormal functions over the unit sphere These functions are spherical harmonics. Since both $e^{im\phi}$ for $0 \leq \phi \leq 2 \pi$ and $P_l(x)$ for $-1 \leq x \leq 1$ are orthogonal then so is their product. So we define the orthonormal spherical harmonics $Y_{lm}(\theta, \phi)$ by:
\begin{bux}
    \begin{split}
        Y_{lm}(\theta, \phi) = \sqrt{\frac{2l+1}{4 \pi}\frac{(l-m)!}{(l+m)!}}P_l^m(\cos \theta)e^{im\phi}
    \end{split}
\end{bux}
We can also see from this that: 
\begin{bux}.,
    \begin{split}
        Y_{l,-m}(\theta, \phi) = (-1)^mY^{\ast}_{lm}(\theta, \phi)
    \end{split}
\end{bux}
\item The full solution to the potential in spherical co-ords can then be written as:
\begin{bux}
\label{eqn:1.67}
    \begin{split}
        \Phi(r,\theta,\phi) = \sum_{l=0}^{\infty}\sum_{m=-l}^{l}[A_lr^{l} + \frac{ B_l}{r^{l+1}}]Y_{lm}(\theta,\phi)
    \end{split}
\end{bux}
The coefficients of this expansion can be found the same way it has been done above. It should be noted that since the spherical harmonics are complex functions so in order to exploit orthogonality we need to multiply by $Y^{\ast}_{lm}(\theta, \phi)$. These functions are also orthonormal meaning there is no factor once you exploit orthogonality. 

\end{itemize}
\subsection{Completeness relation}
\begin{itemize}
    \item For the expansion of any function $f(x)$ on the interval $(a,b)$ over any set of orthonormal function $U_n(x)$, $f(x)$ is written as: 
\begin{bux}
    \begin{split}
 &       f(x) = \sum_{n=1}^{\infty} a_nU_n(x), \text{with} \\
&   a_n = \int_a^bU_n^{\ast}(x) f(x)dx
    \end{split}
\end{bux} 
To have this work we should then have that: 
\begin{bux}
    \begin{split}
        f(x) = \sum_{n=1}^{\infty}\left( \int_a^bU_n^{\ast}(x') f(x')dx'\right)U_n(x) =  \int_a^b\left(\sum_{n=1}^{\infty}U_n^{\ast}(x') U_n(x)\right)f(x')dx'
    \end{split}
\end{bux} Thus:
\begin{bux}
    \begin{split}
        \sum_{n=1}^{\infty}U_n^{\ast}(x') U_n(x) = \delta(x-x')
    \end{split}
\end{bux} 
This is the completeness relation.
\end{itemize}


\subsection{Addition theorem for spherical harmonics} 
\begin{itemize}
    \item This is essentially just the expansion of $ \frac{1}{|\textbf{x}-\textbf{x}'|}$ in spherical harmonics. The result is: 
\begin{bux}
\label{eqn:1.68}
    \begin{split}
         \frac{1}{|\textbf{x}-\textbf{x}'|} = 4 \pi \sum_{l=0}^{\infty}\sum_{m=-l}^{l}\frac{1}{2l+1}\frac{1}{r_>}\left(\frac{r_<}{r_>}\right)^lY_{lm}(\theta,\phi)Y_{lm}^{\ast}(\theta',\phi')
    \end{split}
\end{bux}

\end{itemize}

\subsection{Expansion of Greens functions in spherical co-ords}
\begin{itemize}
    \item If we introduce a conducting sphere of radius $a$, then we know from the method of images that the Greens function for this scenario is:
\begin{bux}
    \begin{split}
        G(\textbf{x},\textbf{x}') = \frac{1}{|\textbf{x}-\textbf{x}'|} - \frac{a}{r'|\textbf{x}-\frac{a^2}{r'^2}\textbf{x}'|}
    \end{split}
\end{bux}
We now wish to find out how this greens function expands on the exterior of the sphere, that is $r,r'>a$.  To do this we expand the above greens functions using \ref{eqn:1.68}:
\end{itemize}
\begin{bux}
    \begin{split}
       G(\textbf{x},\textbf{x}') =  4 \pi \sum_{l=0}^{\infty}\sum_{m=-l}^{l}\frac{1}{2l+1}\left[ \frac{r_<^l}{r_>^{l+1}}-\frac{1}{a}\left( \frac{a^2}{rr'}\right)^{l+1}\right]Y_{lm}(\theta,\phi)Y_{lm}^{\ast}(\theta',\phi')
    \end{split}
\end{bux}
\begin{itemize}
    \item One can note that for either $r$ or $r'$ equal to $a$, the radial factor vanishes, same case for either $r,r' \rightarrow \infty$. 
\end{itemize}


\subsection{Delta functions in spherical co-ords}
\begin{itemize}
    \item To do this we can simply by change the co-ords of the delta function while it is inside an integral picks up the factors of the Jacobin. Thus the delta function in spherical co-ords is:
\begin{bux}
    \begin{split}
        \delta(\textbf{x}-\textbf{x}') = \frac{1}{r^2}\delta(r-r')\delta(\phi-\phi')\delta(\cos \theta - \cos \theta ')
    \end{split}
\end{bux}

\end{itemize}

\subsection{Alternate definition of spherical harmonics}
\begin{itemize}
    \item First we need to look at the Laplacian in spherical co-ords in terms of spherical harmonics, we can rewrite the Laplacian as: 
\begin{bux}
    \begin{split}
       &   \nabla^2 \Phi = \frac{1}{r}\frac{\partial^2}{\partial r^2}(r\Phi) + \frac{1}{r^2}\left[\frac{1}{\sin(\theta)} \frac{\partial}{\partial \theta}(\sin\theta \frac{\partial \Phi}{\partial \theta}) + \frac{1}{\sin^2(\theta)}\frac{\partial ^2 \Phi}{\partial \phi^2}\right] \\ 
& \equiv  \frac{1}{r}\frac{\partial^2}{\partial r^2}(r\Phi) + \frac{1}{r^2}\nabla^2_{\theta,\phi}\Phi =0
    \end{split}
\end{bux}
Here we can once again apply separability so $\Phi = \frac{U(r)}{r}Y(\theta,\phi)$. We all ready know the solution for $U(r)$, via \ref{eqn:1.48} So by imposing that both above terms must be constants, and the $r$ dependant term is equal to $l(l+1)$:
\begin{bux}
    \begin{split}
     & r^2\frac{1}{U}\frac{\partial^2}{\partial r^2}(U) + \frac{1}{Y}\nabla^2_{\theta,\phi}Y  = l(l+1) + \frac{1}{Y}\nabla^2_{\theta,\phi}Y  = 0 \\
& \implies \nabla^2_{\theta,\phi}Y_{lm} = -l(l+1)Y_{lm}
    \end{split}
\end{bux}
Where in the last line we can identify the $Y(\theta ,\phi)$ as the spherical harmonics, as they are the only terms containing $\theta$ and $\phi$, so they must be the same as the $Y_{lm}$'s in equation \ref{eqn:1.67}. 
\end{itemize}



\subsection{Greens function between two spherical shells}
\begin{itemize}
    \item Returning to Greens functions, if we have a Greens function subject to a Dirichlet boundary condition then:
\begin{bux}
    \begin{split}
        \nabla^2_{\textbf{x}} G(\textbf{x},\textbf{x}') = -4 \pi \delta(\textbf{x} - \textbf{x}')
    \end{split}
\end{bux}
With $G(\textbf{x},\textbf{x}')=0$ for either $\textbf{x}$ or $\textbf{x}'$ on the boundary surface. We may now expand $G(\textbf{x},\textbf{x}')$ in terms of the spherical harmonics: 
\begin{bux}
    \begin{split}
        G(\textbf{x},\textbf{x}') =  \sum_{l=0}^{\infty}\sum_{m=-l}^{l}A_{lm}(r;r',\theta',\phi')Y_{lm}(\theta,\phi)
    \end{split}
\end{bux}
Then applying the Laplacian we must have that:
\begin{bux}
    \begin{split}
           \sum_{l=0}^{\infty}\sum_{m=-l}^{l}\left[ \frac{1}{r}\frac{d^2(rA_{lm})}{dr^2}- \frac{l(l+1)}{r^2}A_{lm}\right]Y_{lm}(\theta,\phi)=-4 \pi \delta(\textbf{x} - \textbf{x}')
    \end{split}
\end{bux}
Where here we have used $\nabla^2_{\theta,\phi}Y_{lm} = -l(l+1)Y_{lm}$.  The completeness relation for $Y_{lm}$ allow us to write: 
\begin{bux}
    \begin{split}
          \delta(\textbf{x}-\textbf{x}') = \frac{1}{r^2}\delta(r-r')\sum_{l=0}^{\infty}\sum_{m=-l}^{l}Y^{\ast}_{lm}(\theta',\phi')Y_{lm}(\theta,\phi)
    \end{split}
\end{bux} So we may infer that:
\begin{bux}
\label{eqn:1.81}
    \begin{split}
        \frac{1}{r}\frac{d^2(rA_{lm})}{dr^2}- \frac{l(l+1)}{r^2}A_{lm} =-\frac{4 \pi}{r^2}\delta(r-r')Y^{\ast}_{lm}(\theta',\phi')
    \end{split}
\end{bux} So we may use the Ansatz $A_{lm} = g_l(r,r')Y^{\ast}_{lm}(\theta',\phi')$.  This leaves the following restriction on $g_l(r,r')$:
\begin{bux}
    \begin{split}
        \frac{d^2(rg_l)}{dr^2}=  \frac{l(l+1)}{r}g_{l}
    \end{split}
\end{bux}
This is just the same equation we had for $U(r)/r$ so: 
\begin{bux}
   \begin{split}
     &   g_l(r,r') = Ar^l + Br^{-(l+1)},~~~r<r' \\
     & g_l(r,r') = Cr^l + Dr^{-(l+1)} ~~~,r>r'
   \end{split}
\end{bux}
Now we can consider the case of two concentric spheres of radius $a$ and $b$ with $a<b$. With these conditions $g_l$ must vanish for $r=a$ or $r=b$, so it becomes: 
\begin{bux}
   \begin{split}
     &   g_l(r,r') = A\left(r^l - \frac{a^{2l+1}}{r^{l+1}}\right),~~~r<r' \\
     & g_l(r,r') = B\left(\frac{1}{r^{l+1}} - \frac{r^l}{b^{2l+1}}\right) ~~~,r>r'
   \end{split}
\end{bux}
Then if we require that $r$ and $r'$ can be interchanged:
\begin{bux}
    \begin{split}
        g_l(r,r') = C\left(r_<^l - \frac{a^{2l+1}}{r_<^{l+1}}\right)\left(\frac{1}{r_>^{l+1}} - \frac{r_>^l}{b^{2l+1}}\right)
    \end{split}
\end{bux}
The last constant can be solved by normalising the above equation to the delta function. So if we multiply \ref{eqn:1.81} by $r$ and then integrate from $r'-\epsilon$ to $r'+\epsilon$, this will result in us losing the second term in \ref{eqn:1.81} as it is continuous so should vanish when we take the limit $\epsilon \rightarrow 0$. The first term will not vanish as for $r'- \epsilon$, then  $r_> = r,$ where as $r'+\epsilon$ then $r_> = r$. So: 
\begin{bux}
    \begin{split}
        \lim_{\epsilon \rightarrow 0} \left[\frac{d}{dr}(rg_l) \right]^{r'+\epsilon}_{r'-\epsilon} = -\frac{4 \pi}{r'}
    \end{split}
\end{bux}
This results in: 
\begin{bux}
    \begin{split}
        C = \frac{4 \pi}{(2l+1)\left[1-\left(\frac{a}{b}\right)^{2l+1} \right]}
    \end{split}
\end{bux}
Thus the full greens function expansion is:
\end{itemize}
\begin{bux}
    \begin{split}
        G(\textbf{x},\textbf{x}') =  4 \pi \sum_{l,m=0}^{\infty} \frac{Y^{\ast}_{lm}(\theta',\phi')Y_{lm}(\theta,\phi)}{(2l+1)\left[1-\left(\frac{a}{b}\right)^{2l+1} \right]}\left(r_<^l - \frac{a^{2l+1}}{r_<^{l+1}}\right)\left(\frac{1}{r_>^{l+1}} - \frac{r_>^l}{b^{2l+1}}\right)
    \end{split}
\end{bux}
\begin{itemize}
    \item It should be noted that while the above method is sure to give the right answer, the process is very tedious and if possible one should try to use an ansatz instead. For example if we have a system of two concentric spheres that are held at potentials containing powers of $\cos \theta$ lets say $\cos^2 \theta$ and $\cos \theta$, then as discussed before we can use the expression of powers of $x$ in terms of the polynomials in table \ref{tab:1} to know the solution expressed in terms of Legendre polynomials will only have $l=0,1,2$. From here the full solution can be obtained from applying the boundary conditions and solving for the coefficients. 
\end{itemize}

\subsection{Cylindrical co-ords }
\begin{itemize}
    \item The Laplacian in cylindrical co-ords $(\rho,\phi,z$ is:
\begin{bux}
    \begin{split}
        \nabla^2\Phi = \frac{\partial^2\Phi}{\partial \rho^2} + \frac{1}{\rho}\frac{\partial\Phi}{\partial \rho} + \frac{1}{\rho^2}\frac{\partial^2\Phi}{\partial \phi^2} + \frac{\partial^2\Phi}{\partial z^2} =0
    \end{split}
\end{bux}
To solve this we apply the usual separation of variables $\Phi(\rho,\phi,z) = R(\rho)Q(\phi)Z(z)$.  Subbing this in and dividing across by $\Phi$ we see:
\begin{bux}
    \   \begin{split}
       \frac{1}{R} \frac{\partial^2R}{\partial \rho^2} + \frac{1}{\rho R}\frac{\partial R}{\partial \rho} + \frac{1}{\rho^2 Q}\frac{\partial^2Q}{\partial \phi^2} + \frac{1}{Z}\frac{\partial^2Z}{\partial z^2} = 0
    \end{split}
\end{bux} 
We must have that this is true for all  $\rho,\phi$ and $z$. This leads to the following equations: 
\begin{bux}
    \begin{split}
        & \frac{\partial^2Z}{\partial z^2} = k^2Z  \\ 
        &  \frac{\partial^2Q}{\partial \phi^2} = -\nu^2Q \\
  \frac{\partial^2R}{\partial \rho^2}& + \frac{1}{\rho }\frac{\partial R}{\partial \rho} + \left( k^2 - \frac{\nu^2}{\rho^2}\right)R= 0
    \end{split}
\end{bux}
Here $k$ and $\nu$ are constants. The first two equations are as Jackson lightly puts it "elementary": 
\begin{bux}
    \begin{split}
        Z(z) = e^{\pm kz} \\ 
        Q(\phi) = e^{\pm i \nu \phi } 
    \end{split}
\end{bux}
As argued for the spherical co-ordinate case for the potential to be single valued $\nu$ must be an integer, but $k$ is, barring boundary conditions is arbitrary.  The radial equation can expressed with $x = k \rho $: 
\begin{bux}
    \begin{split}
        \frac{d^2R}{dx^2} + \frac{1}{R}\frac{dR}{dx} + \left( 1- \frac{\nu^2}{x^2}\right)R = 0
    \end{split}
\end{bux}
This is solved again via method of Frobenius and the solutions are called the Bessel functions of order $\nu$: 
\begin{bux}
    \begin{split}
      &  J_{\nu}(x) = \left(\frac{x}{2}\right)^{\nu} \sum_{j=0}^{\infty} \frac{(-1)^{j}}{j!\Gamma(j+\nu+1)}\left(\frac{x}{2}\right)^{2j}   \\ 
& J_{-\nu}(x) = \left(\frac{x}{2}\right)^{-\nu} \sum_{j=0}^{\infty} \frac{(-1)^{j}}{j!\Gamma(j-\nu+1)}\left(\frac{x}{2}\right)^{2j}  
    \end{split}
\end{bux}
These two are linear independent for $\nu \notin \mathbb{Z}$, hence the two seemingly same equations. If $\nu = m$ is an integer then it can be shown from the series solution obtained from the above differential equation, that $J_{-m}(x) = (-1)^mJ_m(x)$, So customarily the functions $J_{\nu}(x)$ and $N_{\nu}(x)$ are used where $N_{\nu}(x)$ is defined as: 
\begin{bux}
    \begin{split}
        N_{\nu}(x) = \frac{J_{\nu}(x)\cos \nu \pi - J_{-\nu}(x)}{\sin \nu \pi }
    \end{split}
\end{bux}
This $N_{\nu}(x)$ is called the Neumann function or the Bessel function of the second kind and linear independent of $J_{\nu}(x)$ for non-integer $\nu$ and in the limit as $\nu \rightarrow$ integer.  There are also the Bessel functions of the third kind called the Hankel functions, which are just a linear combination of  $J_{\nu}(x)$ and $N_{\nu}(x)$: 
\begin{bux}
    \begin{split}
        & H^{(1)}_{\nu}(x) = J_{\nu}(x) + iN_{\nu}(x) \\
        & H^{(2)}_{\nu}(x) = J_{\nu}(x) - iN_{\nu}(x)
    \end{split}
\end{bux}
Which also form a fundamental set of solutions. 
\item The next step would be to proceed how we did with spherical and find a set of orthogonal functions so that $\Phi$ can be expanded over them. This is rather tedious process but the result is as follows. Assuming the Bessel functions are complete, we can expand any arbitrary function of $\rho$ on the interval $0 \leq \rho \leq a$ in a Fourier-Bessel series: 
\begin{bux}
    \begin{split}
        f(\rho) = \sum_{n=1}^{\infty} A_{\nu n} J_{\nu}\left( x_{\nu n}\frac{\rho}{a}\right) 
    \end{split}
\end{bux}
Here $\nu \geq 0 $ is fixed and $x_{\nu n}$ are the roots of this Bessel function, $J_{\nu}(x_{\nu n} )=0$ for $n=1,2,...$, of which there are infinitely many.  The coefficients $A_{\nu n}$ can be found from:
\begin{bux}
    \begin{split}
        A_{\nu n} = \frac{2}{a^2 J^2_{\nu +1}(x_{\nu n})}\int_0^a\rho f(\rho) J_{\nu}\left( x_{\nu n}\frac{\rho}{a}\right) d \rho
    \end{split}
\end{bux}
The $\rho$ in the integral here means that $\sqrt{\rho}J_{\nu}\left( x_{\nu n}\frac{\rho}{a}\right) $ for fixed $\nu \geq 0$ are the orthogonal functions. 

\end{itemize}

\newpage
\section{Magneto-statics}
\subsection{Conservation of charge} 
\begin{itemize}
    \item A current corresponds to a charge in motion and is described by a current density $\textbf{J}$.  This is measured in units of positive charge crossing unit area per unit time.   If we have a volume $V$, with a total charge $Q$, then by definition of $\textbf{J}$ the rate of change of the total charge $Q$ (i.e. the current $I$) is: 
\begin{bux}
    \begin{split}
\label{eqn:2.1}
        \frac{dQ}{dt} = -\oint_S\textbf{J} \cdot d\textbf{A} = - \int_V \nabla \cdot \textbf{J} d^3x
    \end{split}
\end{bux}
Here $S = \partial V$, and the last step comes from the divergence theorem. There is also a minus sign as the total charge of the volume is decreasing. We also know that the total charge $Q$ of the system is related to the charge density $\rho$ by: 
\begin{bux}
    \begin{split}
        Q = \int_V \rho(\textbf{x},t) d^3x
    \end{split}
\end{bux}
So by \ref{eqn:2.1}: 
\begin{bux}
    \begin{split}
\label{eqn:2.3}
        \frac{\partial \rho }{\partial t} + \nabla\cdot \textbf{J} = 0
    \end{split}
\end{bux}
In the case of magneto-statics we have no change in the net charge density anywhere in space, this means $\frac{\partial \rho }{\partial t}=0$, so $\nabla\cdot \textbf{J} = 0$. 
\end{itemize}

\subsection{Gauss' Law for magnetism}
\begin{itemize}
    \item From experimental results it can be noted that the magnetic flux density $d\textbf{B}$ at the position $\textbf{x}$, is related to the wire element $d \textbf{l}$ carrying a current $I$ by: 
\begin{bux}
    \begin{split}
        d\textbf{B} \propto I \frac{d\textbf{l} \times \textbf{x}}{|\textbf{x}|^3}
    \end{split}
\end{bux}
Recognising the current density $\textbf{J} = I d\textbf{l}$ and integrating we get the expression for the magnetic field $\textbf{B}(\textbf{x})$: 
\begin{bux}
    \begin{split}
\label{eqn:2.5}
        \textbf{B}(\textbf{x}) = \frac{\mu_0}{4 \pi} \int \textbf{J}(\textbf{x}') \times \frac{\textbf{x}-\textbf{x}'}{|\textbf{x}-\textbf{x}'|^3}d^3x'
    \end{split}
\end{bux}
Here we can see $\frac{\mu_0}{4 \pi}$ is the constant of proportionality. This is the analogue to \ref{eqn:1.4}, the expression of the electric field.  In the same manner then we can recognise $   \nabla \frac{1}{|\textbf{x} - \textbf{x}'|} = -\frac{\textbf{x}- \textbf{x}'}{|\textbf{x}-\textbf{x}'|^3}$. The minus sign in this expression allows us to swap the terms in the cross product and write magnetic field as: 
\begin{bux}
    \begin{split}
              \textbf{B}(\textbf{x}) = \nabla \times \frac{\mu_0}{4 \pi} \int \frac{ \textbf{J}(\textbf{x}')}{|\textbf{x}-\textbf{x}'|}d^3x'
    \end{split}
\end{bux}
This allows us to write $ \textbf{B}(\textbf{x}) = \nabla \times \textbf{A}$,  i.e. the magnetic field is the curl of an underlying vector field $\textbf{A}$, where: 
\begin{bux}
\begin{split}
\label{eqn:2.7}
    \textbf{A} = \frac{\mu_0}{4 \pi} \int \frac{ \textbf{J}(\textbf{x}')}{|\textbf{x}-\textbf{x}'|}d^3x'
\end{split}
\end{bux}
\item This expression of $\textbf{B}$ also allows us to see that $\nabla \cdot \textbf{B} = 0$, which means there are no magnetic mono poles. This is one of the differential Maxwell equations known as Gauss's law for magnetism.  

\end{itemize}

\subsection{Amperes Law}
\begin{itemize}
    \item Taking the curl of $    \textbf{B}(\textbf{x})$ and using the identity, $\nabla \times (\nabla\times \textbf{A}) = \nabla (\nabla \cdot \textbf{A}) - \nabla^2 \textbf{A}$: 
\begin{bux}
    \begin{split}
        \nabla \times \textbf{B}  =  \frac{\mu_0}{4 \pi} \nabla \int \textbf{J}(\textbf{x}') \cdot \nabla \left(\frac{ 1}{|\textbf{x}-\textbf{x}'|}\right)d^3x' - \frac{\mu_0}{4 \pi} \int \textbf{J}(\textbf{x}') \cdot \nabla^2 \left(\frac{ 1}{|\textbf{x}-\textbf{x}'|}\right)d^3x'
    \end{split}
\end{bux}
Then recognising by equation \ref{eqn:1.15} the second term here is just the integral of a delta function as $\nabla^2 \left(\frac{ 1}{|\textbf{x}-\textbf{x}'|}\right) = -4\pi\delta(\textbf{x}-\textbf{x}')$ and the fist term can be written out as: 
\begin{bux}
    \begin{split}
         \frac{\mu_0}{4 \pi} \nabla &\int \textbf{J}(\textbf{x}') \cdot \nabla \left(\frac{ 1}{|\textbf{x}-\textbf{x}'|}\right)d^3x' =   \frac{\mu_0}{4 \pi} \nabla \int \left[ \nabla\left(\frac{ \textbf{J}(\textbf{x}')}{|\textbf{x}-\textbf{x}'|}\right) - \frac{ 1}{|\textbf{x}-\textbf{x}'|} \nabla \cdot \textbf{J}(\textbf{x}')\right] \\
    \end{split}
\end{bux}
The first term on the RHS is $\frac{ \textbf{J}(\textbf{x}')}{|\textbf{x}-\textbf{x}'|}$ evaluated at infinity, so it vanishes as we assume there to be no sources at infinity and the second terms contains $\nabla\cdot \textbf{J} = 0$, which we know is $0$ in magneto-statics.  Thus this whole term is $0$ and we can write the curl of the magnetic field as: 
\begin{bux}
    \begin{split}
\label{eqn:2.10}
        \nabla \times \textbf{B}=  \mu_0\textbf{J}
    \end{split}
\end{bux}
This is another differential Maxwell equations known as Ampere's law, though it is time independent we will discuss the time dependence later.  We can turn this into the integral form by taking the integral over an open surface through which charge is flowing: 
\begin{bux}
    \begin{split}
\label{eqn:2.11}
        \int_S  \nabla \times \textbf{B} \cdot d\textbf{A} = \oint _{\partial S}\textbf{B}\cdot d\textbf{l} = \mu_0 \int_S \textbf{J}\cdot d\textbf{A} = \mu_0I
    \end{split}
\end{bux}
Where here the first step uses stokes theorem.   
\end{itemize}

\subsection{Biot Savart Law}
\begin{itemize}
    \item If we consider the scenario of an infinitely long wire carrying current in the $\hat{\textbf{z}}$ direction so $\textbf{J} = J\hat{\textbf{z}}$, then \ref{eqn:2.5} becomes: 
\begin{bux}
    \begin{split}
        &  \textbf{B}(\textbf{x}) = \frac{\mu_0}{4 \pi} \int_S da' J \int \hat{\textbf{z}} \times \frac{\textbf{x}-\textbf{z}}{(z^2+x^2)^{\frac{3}{2}}}dz \\ 
  =  \textbf{B}(\textbf{x}) &= \frac{\mu_0}{4 \pi}I \int \frac{x}{(z^2+x^2)^{\frac{3}{2}}}dz (\hat{\textbf{z}} \times \hat{\textbf{x}}) = \frac{\mu_0I}{4 \pi} (\frac{2}{x}) (\hat{\textbf{z}} \times \hat{\textbf{x}})   = \frac{\mu_0I}{2 \pi x}\hat{\phi} 
    \end{split}
\end{bux}
One could also easily obtain this result from \ref{eqn:2.11}. 
\end{itemize}

\subsection{Gauge transformation of the Vector potential }
\begin{itemize}
    \item One can note that since $\textbf{B} = \nabla \times \textbf{A}$,  then any shift of $\textbf{A}$ by $\textbf{A}' = \textbf{A} + \nabla \chi $, is physically equivalent to the original vector potential as we still have that $\textbf{B} = \nabla \times \textbf{A}'$ due to the identity $\nabla \times (\nabla \cdot \chi )= 0$ for any scalar field $\chi$. All vector potentials which differ by a Gauge transformations are physically equivalent. 
\end{itemize}

\subsection{Coulomb Gauge}
\begin{itemize}
    \item We can use this gauge to impose certain conditions on $\textbf{A}$. Suppose $\nabla \cdot \textbf{A} = \psi$ and $\textbf{A}' = \textbf{A}+ \nabla \chi \implies \nabla \cdot \textbf{A} = \psi + \nabla^2 \chi$. Thus  if we can find $\chi$ st that $ \nabla^2 \chi = - \psi$ we then have that 
$\nabla \cdot \textbf{A}=0$. This is something we can always do as we know the greens function for the Poisson equation and the solution is always unique. This is the Coulomb gauge. 

\item We can then see that with this gauge taking the cross of $\textbf{B}$ results in $\nabla \times (\nabla \times \textbf{A}) = \nabla(\nabla \cdot \textbf{A}) - \nabla^2\textbf{A}$.  So using the Coulomb gauge this is just $- \nabla^2\textbf{A}$, then using \ref{eqn:2.10} we see that: 
\begin{bux}
    \begin{split}
        \nabla^2\textbf{A} = - \mu_0 \textbf{J}
    \end{split}
\end{bux}
\item We can then deduce the form of $\chi$ that results in this gauge in magneto statics by analysing the divergence of  \ref{eqn:2.7}:
\begin{bux}
    \begin{split}
        \frac{4 \pi}{\mu_0} \nabla \cdot \textbf{A} = \int \textbf{J}(\textbf{x}') \cdot \nabla  \left(\frac{ d^3x'}{|\textbf{x}-\textbf{x}'|}\right)    = -\int \textbf{J}(\textbf{x}') \cdot \nabla'  \left(\frac{ d^3x'}{|\textbf{x}-\textbf{x}'|}\right) 
        \end{split}
\end{bux}
Here we have just used the fact that $\nabla  \left(\frac{ 1}{|\textbf{x}-\textbf{x}'|}\right)=-\nabla '\left(\frac{ 1}{|\textbf{x}-\textbf{x}'|}\right) $. Then we can write this as: 
\begin{bux}
    \begin{split}
        = - \int  \nabla' \cdot \left(\frac{ \textbf{J}(\textbf{x}')}{|\textbf{x}-\textbf{x}'|}\right)d^3x' + \int \frac{\nabla \cdot \textbf{J}}{|\textbf{x}-\textbf{x}'|}d^3x' =0 
    \end{split}
\end{bux}
Where the this is the same as before, the first term on the RHS is $\frac{ \textbf{J}(\textbf{x}')}{|\textbf{x}-\textbf{x}'|}$ evaluated at infinity, so it vanishes as we assume there to be no sources at infinity and the second terms contains $\nabla\cdot \textbf{J} = 0$, which we know is $0$ in magneto-statics.  So we have that $\nabla \cdot \textbf{A} = 0$, reduces to $\nabla^2\chi =0 $ so $\chi$ is at most constant. Having $\chi = k\hat{\textbf{x}}$ or $\chi = k\hat{\textbf{y}}$ or $\chi = k\hat{\textbf{z}}$, would also satisfy this equation but these would then have that $\textbf{A}$ does not vanish at infinity, so we restrict $\chi$ to being constant. 
\end{itemize}
\newpage 
\section{Magneto-dynamics}
\subsection{Faraday's law}
\begin{itemize}
    \item Michael Faraday's observations were that a changing magnetic flux induced an electric field around the circuit.This can be expressed mathematically with the following procedure. Let a circuit $C$ be bounding by an open surface $S$, with a magnetic field $\textbf{B}$ in the region. The magnetic flux is then defined as:
\begin{bux}
    \begin{split}
        \varphi_B = \int_S \textbf{B} \cdot d\textbf{A}
    \end{split}
\end{bux}
There is also the electromotive force around the circuit which is defined as: 
\begin{bux}
    \begin{split}
       \mathscr{E} =  \oint_C \mathbf{E}\cdot d \textbf{l}
    \end{split}
\end{bux}
Faraday's observations can be summed up as: 
\begin{bux}
    \begin{split}
        \mathscr{E} \propto - \frac{d\varphi_B}{dt}
    \end{split}
\end{bux}
So we can write this as:
\begin{bux}
    \begin{split}
\label{eqn:2.19}
        \oint_C \mathbf{E}\cdot d \textbf{l} = - k \frac{d}{dt}\int_S \textbf{B} \cdot d\textbf{A}
    \end{split}
\end{bux}
The flux can be changed by either changing the shape, orientation or place of the circuit. If the circuit is moving with a velocity $\textbf{v}$ having a different electric field $\textbf{E}'$. Then the right hand side of the above equation \ref{eqn:2.19} can be split apart as follows using the connective/material derivative (fancy name for doing out the chain rule component wise):
\begin{bux}
    \begin{split}
        \frac{d \textbf{B}}{dt} = \frac{\partial \textbf{B}}{\partial t} + (\textbf{v}\cdot \nabla)\textbf{B} = \frac{\partial \textbf{B}}{\partial t} + \nabla \times (\textbf{B}\times \textbf{v}) - \textbf{v}(\nabla \cdot \textbf{B})
    \end{split}
\end{bux}
From the "no mono poles" Maxwell equation the last term vanishes and when we integrate over the open surface by stokes theorem the second term turns into a line integral changing \ref{eqn:2.19} to: 
\begin{bux}
    \begin{split}
\label{eqn:2.20}
        \oint_C [\mathbf{E}'+k(\textbf{B}\times \textbf{v})]\cdot d \textbf{l} = -  \int_S \frac{\partial \textbf{B}}{\partial t} \cdot d\textbf{A}
    \end{split}
\end{bux}
Now if we consider the circuit instantaneously at a certain position in space in the laboratory. Then the time derivative in the original equation \ref{eqn:2.19} becomes a partial derivative by the assumption of Galilean invariance "the laws of motion are the same in all inertial frames of reference"  meaning the left hand side of both \ref{eqn:2.19} and \ref{eqn:2.20} must be the same so: 
\begin{bux}
    \begin{split}
        \mathbf{E}' = \textbf{E} + k(\textbf{v}\times \textbf{B}) 
    \end{split}
\end{bux}
Using this \ref{eqn:2.20} can then be written via stokes theorem once more as:
\begin{bux}
    \begin{split}
          \int_S \left(\nabla \times \textbf{E}+\frac{\partial \textbf{B}}{\partial t} \right)\cdot d\textbf{A} =0
    \end{split}
\end{bux}
Where we have dropped the factor of $k$ as it turns out to be identity in SI units. This equation then gives us the final time dependant Maxwell equation, known as the Maxwell–Faraday equation. 
\begin{bux}
    \begin{split}
        \nabla \times \textbf{E}+\frac{\partial \textbf{B}}{\partial t} =0
    \end{split}
\end{bux}




\end{itemize}

\subsection{Time dependant Ampere's Law}
\begin{itemize}
    \item The last Maxwell equation can be obtained by adding time dependence to the previously discussed Ampere's law, i.e. $\nabla \times \textbf{B} = \mu_0 \textbf{J} $. Note that using this expression we can write the conservation of charge equation \ref{eqn:2.3} as:
\begin{bux}
    \begin{split}
        \nabla \cdot ( \nabla \times \textbf{B} ) = -\mu_0 \frac{\partial \rho}{\partial t}
    \end{split}
\end{bux}
The left side is identically zero, but for the right side we know this is not always the case. To fix this we can use \ref{eqn:1.15} to write: 
\begin{bux}
    \begin{split}
       & \nabla \frac{\partial \textbf{E}}{\partial t} = \frac{1}{\epsilon_0}\frac{\partial \rho}{\partial t} \\
\implies &\nabla\left( \textbf{J} + \epsilon_0 \frac{\partial \textbf{E}}{\partial t}\right) = 0 \\
\implies \nabla \times \textbf{B}& = \mu_0\left( \textbf{J} + \epsilon_0 \frac{\partial \textbf{E}}{\partial t}\right)
    \end{split}
\end{bux}
The last Maxwell relation. 

\end{itemize}
\newpage
\section{Maxwell's equations }
\subsection{Maxwell's equations}
\begin{itemize}
    \item We begin this section by laying out the Maxwell equations which we have motivated in the above chapters:
\begin{bux}
    \begin{split}
\label{max}
       & ~~\nabla \cdot \textbf{B}  = 0 ~~~ (1)  ~~~~~~~~~~~~~~\nabla \times \textbf{E}+\frac{\partial \textbf{B}}{\partial t} =0~~~~~(2) \\
&  \nabla \cdot \textbf{E}= \frac{\rho}{\epsilon_0}  ~~~(3) ~~~~~\nabla \times \textbf{B} = \mu_0\left( \textbf{J} + \epsilon_0 \frac{\partial \textbf{E}}{\partial t}\right) ~~~(4)
    \end{split}
\end{bux}
The top two equations are refereed to as the "homogeneous Maxwell equations" where as the bottom two are the "inhomogeneous Maxwell equations".  
\end{itemize}

\subsection{Electromagnetic Wave}
\begin{itemize}
    \item Let us consider the electric and magnetic fields in the vacuum ($\rho = \textbf{J}=0$). Looking at the second Maxwell relation \ref{max}, if we take the curl of both sides we see: 
\begin{bux}
    \begin{split}
        \nabla \times (\nabla \times \textbf{E}) = \nabla (\nabla \cdot \textbf{E}) - \nabla^2 \textbf{E} + \frac{\partial}{\partial t}\nabla \times \textbf{B} = 0
    \end{split}
\end{bux}
The first Maxwell relation \ref{max} now tells us this first term is zero and the fourth equation reduces this too: 
\begin{bux}
    \begin{split}
         -\nabla^2 \textbf{E} + \frac{1}{c^2}\frac{	\partial^2}{\partial t^2}\textbf{E} = 0
    \end{split}
\end{bux}
Here $\frac{1}{c^2} = \mu_0 \epsilon_0$ (the speed of light). In a similar manner we can find the following expression for $\textbf{B}$:
\begin{bux}
    \begin{split}
        - \nabla^2\textbf{B} + \frac{1}{c^2}\frac{\partial^2 \textbf{B}}{\partial t^2} =0 
    \end{split}
\end{bux}
Both of these can be recognised as waves equations, meaning in the vacuum these fields travel as oscillating waves.  This means:
\begin{bux}
    \begin{split}
        & \textbf{E}  =  \text{Re}(\textbf{E}_0e^{i\textbf{k}\cdot\textbf{x}-i\omega t}) \\
& \textbf{B}  =  \text{Re}(\textbf{B}_0e^{i\textbf{k}\cdot\textbf{x}-i\omega t})
    \end{split}
\end{bux}
And the same for $\textbf{B}$. Then seeing that in the vacuum both $\nabla \cdot \textbf{E} =\nabla \cdot \textbf{B}=0$ $\implies \textbf{k}\cdot \textbf{E}_0 = \textbf{k}\cdot \textbf{B}_0 =0$. Meaning both these waves are transverse wave. 
\end{itemize}

 \subsection{Vector and Scalar potentials }
 \begin{itemize}
     \item As discussed before the magnetic field has an underlying vector potential such that $\textbf{B} = \nabla \times \textbf{A}$ , (from the first Maxwell relation \ref{max}) and the electric field has an underlying scalar field such that $\textbf{E} = - \nabla \Phi$. But this last equation was under the assumptions of electrostatics. To find the right solution we can write the second Maxwell equation \ref{max} as:
\begin{bux}
    \begin{split}
        \nabla\times \textbf{E} + \frac{\partial}{\partial t} \nabla \times \textbf{A} \iff \nabla \times \left( \textbf{E} + \frac{\partial \textbf{A}}{\partial t}\right) = 0
    \end{split}
\end{bux}
This last expression means the terms in the brackets can be written as (minus) the gradient of a scalar field $\Phi$, so: 
\begin{bux}
    \begin{split}
\label{eqn:4.7}
        \textbf{E} = - \nabla \Phi - \frac{\partial \textbf{A}}{\partial t}
    \end{split}
\end{bux}
\item Now the third Maxwell equation \ref{max} becomes: 
\begin{bux}
    \begin{split}
         \nabla^2 \Phi + \frac{\partial}{\partial t} \nabla \cdot \textbf{A} =  - \frac{\rho}{\epsilon_0}
    \end{split}
\end{bux}
And the fourth Maxwell equation becomes: 
\begin{bux}
    \begin{split}
\label{eqn:4.9}
    \nabla^2 \textbf{A} - \frac{1}{c^2} \frac{\partial^2 \textbf{A}}{\partial t^2} - \nabla \left( \nabla \cdot \textbf{A} + \frac{1}{c^2}\frac{\partial \Phi}{\partial t}\right) = - \mu_0\textbf{J}  
    \end{split}
\end{bux}
Where we have written $\mu_0 \epsilon_0$ as $\frac{1}{c^2}$. We have now reduced the Maxwell equations to, two coupled equations. 

\end{itemize}
\subsubsection{Lorenz gauge}
\begin{itemize}
\item To decouple these equations we exploit the fact that these potentials can be Gauge transformed. As discussed before we can transform $\textbf{A}$ by:  
\begin{bux}
    \begin{split}
\label{eqn:4.10}
        \textbf{A}' = \textbf{A} + \nabla \chi 
    \end{split}
\end{bux}
As this gives rise to the same magnetic field $\textbf{B}$. Now we must also then transform the scalar potential $\Phi$ so that it gives rise to the same electric field $\textbf{E}$. For this to happen according to \ref{eqn:4.7}:
\begin{bux}
    \begin{split}
\label{eqn:4.11}
        \Phi' = \Phi - \frac{\partial \chi}{\partial t}
    \end{split}
\end{bux}
The invariance of the fields under these transformations is called \emph{gauge invariance}. Equations \ref{eqn:4.10} and \ref{eqn:4.11} allow us to choose $\chi$ such that: 
\begin{bux}
    \begin{split}
\label{eqn:4.12}
        \nabla \cdot \textbf{A} + \frac{1}{c^2} \frac{\partial  \Phi}{\partial t } = 0 
    \end{split}
\end{bux}
This is the \emph{Lorenz Gauge} and makes it so that \ref{eqn:4.8} can be written as: 
\begin{bux}
    \begin{split}
\label{eqn:4.13}
         \nabla^2 \Phi - \frac{1}{c^2}\frac{\partial^2 \Phi}{\partial t^2}  =  - \frac{\rho}{\epsilon_0}
    \end{split}
\end{bux}
And  \ref{eqn:4.9} becomes: 
\begin{bux}
    \begin{split}
\label{eqn:4.14}
          \nabla^2 \textbf{A} - \frac{1}{c^2} \frac{\partial^2 \textbf{A}}{\partial t^2}= - \mu_0\textbf{J} 
    \end{split}
\end{bux}
These two equations plus the Lorentz gauge condition form a set of equations that are equivalent in all respects to the Maxwell equations in vacuum. 

\item We can also note that the Lorenz condition does not completely exhaust the arbitrariness of the potentials. Namely choosing: 
\begin{bux}
    \begin{split}
        \nabla^2 \chi - \frac{1}{c^2} \frac{\partial^2 \chi}{\partial t ^2} = 0
    \end{split}
\end{bux}
Will still satisfy the Lorenz gauge. Notably this means $\chi$ satisfies the wave equation so in the vacuum where \ref{eqn:4.13} and \ref{eqn:4.14} are also wave equations, $\chi$ can be chosen such that $\Phi=0$ as per \ref{eqn:4.11}. This mean the fields only depend on $\textbf{A}$, which is a plane wave as now from \ref{eqn:4.12} $\nabla \cdot \textbf{A}=0$. 



 \end{itemize}

\newpage 
\section{Tensors}
\subsection{Vectors}
\begin{itemize}
    \item Rotations (SO$(3)$) leave the magnitude of vectors $|\textbf{x}|$ invariant, where the magnitude is defined as:
\begin{bux}
    \begin{split}
        |\textbf{x}|^2  = \sum_i x^ix^i
    \end{split}
\end{bux}
This has the following consequences. If each co-ordinate transforms by $x^i \rightarrow x'^i = \sum_ja^{ij}x^i$, then the new magnitude is: 
\begin{bux}
    \begin{split}
        |\textbf{x}'| =& \sum_ix'^ix'^i = \sum_{i,j,k}a^{ij}a^{ik}x^jx^k = \sum_jx^jx^j,~~~\forall ~\textbf{x} \\ 
&\implies  \sum_ia^{ij}a^{ik} = \delta^{jk} \iff A^TA = \mathbb{I}
    \end{split}
\end{bux}
Where by $A$ here we mean the matrix $A$ that has components $a^{ij}$. This condition $A^TA = \mathbb{I}$ also implies $\text{det}(A)^2 = 1 \implies \text{det}(A)=\pm 1$. 
\end{itemize}

\end{document}
