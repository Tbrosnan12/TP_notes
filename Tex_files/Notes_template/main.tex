\documentclass[11pt]{article}

\usepackage[letterpaper,top=2cm,bottom=2cm,left=2cm,right=2cm,marginparwidth=1.75cm]{geometry}
\usepackage{hyperref}
\usepackage{biblatex}
\addbibresource{Bib.bib}
\usepackage{mathtools}
\DeclarePairedDelimiterXPP\BigOSI[2]%
  {\mathcal{O}}{(}{)}{}%
  {\SI{#1}{#2}}
\usepackage{xcolor}
\usepackage{empheq}
\usepackage[most]{tcolorbox}
\usepackage{amsmath}
\usepackage{amssymb}
\usepackage{mathrsfs}
\usepackage[utf8]{inputenc}
\usepackage{graphicx}
\usepackage{float}
\usepackage{parskip}
\usepackage{comment}
\usepackage{mhchem}
 \usepackage{tabularx}
 \usepackage{titling}
 \usepackage{amsmath,environ}
 \usepackage[explicit]{titlesec}
\usepackage{fancyhdr}
\setlength{\droptitle}{3em} 

\title{Notes Template}
\author{Thomas Brosnan}
\date{Notes taken in Professor Oppenheimer's class, Michaelmas Term 1939}


\newtcbox{\mymath}[1][]{%
    nobeforeafter, math upper, tcbox raise base,
    enhanced, colframe=blue!30!black,
    colback=blue!30, boxrule=1pt,
    #1}
\tcbset{highlight math style={boxsep=2mm,,colback=blue!0!green!0!red!0!}}

  \newenvironment{bux}
    {
    \empheq[box=\tcbhighmath]{equation}
   }{
    \endempheq
    }
    \newenvironment{bux*}
    {
    \empheq[box=\tcbhighmath]{equation*}
   }{
    \endempheq
    }
\newcommand{\hsp}{\hspace{8pt}}

\newcommand*{\sectionFont}{%
  \LARGE\bfseries
}

\newenvironment{eq}{\begin{equation}}{\end{equation}}
    
\numberwithin{equation}{section}

\makeatletter
\let\Title\@title % Copy the title to a new command
\makeatother

%change this RGB value to change the section background colour 
\definecolor{mycolor1}{RGB}{255, 128, 128}
\colorlet{SectionColour}{mycolor1}
%subsection background colour 
\definecolor{mycolor2}{gray}{0.8}
\colorlet{subSectionColour}{mycolor2}
%subsubsection background colour 
\definecolor{mycolor3}{RGB}{255,255,255}
\colorlet{subsubSectionColour}{mycolor3}


\begin{document}

\maketitle

\newpage
\topskip0pt
\vspace*{\fill}
\begin{center}
\Large
    "If you can't explain it simply enough you don't understand it well enough"
    
    - Albert Einstein
\end{center}
\vspace*{\fill}
\newpage 
\tableofcontents
% For \section
 \titleformat{\section}[block]{\sectionFont}{}{0pt}{%
 \fcolorbox{black}{SectionColour}{\noindent\begin{minipage}{\dimexpr\textwidth-2\fboxsep-2\fboxrule\relax}\thesection  \hsp #1 {\strut} \end{minipage}}}
% For \subsection
 \titleformat{\subsection}[block]{\bfseries}{}{0pt}{%
 \fcolorbox{black}{subSectionColour}{\noindent\begin{minipage}{\dimexpr\textwidth-2\fboxsep-2\fboxrule\relax}\thesubsection  \hsp #1 {\strut} \end{minipage}}}
% For \section*
 \titleformat{name=\section, numberless}[block]{\sectionFont}{}{0pt}{%
 \fcolorbox{black}{SectionColour}{\noindent\begin{minipage}{\dimexpr\textwidth-2\fboxsep-2\fboxrule\relax} #1 {\strut} \end{minipage}}}
  % For \subsection*
 \titleformat{name=\subsection, numberless}[block]{\bfseries}{}{0pt}{%
 \fcolorbox{black}{subSectionColour}{\noindent\begin{minipage}{\dimexpr\textwidth-2\fboxsep-2\fboxrule\relax} #1 {\strut} \end{minipage}}}
 % For \subsubsection
 \titleformat{\subsubsection}[block]{\bfseries}{}{0pt}{%
 \fcolorbox{black}{subsubSectionColour}{\noindent\begin{minipage}{15cm}\thesubsubsection \hsp #1 {\strut} \end{minipage}}}
  % For \subsubsection*
 \titleformat{name=\subsubsection, numberless}[block]{\bfseries}{}{0pt}{%
 \fcolorbox{black}{subsubSectionColour}{\noindent\begin{minipage}{15cm} #1 {\strut} \end{minipage}}}
\newpage 
%header 
\pagestyle{fancy}
\fancyhf{} % Clear all header and footer fields
\fancyhead[L]{\Title}
\fancyhead[R]{\nouppercase{\leftmark}}
\fancyfoot[C]{-~\thepage~-}
\renewcommand{\headrulewidth}{1pt}





%starting document 
\normalsize
\newpage
\section{Section}

\subsection{Theorem:} let $A$ be an element of $R$ such that:
\begin{align}
\begin{split}
    c_i = \langle\psi|\phi\rangle ,~~~~ c_i = \langle\psi|\phi\rangle  \\
    c_i = \langle\psi|\phi\rangle ,~~~~ c_i = \langle\psi|\phi\rangle
\end{split}
\end{align}

%the bux 

Then the final result is:

\begin{bux}
\begin{split}
    c_i = \langle\psi|\phi\rangle ,~~~~ c_i = \langle\psi|\phi\rangle  \\
    c_i = \langle\psi|\phi\rangle ,~~~~ c_i = \langle\psi|\phi\rangle
\end{split}
\end{bux}

\begin{figure}[H]
\centering
\includegraphics[width=0.8\textwidth]{Screenshot 2023-06-15 172321.png}
\caption{\label{fig:2}\emph{Diagram of the experimental setup circuit}}
\end{figure}

\newpage
let $A$ be an element of  such that:

\begin{align}
\frac{1}{2} =     
\end{align}

\end{document}
