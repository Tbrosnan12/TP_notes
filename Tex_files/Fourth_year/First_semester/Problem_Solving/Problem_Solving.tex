\documentclass[11pt]{article}

%Format and referencing 
\usepackage[letterpaper,top=2cm,bottom=2cm,left=2cm,right=2cm,marginparwidth=1.75cm]{geometry} % for setting margins VERY IMPORTANT!
\usepackage{hyperref}  % for referencing equations 
\usepackage{biblatex}  % for referencing articles 
\addbibresource{Bib.bib}


%math packages 
\usepackage{mathtools}  % allows pre-scripts and other neiche math features
\usepackage{amsmath} % equation related commands 
\usepackage{amssymb} % Fancy R ,C and other set fonts using mathbb{}
\usepackage{braket}   % for Dirac notation 
\usepackage{mathrsfs}  % for the mathscr text 

% miscellaneous
\usepackage{empheq}   % for formatting custon math boxs
\usepackage{xcolor}   % allows the defining of colours 
\usepackage[most]{tcolorbox}  % custom math boxs
\usepackage[utf8]{inputenc}   % redundant since 2018? 
%(I'm not taking out in case it breaks stuff)

\usepackage{graphicx}   % alows image insertion 
\usepackage{float}  % for making images o where you want 
\usepackage{parskip} % for getting red of paragraph indent 

\usepackage{comment} %easier multi line comments 
 \usepackage{tabularx} % for formatting tables 
 \usepackage{titling} % for making Title a variable that can be used in header 
 \usepackage{environ} % for creating and modifying environments 
 \usepackage[explicit]{titlesec}
 % for making section titles variables that can be used in header 
\usepackage{fancyhdr} % for fancy headers 
\usepackage{bm}
% Random commands and what not 
\numberwithin{equation}{section}

\setlength{\droptitle}{3em} 

\title{Problem Solving}
\author{Thomas Brosnan}
\date{Sheet of equations useful for the Problem solving paper}

\DeclarePairedDelimiterXPP\BigOSI[2]%
  {\mathcal{O}}{(}{)}{}%
  {\SI{#1}{#2}}

\newtcbox{\mymath}[1][]{%
    nobeforeafter, math upper, tcbox raise base,
    enhanced, colframe=blue!30!black,
    colback=blue!30, boxrule=1pt,
    #1}
\tcbset{highlight math style={boxsep=2mm,,colback=blue!0!green!0!red!0!}}

\newenvironment{bux}{\empheq[box=\tcbhighmath]{align}}{\endempheq}
\newenvironment{bux*}{\empheq[box=\tcbhighmath]{align*}}{\endempheq}
\renewenvironment{flalign}{\vspace{-2mm}\empheq[box=\tcbhighmath]{align}}{\endempheq}
\renewenvironment{flalign*}{\vspace{-2mm}\empheq[box=\tcbhighmath]{align*}}{\endempheq}
%\renewenvironment{align}{\vspace{-5mm}\begin{align}}{\end{align}}
%\renewenvironment{align*}{\vspace{-5mm}\begin{align*}}{\end{align*}}
\renewenvironment{alignat}{\empheq{align*}}{\endempheq}


\newcommand{\hsp}{\hspace{8pt}}

\newcommand*{\sectionFont}{%
  \LARGE\bfseries
}

\newenvironment{eq}{\begin{equation}}{\end{equation}}
    


\makeatletter
\let\Title\@title % Copy the title to a new command
\makeatother

%change this RGB value to change the section background colour 
\definecolor{mycolor1}{RGB}{168, 182, 225}
\colorlet{SectionColour}{mycolor1}
%subsection background colour 
\definecolor{mycolor2}{gray}{0.8}
\colorlet{subSectionColour}{mycolor2}
%subsubsection background colour 
\definecolor{mycolor3}{RGB}{255,255,255}
\colorlet{subsubSectionColour}{mycolor3}



\begin{document}

\maketitle

\newpage
\topskip0pt
\vspace*{\fill}
\begin{center}
\Large
    "Mo money mo Problems"
    
    - Samson
\end{center}
\vspace*{\fill}
\newpage 
\tableofcontents
% For \section
 \titleformat{\section}[block]{\sectionFont}{}{0pt}{%
 \fcolorbox{black}{SectionColour}{\noindent\begin{minipage}{\dimexpr\textwidth-2\fboxsep-2\fboxrule\relax}\thesection  \hsp #1 {\strut} \end{minipage}}}
% For \subsection
 \titleformat{\subsection}[block]{\bfseries}{}{0pt}{%
 \fcolorbox{black}{subSectionColour}{\noindent\begin{minipage}{\dimexpr\textwidth-2\fboxsep-2\fboxrule\relax}\thesubsection  \hsp #1 {\strut} \end{minipage}}}
% For \section*
 \titleformat{name=\section, numberless}[block]{\sectionFont}{}{0pt}{%
 \fcolorbox{black}{SectionColour}{\noindent\begin{minipage}{\dimexpr\textwidth-2\fboxsep-2\fboxrule\relax} #1 {\strut} \end{minipage}}}
  % For \subsection*
 \titleformat{name=\subsection, numberless}[block]{\bfseries}{}{0pt}{%
 \fcolorbox{black}{subSectionColour}{\noindent\begin{minipage}{\dimexpr\textwidth-2\fboxsep-2\fboxrule\relax} #1 {\strut} \end{minipage}}}
 % For \subsubsection
 \titleformat{\subsubsection}[block]{\bfseries}{}{0pt}{%
 \fcolorbox{black}{subsubSectionColour}{\noindent\begin{minipage}{15cm}\thesubsubsection \hsp #1 {\strut} \end{minipage}}}
  % For \subsubsection*
 \titleformat{name=\subsubsection, numberless}[block]{\bfseries}{}{0pt}{%
 \fcolorbox{black}{subsubSectionColour}{\noindent\begin{minipage}{15cm} #1 {\strut} \end{minipage}}}
\newpage 
%header and footer
\pagestyle{fancy}
\fancyhf{} % Clear all header and footer fields
\fancyhead[L]{\Title}
\fancyhead[R]{\nouppercase{\leftmark}}
\fancyfoot[C]{-~\thepage~-}
\renewcommand{\headrulewidth}{1pt}





%starting document 
\normalsize
\newpage
\section{Mechanics}
\subsection{Constant acceleration}
\begin{itemize}
    \item For constant acceleration $a$ the equations of motion can be written in three useful forms. The variables here are initial velocity $u$ final velocity $v$, time $t$ and displacement $s$:
    \begin{flalign*}
        & v = u + at \\
        & v^2 =u^2 +2as \\
        & s = ut + \frac{1}{2}at^2
    \end{flalign*}
\end{itemize}
\subsection{Force}
\begin{itemize}
    \item The force due to a potential $U$ is:
    \begin{flalign*}
        \textbf{F} = - \nabla U 
    \end{flalign*}
\end{itemize}

\subsection{Friction}
\begin{itemize}
    \item Friction is given by the following expression, where $\mu$ is the \emph{co-efficient of friction} and $N$ is the normal force acted on the object by the surface it is sliding across. 
    \begin{flalign*}
         F_{\text{Frict}} = \mu N
     \end{flalign*} 
\end{itemize}

\subsection{Work}
\begin{itemize}
    \item For constant Force work is just defined $W =\textbf{F}\cdot\textbf{s}$. (Force times displacement). If the force is not constant:
    \begin{flalign*}
        W = \int_{a}^{b}\textbf{F} \cdot d\textbf{s}
    \end{flalign*} 
\end{itemize}
\subsubsection{Power}
\begin{itemize}
        \item \emph{Power} is defined as:
    \begin{flalign*}
        P = \frac{dW}{dr} = \textbf{F}\cdot \textbf{v}
    \end{flalign*}
    Where $\textbf{v}$ is velocity.
\end{itemize}

\subsection{Circular motion \& Rotation}
\begin{itemize}
    \item In circular motion the force acting as the \emph{centripetal} force (force pulling to wards the center, gravity, tension in rope, ect) is balanced by a \emph{centrifugal} force equal in magnitude but opposite direction, given by:
    \begin{flalign*}
        F_{\text{fugal}} = \frac{mv^2}{r} = mr\omega^2, ~~~(\text{as}~v=\omega r)
    \end{flalign*}
    Where here $v$ is the tangential velocity, $r$ is the radius, $m$ is the mass of the object in motion and $\omega$ is the angular velocity. 
\end{itemize}
\subsubsection{Rotational Kinetic energy}
\begin{itemize}
    \item Usually kinetic energy is just $\frac{1}{2}mv^2$, but for rigid bodies rotation it is:
    \begin{flalign*}
        K = \frac{1}{2}I\omega^2
    \end{flalign*}
    Where $I$ is the moment of inertia. A list of these can be found in the formula and tables booklet for different geometries. 
\end{itemize}
\subsubsection{Parallel axis Theorem}
\begin{itemize}
    \item This is the rule that tells us how to add the self rotation moment of inertia $I_{\text{self}}$ to the orbital rotation of a mass $M$ at radius $R$:
    \begin{flalign*}
        I_{\text{total}} = I_{\text{self}} + MR^2
    \end{flalign*}
\end{itemize}
\subsubsection{Torque}
\begin{itemize}
    \item This is defined as:
    \begin{flalign*}
        \boldsymbol{\tau} = \textbf{r} \times \textbf{F}
    \end{flalign*}
    We also have that:
    \begin{flalign*}
        |\boldsymbol{\tau}| = \alpha I
    \end{flalign*}
    Where $\alpha = \frac{d \omega}{dt}$.  We can then find that the work done by a source the torque is:
    \begin{flalign*}
         W = \int_{\theta_{0}}^{\theta_1}\boldsymbol{\tau}d\theta, ~~~\implies P = \tau \omega
     \end{flalign*} 
     Where $P$ is the power. 
\end{itemize}
\subsubsection{Angular momentum}
\begin{itemize}
    \item The standard definition is just:
    \begin{flalign*}
        & \textbf{L} = \textbf{r}\times \textbf{p}, ~~~\text{for a particle} \\
        &I \boldsymbol{\omega}, ~~~\text{Ridged body motion} 
    \end{flalign*}
    We can also write the torque as $\boldsymbol{\tau} = \frac{d \textbf{L}}{dt}$. 
\end{itemize}
\subsection{Center of Mass}
\begin{itemize}
    \item This is given by:
    \begin{flalign*}
        \textbf{R}_{CM} = \frac{\sum_{i}m_i\textbf{r}_i}{\sum_im_i} 
    \end{flalign*}
\end{itemize}
\subsection{Stress/strain}
\begin{itemize}
    \item Shear modulus $G$ is:
    \begin{flalign*}
        G = \frac{\text{shear stress}}{\text{shear strain}}
    \end{flalign*}
    Where shear stress $ = F/A$ (Force per Area) and shear strain  = $\frac{\Delta s}{l}$. Length displaced $\Delta s$ over the length of the object $l$. The same sort of equation holds for the other moduli. Youngs modulus is $G$ with the word ``shear'' replaced with ``Tensile''. Same goes with Bulk and Elastic. 
\end{itemize}


\subsection{Gravitation}
\begin{itemize}
    \item The Gravitational potential energy of a mass $m$ in a gravitational field produced by a body of mass $M$ is:
    \begin{flalign*}
        U = -\frac{GMm}{r} ~ ~~\implies \textbf{F}_{\text{grav}} = -\frac{GMm}{r^2}\hat{r}
    \end{flalign*}

\end{itemize}
\subsubsection{Acceleration due to gravity} 
\begin{itemize}
    \item This is just the gravitational force $F_{\text{grav}}$ per unit mass:
    \begin{flalign*}
         g = \frac{GM}{r^2}
     \end{flalign*} 
\end{itemize}
\subsection{Orbits}
\begin{itemize}
    \item To find the escape velocity we just find the velocity that makes the total energy $0$ when $r=R$, this results in:
    \begin{flalign*}
         v_{\text{esc}} = \sqrt{\frac{2GM}{R}}
     \end{flalign*} 
     The Schwartzchild radius is when this velocity $v$ is the speed of light $c$. 
\end{itemize}
\subsubsection{Kepler's Period}
\begin{itemize}
    \item This is the orbital period for a body in elliptical motion with semi-major axis $a$:
    \begin{flalign*}
        T^2 =\frac{4\pi^2a^3}{GM} 
    \end{flalign*}
\end{itemize}
\subsection{Periodic motion}
\begin{itemize}
    \item \emph{Simple Harmonic motion} is defined as motion where the force is:
    \begin{flalign*}
        F = - kx
    \end{flalign*}

\item This motion will then have a angular frequency $\omega$, which is related to the period of the motion $T$, by: $T = 2\pi/\omega$.
    \item Frequency is defined as one over the period $f = 1/T = \omega/2\pi$. 
\end{itemize}
\subsubsection{Pendulum}
\begin{itemize}
 \item For a pendulum of length $L$ we have that $\omega = \sqrt{\frac{g}{L}}$. If this pendulum is a rigid body with moment of inertia $I$, then this takes the form: $\omega = \sqrt{\frac{mgL}{I}}$. 
\end{itemize}

\subsubsection{Damped Harmonic Motion}
\begin{itemize}
    \item The General solution to damped harmonic motion takes the form:
    \begin{flalign*}
        x(t) = Ae^{-\gamma t}cos(\omega't)
    \end{flalign*}
    Where $\omega' = \sqrt{\frac{k}{m}-\frac{k^2}{4m^2}}$.  
\end{itemize}

\subsection{Density and Pressure}
\begin{itemize}
    \item Density is defined as:
    \begin{flalign*}
        \rho = \frac{m}{V}
    \end{flalign*}
    \item Pressure is in its most general form perpendicular Force $F_{\perp}$ divided by area $A$:
    \begin{flalign*}
        P  = \frac{F_{\perp}}{A}
    \end{flalign*}
\end{itemize}

\subsubsection{Pressure in a fluid}
\begin{itemize}
    \item At a height $h$ below the surface is:
    \begin{flalign*}
        P = P_0 + \rho gh
    \end{flalign*}
\end{itemize}
\subsubsection{Fluid Flow}
\begin{itemize}
    \item For a fluid flowing through a pipe of changing cross sectional area the conserved quantity is the area $A$ times the velocity of the fluid $v$. ($A_1v_1 = A_2v_2$). 
\end{itemize}
\subsubsection{Bernoulli's equation}
\begin{itemize}
    \item The relations ship between pressure and velocity in a fluid is given by:
    \begin{flalign*}
        P + \frac{1}{2}\rho v^2 = \text{constant}
    \end{flalign*}
\end{itemize}
\subsubsection{Wave equation}
\begin{itemize}
    \item This takes the form:
    \begin{flalign*}
        \frac{\partial^2y}{\partial x^2} = \frac{1}{v^2}\frac{\partial^2 y}{\partial t^2}
    \end{flalign*}
    Where $y- y(x,t)$ and $v = \lambda * f$ (wavelength by frequency) is the speed of the wave. The general form to the solutions of this equation are:
    \begin{flalign*}
        y(x,t) = A\cos(kx-\omega t)
    \end{flalign*}
\end{itemize}

\subsubsection{Wave Power}
\begin{itemize}
    \item Given the tension $T$ in a string with mass per length $\mu = m/L$ the wave power is:
    \begin{flalign*}
        P = \frac{1}{2}\sqrt{\mu T}\omega^2A^2
    \end{flalign*}
\end{itemize}
\subsubsection{Beat frequency}
\begin{itemize}
    \item When there is two sources of oscillatory motion the beat frequency is the difference between the two underlying  frequencies $f_{\text{beat}} = |f_1-f_2|$. 
\end{itemize}

\subsubsection{Doppler affect}
\begin{itemize}
    \item The observed frequency of a source emitting a frequency $f_0$ when the receiver is moving at velocity $v_r$ and the source at velocity $v_s$ is:
    \begin{flalign*}
         f' = \left(\frac{c\pm v_r}{c \mp v_s}\right)f_0
     \end{flalign*}
     Where we have $+$ in the numerator if receiver is moving towards source and a $-$ for away. And we have a $+$ in the denominator if the source is moving towards the medium and a $-$ if away. 
\end{itemize}


\newpage 

\section{Thermodynamics}
\subsection{Heat}
\subsubsection{Heat Capacity}
\begin{itemize}
\item Heat capacity is defined as the temperature derivative of the Energy:
    \begin{flalign*}
        C_V = \left(\frac{\partial E}{\partial T}\right)_V
    \end{flalign*}
    For $C_{P}$ just replace $V$ with $P$. 
    \item Heat energy change due to a temperature change $\Delta T$ for a substance with heat specific capacity $c \equiv C/m$ and mass $m$:
    \begin{flalign*}
        \Delta Q  = mc \Delta T
    \end{flalign*}
    \item The two heat capacities are related by:
    \begin{flalign*}
         C_p = C_v +nR, ~~~\gamma = \frac{C_p}{C_v} 
     \end{flalign*} 
     \item At constant volume we have from the first law since $dV = 0$:
     \begin{flalign*}
         C_v = \frac{dQ}{dT} = \left(\frac{\partial U}{\partial T}\right)_V = T\left(\frac{\partial S}{\partial T}\right)
     \end{flalign*}
      \end{itemize}


    \subsubsection{Latent Heat}
    \begin{itemize}
        \item For a mass $m$ and specific latent heat $l$:
        \begin{flalign*}
            \Delta E = ml
        \end{flalign*}
    \end{itemize}

    \subsubsection{Heat Flux}
    \begin{itemize}
        \item This is the heat flux radiated by a source of temperature $T$ with emissivity $\varepsilon$: 

        \begin{flalign*}
            F = \varepsilon\sigma T^4
        \end{flalign*}
        Where $\sigma$ is the Stephan Boltzmann constant. Flux is also related to power $P$ and area $A$ via:

        \begin{flalign*}
            F = P/A
        \end{flalign*}
    \end{itemize}

\subsection{Ideal Gas}
\begin{itemize}
        \item The ideal Gas Law is:
        \begin{flalign*}
            PV = NkT
        \end{flalign*}
                  N is the number of particles and $k$ the Boltzmann factor. 

          \end{itemize}
\subsubsection{Energy of a Gas}
\begin{itemize}
\item The relation for energy of a Gas is:
\begin{flalign*}
    E = \frac{d}{2}NkT
\end{flalign*}
Where $d$ is the degrees of freedom of the gas molecules, for ideal Gas this is just $d=3$. Note that this means the heat capacity is $C_v = \frac{d}{2}Nk$

\item This can also be expressed as:
\begin{flalign*}
    E = \frac{1}{2}m N\bar{v}^2
\end{flalign*}
Where $\bar{v}$ is the average velocity given by $\bar{v} = \sqrt{\frac{dkT}{m}}$
\end{itemize}

\subsection{Mean free Path}
\begin{itemize}
    \item This is just defined as the velocity of the particles times the mean free time $t_{\text{mean}}$. Which can be expressed as:
    \begin{flalign*}
        \lambda = \frac{1}{\sigma n}
    \end{flalign*}
    Where $\sigma$ is the cross section and $n$ the number of particles per unit volume. This is assuming stationary particles, or that the incoming particle has a much higher velocity then the targets. If this is not the case, i.e. thee incoming particle is at thermal equilibrium with the targets then the mean free path changes as the number of collisions becomes $\sqrt{2}$ times that of the stationary case. $l = 1/(\sqrt{2}n\sigma)$. 
    
    \item The intensity of particles in to a medium falls off exponentially, depending on the mean free path, $I = I_0e^{x/l}$. The probability can then be calculated from this:
    \begin{flalign*}
     dP(x) = \frac{I(x)-I(x+dx)}{dx} = \frac{1}{l}e^{-x/l}dx
     \end{flalign*} 
\end{itemize}


\subsection{First Law} % (fold)
\label{sub:first_law}
\begin{itemize}
    \item The first law of thermodynamics is:
    \begin{flalign*}
        dU = dW +dQ
    \end{flalign*}
    Where $U$ is internal energy, $W$ the work (which is not always exact) and $Q$ the heat. Often we will have the expression that the infinitesimal work done is $dW = PdV$ with the sign convention that work done on the system is positive.  
\end{itemize}
% subsection first_law (end)

\subsection{Processes} % (fold)
\label{sub:processes}
\begin{itemize}
    \item The following processes mean the following things:
    \begin{itemize}
        \item Adiabatic: $\Delta Q=0$, no heat transfer.
        \item Isochoric/mechanically isolated: $\Delta W = 0 $ (constant volume).
        \item Isobaric: (constant pressure)
        \item Isothermal: $\Delta T = 0$. (constant temperature).   
    \end{itemize}
\end{itemize}
\subsubsection{Adiabatic Process} % (fold)
\label{ssub:adiabatic_process}
\begin{itemize}
    \item In this case we have the following expression where $\gamma$ is the ratio of heat capacities:
    \begin{flalign*}
         TV^{\gamma-1} = \emph{const},~~~PV^{\gamma-1} = \emph{const}
     \end{flalign*} 
     \item $dW = dU \implies W = C_vNkT$. 

\end{itemize}
% subsubsection adiabatic_process (end)
% subsection processes (end)

\subsection{Efficacy of a heat engine} % (fold)
\label{sub:efficacy}
\begin{itemize}
    \item This is defined as the ratio of output work to input heat:
    \begin{flalign*}
        \varepsilon = \frac{W}{Q_2} = \frac{Q_2-Q_1}{Q_1} = 1-\frac{Q_1}{Q_2}
    \end{flalign*}
\end{itemize}
\subsubsection{Carnot cycle} % (fold)
\label{ssub:carnot_cycle}
\begin{itemize}
    \item In the case of a Carnot cycle we can find a relationship between the heat and the temperatures such that:
    \begin{flalign*}
        \eta = 1 - \frac{T_1}{T_2}
    \end{flalign*}
    In the case of a refrigerator the cycle is performed in the opposite direction so we have a \emph{coefficient of performance } defined as:
    \begin{flalign*}
          c =   \frac{Q_1}{Q_2-Q_1}
      \end{flalign*}  
\end{itemize}
% subsubsection carnot_cycle (end)
% subsection efficacy (end)

\subsection{Entropy} % (fold)
\label{sub:entropy}
\begin{itemize}
    \item From the definition, for a reversible process, Entropy is defined as:
    \begin{flalign*}
        dS = \frac{dQ}{T}
    \end{flalign*}
    We have from the Clausius Inequality that $dS \geq 0$. If we have an isobaric process then the heat is just $dQ = C_pdT$ 
\end{itemize}
% subsection entropy (end)

\subsection{Thermal conductivity} % (fold)
\label{sub:thermal_conductivity}
\begin{itemize}
    \item This is a reformulation of Fourier's law:

    $k$ is the \emph{thermal conductivity}. 
\end{itemize}
% subsection thermal_conductivity (end)

\subsection{Thermodynamic potentials} % (fold)
\label{sub:thermodynamic_potentials}

% subsection thermodynamic_potentials (end)
\subsubsection{Helmholtz Free} 
\begin{itemize}
    \item This is $\mathcal{F}(T,V,N_i)$ and is given by:
\begin{empheq}[box=\tcbhighmath]{equation}
\begin{split}
 \mathcal{F} = E-TS
\end{split}
\end{empheq}
We use $\mathcal{F}$ in the case of an isothermal process along with there being chemical and mechanical isolation. $\mathcal{F}$ also has:
\begin{empheq}[box=\tcbhighmath]{equation}
\begin{split}
 - P = \left(\frac{\partial F}{\partial V}\right),~~~ - S = \left(\frac{\partial F}{\partial T}\right),~~~ \mu_i = \left(\frac{\partial F}{\partial N_i}\right)
\end{split}
\end{empheq}

\end{itemize}
\subsubsection{Gibbs free}
\begin{itemize}
    \item This is $\mathcal{G}(T,P,N_i)$ and is given by:
\begin{empheq}[box=\tcbhighmath]{equation}
\begin{split}
  \mathcal{G} = F+PV
\end{split}
\end{empheq}
We use $\mathcal{G}$ in the case of an isothermal and Isobaric system along with chemical and mechanical isolation.  $\mathcal{G}$ also has:
\begin{empheq}[box=\tcbhighmath]{equation}
\begin{split}
  V = \left(\frac{\partial G}{\partial P}\right),~~~ - S = \left(\frac{\partial G}{\partial T}\right),~~~ N_i = \left(\frac{\partial G}{\partial \mu_i}\right)
\end{split}
\end{empheq}
\end{itemize}
\subsubsection{Enthalpy}
\begin{itemize}
    \item This is $\mathcal{H}(S,P,N_i)$ and is given by:
\begin{empheq}[box=\tcbhighmath]{equation}
\begin{split}
\mathcal{H} = E + PV
\end{split}
\end{empheq}
We use $\mathcal{H}$ in the case of an Isobaric process along with thermal and chemical isolation.  $\mathcal{H}$ also has:
\begin{empheq}[box=\tcbhighmath]{equation}
\begin{split}
  P = \left(\frac{\partial H}{\partial V}\right),~~~  S = \left(\frac{\partial H}{\partial T}\right),~~~ \mu_i = \left(\frac{\partial H}{\partial N_i}\right)
\end{split}
\end{empheq}
\end{itemize}

\newpage
\section{Electromagnetism} % (fold)
\label{sec:electromagnetism}
\subsection{Coulombs Law} % (fold)
\label{sub:coulombs_law}
\begin{itemize}
    \item The force of between two charged particles $q_1$ and $q_2$
    \begin{flalign*}
    F = \frac{1}{4\pi\epsilon_{0}}\frac{q_1q_2}{r^2}
    \end{flalign*}
\end{itemize}
% subsection coulombs_law (end)
\subsection{Electric Field} % (fold)
\label{sub:electric_field}
\begin{itemize}
    \item The electric field is related to the force of coulombs law via:
    \begin{flalign*}
\boldsymbol{E} = \frac{\boldsymbol{F}}{q} = \frac{1}{4\pi\epsilon_{0}}\frac{q}{r^2}\hat{r}
    \end{flalign*}
\end{itemize}
% subsection electric_field (end)
\subsection{Electric Potential} % (fold)
\label{sub:electric_potential}
\begin{itemize}
    \item This is defined as, $\phi $ such that $\boldsymbol{E}= - \nabla \phi $. This means the potential difference between two points along some curve is:
    \begin{flalign*}
     \phi = \int_{\gamma}\textbf{E}\cdot d\textbf{l}
     \end{flalign*} 
\end{itemize}
% subsection electric_potential (end)

\subsubsection{Electric Energy Density} % (fold)
\label{ssub:electric_energy_density}
\begin{itemize}
    \item This is:
    \begin{flalign*}
    u = \frac{1}{2}\epsilon_0E^2
    \end{flalign*}
\end{itemize}
% subsubsection electric_energy_density (end)
\subsection{Dipoles} % (fold)
\label{sub:dipoles}
\begin{itemize}
    \item Dipoles have a quantity called the dipole moment $\boldsymbol{\mu}$, this has magnitude $qd$, where $q$ is the  charge and $d$ the separation. he potential energy due to a dipole in an electric field is: 
    \begin{flalign*}
    U = - \boldsymbol{E}\cdot\boldsymbol{\mu} 
    \end{flalign*}
    The torque on a dipole is $\boldsymbol{\tau} = \boldsymbol{\mu}\times \boldsymbol{E}$
\end{itemize}
% subsection dipoles (end)

\subsection{Gauss' Law} % (fold)
\label{sub:gauss_law}
\begin{itemize}
    \item This says the total electric flux (integral of electric field perpendicular to a surface) is the same as the charge enclosed by that surface:
    \begin{flalign*}
    \frac{Q_{\text{encl}}}{\epsilon_0} = \int_{S}\textbf{E}\cdot d\textbf{A}
    \end{flalign*}
\end{itemize}
% subsection gauss_law (end)


\subsection{Capacitors} % (fold)
\label{sub:capacitors}
\begin{itemize}
    \item Capacitance is defined as:
    \begin{flalign*}
    C =\frac{Q}{V}
    \end{flalign*}
    Where $V$ is the potential or voltage. For two plate capacitor of area $A$ and separation $d$, this is $C = \epsilon_0\frac{A}{d}$. 

    \item Capacitors add in series via:
    \begin{flalign*}
    \frac{1}{C} = \frac{1}{C_1}+ \frac{1}{C_2}
    \end{flalign*}
    And in parallel via:
    \begin{flalign*}
    C = C_1 + C_2
    \end{flalign*}
\end{itemize}
% subsection capacitors (end)

\subsubsection{Energy in a Capacitor} % (fold)
\label{ssub:energy_in_a_capacitor}
\begin{itemize}
    \item This is:
    \begin{flalign*}
    U = \frac{Q^2}{2C} = \frac{1}{2}CV^2 = \frac{1}{2}QV
    \end{flalign*}
\end{itemize}
% subsubsection energy_in_a_capacitor (end)

\subsection{Di-electric} % (fold)
\label{sub:Di-electric}
\begin{itemize}
    \item In a dielectric material $\epsilon_0$ changes to $\epsilon=K\epsilon_0$ where $K$ is the dielectric constant. 
\end{itemize}
% subsection  (end)

\subsection{Current} % (fold)
\label{sub:current}
\begin{itemize}
    \item This is defined as:
    \begin{flalign*}
        I = \frac{dQ}{dt} = n|q|vA
    \end{flalign*}
\end{itemize}
% subsection current (end)
\subsubsection{Current density} % (fold)
\label{ssub:current_density}
\begin{itemize}
    \item This is defined as:
    \begin{flalign*}
        \textbf{J} = n|q|\textbf{v}
    \end{flalign*}
\end{itemize}
% subsubsection current_density (end)

\subsection{Resistivity} % (fold)
\label{sub:resistivity}
\begin{itemize}
    \item This is the ratio of electric field to the current density:
    \begin{flalign*}
        \rho = \frac{E}{J}
    \end{flalign*}
    \item The resistivity depends on the temperature via:
    \begin{flalign*}
        \rho(T) = \rho_0(1+\alpha(T-T_0))
    \end{flalign*}
    Where $\alpha$ is the \emph{temperature coefficient of resistivity}.
\end{itemize}
% subsection resistivity (end)

\subsection{Resistance} % (fold)
\label{sub:resistance}
\begin{itemize}
    \item Ohms law is that the potential is related linearly to the current via the resistance:
    \begin{flalign*}
        V = IR
    \end{flalign*}
    If we have a resistor of length of length $L$ and cross section $A$:
    \begin{flalign*}
         R = \frac{\rho L}{A}
     \end{flalign*} 
\end{itemize}
% subsection resistance (end)

% section electromagnetism (end)
\end{document}
