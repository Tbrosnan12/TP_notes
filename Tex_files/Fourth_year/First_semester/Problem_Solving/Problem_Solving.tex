\documentclass[11pt]{article}

%Format and referencing 
\usepackage[letterpaper,top=2cm,bottom=2cm,left=2cm,right=2cm,marginparwidth=1.75cm]{geometry} % for setting margins VERY IMPORTANT!
\usepackage{hyperref}  % for referencing equations 
\usepackage{biblatex}  % for referencing articles 
\addbibresource{Bib.bib}


%math packages 
\usepackage{mathtools}  % allows pre-scripts and other neiche math features
\usepackage{amsmath} % equation related commands 
\usepackage{amssymb} % Fancy R ,C and other set fonts using mathbb{}
\usepackage{braket}   % for Dirac notation 
\usepackage{mathrsfs}  % for the mathscr text 

% miscellaneous
\usepackage{empheq}   % for formatting custon math boxs
\usepackage{xcolor}   % allows the defining of colours 
\usepackage[most]{tcolorbox}  % custom math boxs
\usepackage[utf8]{inputenc}   % redundant since 2018? 
%(I'm not taking out in case it breaks stuff)

\usepackage{graphicx}   % alows image insertion 
\usepackage{float}  % for making images o where you want 
\usepackage{parskip} % for getting red of paragraph indent 

\usepackage{comment} %easier multi line comments 
 \usepackage{tabularx} % for formatting tables 
 \usepackage{titling} % for making Title a variable that can be used in header 
 \usepackage{environ} % for creating and modifying environments 
 \usepackage[explicit]{titlesec}
 % for making section titles variables that can be used in header 
\usepackage{fancyhdr} % for fancy headers 

% Random commands and what not 
\numberwithin{equation}{section}

\setlength{\droptitle}{3em} 

\title{Problem Solving}
\author{Thomas Brosnan}
\date{Sheet of equations useful for the Problem solving paper}

\DeclarePairedDelimiterXPP\BigOSI[2]%
  {\mathcal{O}}{(}{)}{}%
  {\SI{#1}{#2}}

\newtcbox{\mymath}[1][]{%
    nobeforeafter, math upper, tcbox raise base,
    enhanced, colframe=blue!30!black,
    colback=blue!30, boxrule=1pt,
    #1}
\tcbset{highlight math style={boxsep=2mm,,colback=blue!0!green!0!red!0!}}

\newenvironment{bux}{\empheq[box=\tcbhighmath]{align}}{\endempheq}
\newenvironment{bux*}{\empheq[box=\tcbhighmath]{align*}}{\endempheq}
\renewenvironment{flalign}{\vspace{-3mm}\empheq[box=\tcbhighmath]{align}}{\endempheq}
\renewenvironment{flalign*}{\vspace{-3mm}\empheq[box=\tcbhighmath]{align*}}{\endempheq}
%\renewenvironment{align}{\vspace{-5mm}\begin{align}}{\end{align}}
%\renewenvironment{align*}{\vspace{-5mm}\begin{align*}}{\end{align*}}
\renewenvironment{alignat}{\empheq{align*}}{\endempheq}


\newcommand{\hsp}{\hspace{8pt}}

\newcommand*{\sectionFont}{%
  \LARGE\bfseries
}

\newenvironment{eq}{\begin{equation}}{\end{equation}}
    


\makeatletter
\let\Title\@title % Copy the title to a new command
\makeatother

%change this RGB value to change the section background colour 
\definecolor{mycolor1}{RGB}{125, 187, 242}
\colorlet{SectionColour}{mycolor1}
%subsection background colour 
\definecolor{mycolor2}{gray}{0.8}
\colorlet{subSectionColour}{mycolor2}
%subsubsection background colour 
\definecolor{mycolor3}{RGB}{255,255,255}
\colorlet{subsubSectionColour}{mycolor3}



\begin{document}

\maketitle

\newpage
\topskip0pt
\vspace*{\fill}
\begin{center}
\Large
    "Mo money mo Problems"
    
    - Samson
\end{center}
\vspace*{\fill}
\newpage 
\tableofcontents
% For \section
 \titleformat{\section}[block]{\sectionFont}{}{0pt}{%
 \fcolorbox{black}{SectionColour}{\noindent\begin{minipage}{\dimexpr\textwidth-2\fboxsep-2\fboxrule\relax}\thesection  \hsp #1 {\strut} \end{minipage}}}
% For \subsection
 \titleformat{\subsection}[block]{\bfseries}{}{0pt}{%
 \fcolorbox{black}{subSectionColour}{\noindent\begin{minipage}{\dimexpr\textwidth-2\fboxsep-2\fboxrule\relax}\thesubsection  \hsp #1 {\strut} \end{minipage}}}
% For \section*
 \titleformat{name=\section, numberless}[block]{\sectionFont}{}{0pt}{%
 \fcolorbox{black}{SectionColour}{\noindent\begin{minipage}{\dimexpr\textwidth-2\fboxsep-2\fboxrule\relax} #1 {\strut} \end{minipage}}}
  % For \subsection*
 \titleformat{name=\subsection, numberless}[block]{\bfseries}{}{0pt}{%
 \fcolorbox{black}{subSectionColour}{\noindent\begin{minipage}{\dimexpr\textwidth-2\fboxsep-2\fboxrule\relax} #1 {\strut} \end{minipage}}}
 % For \subsubsection
 \titleformat{\subsubsection}[block]{\bfseries}{}{0pt}{%
 \fcolorbox{black}{subsubSectionColour}{\noindent\begin{minipage}{15cm}\thesubsubsection \hsp #1 {\strut} \end{minipage}}}
  % For \subsubsection*
 \titleformat{name=\subsubsection, numberless}[block]{\bfseries}{}{0pt}{%
 \fcolorbox{black}{subsubSectionColour}{\noindent\begin{minipage}{15cm} #1 {\strut} \end{minipage}}}
\newpage 
%header and footer
\pagestyle{fancy}
\fancyhf{} % Clear all header and footer fields
\fancyhead[L]{\Title}
\fancyhead[R]{\nouppercase{\leftmark}}
\fancyfoot[C]{-~\thepage~-}
\renewcommand{\headrulewidth}{1pt}





%starting document 
\normalsize
\newpage
\section{Mechanics}
\subsection{Constant acceleration}
\begin{itemize}
    \item For constant acceleration $a$ the equations of motion can be written in three useful forms. The variables here are initial velocity $u$ final velocity $v$, time $t$ and displacement $s$:
    \begin{flalign*}
        & v = u + at \\
        & v^2 =u^2 +2as \\
        & s = ut + \frac{1}{2}at^2
    \end{flalign*}
\end{itemize}
\subsection{Force}
\begin{itemize}
    \item The force due to a potential $U$ is:
    \begin{flalign*}
        \textbf{F} = - \nabla U 
    \end{flalign*}
\end{itemize}

\subsection{Friction}
\begin{itemize}
    \item Friction is given by the following expression, where $\mu$ is the \emph{co-efficient of friction} and $N$ is the normal force acted on the object by the surface it is sliding across. 
    \begin{flalign*}
         F_{\text{Frict}} = \mu N
     \end{flalign*} 
\end{itemize}

\subsection{Work}
\begin{itemize}
    \item For constant Force work is just defined $W =\textbf{F}\cdot\textbf{s}$. (Force times displacement). If the force is not constant:
    \begin{flalign*}
        W = \int_{a}^{b}\textbf{F} \cdot d\textbf{s}
    \end{flalign*} 
\end{itemize}
\subsubsection{Power}
\begin{itemize}
        \item \emph{Power} is defined as:
    \begin{flalign*}
        P = \frac{dW}{dr} = \textbf{F}\cdot \textbf{v}
    \end{flalign*}
    Where $\textbf{v}$ is velocity.
\end{itemize}

\subsection{Circular motion \& Rotation}
\begin{itemize}
    \item In circular motion the force acting as the \emph{centripetal} force (force pulling to wards the center, gravity, tension in rope, ect) is balanced by a \emph{centrifugal} force equal in magnitude but opposite direction, given by:
    \begin{flalign*}
        F_{\text{fugal}} = \frac{mv^2}{r} = mr\omega^2, ~~~(\text{as}~v=\omega r)
    \end{flalign*}
    Where here $v$ is the tangential velocity, $r$ is the radius, $m$ is the mass of the object in motion and $\omega$ is the angular velocity. 
\end{itemize}
\subsubsection{Rotational Kinetic energy}
\begin{itemize}
    \item Usually kinetic energy is just $\frac{1}{2}mv^2$, but for rigid bodies rotation it is:
    \begin{flalign*}
        K = \frac{1}{2}I\omega^2
    \end{flalign*}
    Where $I$ is the moment of inertia. A list of these can be found in the formula and tables booklet for different geometries. 
\end{itemize}
\subsubsection{Parallel axis Theorem}
\begin{itemize}
    \item This is the rule that tells us how to add the self rotation moment of inertia $I_{\text{self}}$ to the orbital rotation of a mass $M$ at radius $R$:
    \begin{align*}
        I_{\text{total}} = I_{\text{self}} + MR^2
    \end{align*}
\end{itemize}
\subsubsection{Torque}
\begin{itemize}
    \item This is defined as:
    \begin{flalign*}
        \boldsymbol{\tau} = \textbf{r} \times \textbf{F}
    \end{flalign*}
    We also have that:
    \begin{flalign*}
        |\boldsymbol{\tau}| = \alpha I
    \end{flalign*}
    Where $\alpha = \frac{d \omega}{dt}$.  We can then find that the work done by a source the torque is:
    \begin{flalign*}
         W = \int_{\theta_{0}}^{\theta_1}\boldsymbol{\tau}d\theta, ~~~\implies P = \tau \omega
     \end{flalign*} 
     Where $P$ is the power. 
\end{itemize}
\subsubsection{Angular momentum}
\begin{itemize}
    \item The standard definition is just:
    \begin{flalign*}
        & \textbf{L} = \textbf{r}\times \textbf{p}, ~~~\text{for a particle} \\
        &I \boldsymbol{\omega}, ~~~\text{Ridged body motion} 
    \end{flalign*}
    We can also write the torque as $\boldsymbol{\tau} = \frac{d \textbf{L}}{dt}$. 
\end{itemize}
\subsection{Center of Mass}
\begin{itemize}
    \item This is given by:
    \begin{flalign*}
        \textbf{R}_{CM} = \frac{\sum_{i}m_i\textbf{r}_i}{\sum_im_i} 
    \end{flalign*}
\end{itemize}
\subsection{Stress/strain}
\begin{itemize}
    \item Shear modulus $G$ is:
    \begin{flalign*}
        G = \frac{\text{shear stress}}{\text{shear strain}}
    \end{flalign*}
    Where shear stress $ = F/A$ (Force per Area) and shear strain  = $\frac{\Delta s}{l}$. Length displaced $\Delta s$ over the length of the object $l$. The same sort of equation holds for the other moduli. Youngs modulus is $G$ with the word ``shear'' replaced with ``Tensile''. Same goes with Bulk and Elastic. 
\end{itemize}
\subsubsection{Pressure}
\begin{itemize}
    \item Pressure is in its most general form perpendicular Force $F_{\perp}$ divided by area $A$:
    \begin{flalign*}
        P  = \frac{F_{\perp}}{A}
    \end{flalign*}
\end{itemize}

\subsection{Gravitation}
\begin{itemize}
    \item The Gravitational potential energy of a mass $m$ in a gravitational field produced by a body of mass $M$ is:
    \begin{flalign*}
        U = -\frac{GMm}{r} ~ ~~\implies \textbf{F}_{\text{grav}} = -\frac{GMm}{r^2}\hat{r}
    \end{flalign*}

\end{itemize}

\end{document}
