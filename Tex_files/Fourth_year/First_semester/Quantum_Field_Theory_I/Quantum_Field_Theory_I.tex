\documentclass[11pt]{article}

\usepackage[letterpaper,top=2cm,bottom=2cm,left=2cm,right=2cm,marginparwidth=1.75cm]{geometry}
\usepackage{hyperref}
\usepackage{biblatex}
\addbibresource{Bib.bib}
\usepackage{mathtools}
\DeclarePairedDelimiterXPP\BigOSI[2]%
  {\mathcal{O}}{(}{)}{}%
  {\SI{#1}{#2}}
\usepackage{xcolor}
\usepackage{empheq}
\usepackage[most]{tcolorbox}
\usepackage{amsmath}
\usepackage{amssymb}
\usepackage{mathrsfs}
\usepackage[utf8]{inputenc}
\usepackage{graphicx}
\usepackage{float}
\usepackage{parskip}
\usepackage{comment}
%\usepackage{mhchem}
 \usepackage{tabularx}
 \usepackage{titling}
 \usepackage{amsmath,environ}
 \usepackage[explicit]{titlesec}
\usepackage{fancyhdr}
\usepackage{braket}
\setlength{\droptitle}{3em} 

\title{Quantum Field Theory I}
\author{Thomas Brosnan}
\date{Notes taken in Professor Samson Shatashvili class, Michaelmas Term 2024}


\newtcbox{\mymath}[1][]{%
    nobeforeafter, math upper, tcbox raise base,
    enhanced, colframe=blue!30!black,
    colback=blue!30, boxrule=1pt,
    #1}
\tcbset{highlight math style={boxsep=2mm,,colback=blue!0!green!0!red!0!}}

\newenvironment{bux}{\empheq[box=\tcbhighmath]{align}}{\endempheq}
\newenvironment{bux*}{\empheq[box=\tcbhighmath]{align*}}{\endempheq}
\renewenvironment{flalign}{\vspace{-3mm}\empheq[box=\tcbhighmath]{align}}{\endempheq}
\renewenvironment{flalign*}{\vspace{-3mm}\empheq[box=\tcbhighmath]{align*}}{\endempheq}
%\renewenvironment{align}{\vspace{-5mm}\begin{align}}{\end{align}}
%\renewenvironment{align*}{\vspace{-5mm}\begin{align*}}{\end{align*}}






\newcommand{\hsp}{\hspace{8pt}}

\newcommand*{\sectionFont}{%
  \LARGE\bfseries
}

\newenvironment{eq}{\begin{equation}}{\end{equation}}
    
\numberwithin{equation}{section}

\makeatletter
\let\Title\@title % Copy the title to a new command
\makeatother

%change this RGB value to change the section background colour 
\definecolor{mycolor1}{RGB}{125, 187, 242}
\colorlet{SectionColour}{mycolor1}
%subsection background colour 
\definecolor{mycolor2}{gray}{0.8}
\colorlet{subSectionColour}{mycolor2}
%subsubsection background colour 
\definecolor{mycolor3}{RGB}{255,255,255}
\colorlet{subsubSectionColour}{mycolor3}


\begin{document}

\maketitle

\newpage
\topskip0pt
\vspace*{\fill}
\begin{center}
\Large
    "We will work in "God-given" units, where $\hbar = 1 = c$" 

    -Peskin \& Schroeder  
\end{center}
\vspace*{\fill} 
\newpage 
\tableofcontents
% For \section
 \titleformat{\section}[block]{\sectionFont}{}{0pt}{%
 \fcolorbox{black}{SectionColour}{\noindent\begin{minipage}{\dimexpr\textwidth-2\fboxsep-2\fboxrule\relax}\thesection  \hsp #1 {\strut} \end{minipage}}}
% For \subsection
 \titleformat{\subsection}[block]{\bfseries}{}{0pt}{%
 \fcolorbox{black}{subSectionColour}{\noindent\begin{minipage}{\dimexpr\textwidth-2\fboxsep-2\fboxrule\relax}\thesubsection  \hsp #1 {\strut} \end{minipage}}}
% For \section*
 \titleformat{name=\section, numberless}[block]{\sectionFont}{}{0pt}{%
 \fcolorbox{black}{SectionColour}{\noindent\begin{minipage}{\dimexpr\textwidth-2\fboxsep-2\fboxrule\relax} #1 {\strut} \end{minipage}}}
  % For \subsection*
 \titleformat{name=\subsection, numberless}[block]{\bfseries}{}{0pt}{%
 \fcolorbox{black}{subSectionColour}{\noindent\begin{minipage}{\dimexpr\textwidth-2\fboxsep-2\fboxrule\relax} #1 {\strut} \end{minipage}}}
 % For \subsubsection
 \titleformat{\subsubsection}[block]{\bfseries}{}{0pt}{%
 \fcolorbox{black}{subsubSectionColour}{\noindent\begin{minipage}{15cm}\thesubsubsection \hsp #1 {\strut} \end{minipage}}}
  % For \subsubsection*
 \titleformat{name=\subsubsection, numberless}[block]{\bfseries}{}{0pt}{%
 \fcolorbox{black}{subsubSectionColour}{\noindent\begin{minipage}{15cm} #1 {\strut} \end{minipage}}}
\newpage 
%header 
\pagestyle{fancy}
\fancyhf{} % Clear all header and footer fields
\fancyhead[L]{\Title}
\fancyhead[R]{\nouppercase{\leftmark}}
\fancyfoot[C]{-~\thepage~-}
\renewcommand{\headrulewidth}{1pt}





%starting document 
\normalsize
\newpage
\section{QFT trailer}
\begin{itemize}
  \item We wish to see if we can perform a calculation in quantum field theory, just by elementary means, i.e. via dimensional analysis ect.

  Consider the head on collision of an electron $e^{-}$ and $e^{+}$ that results in the production of a muon $\mu^{-}$ anti muon $\mu^{+}$ pair, shown below: 
\begin{figure}[H]
\centering
\includegraphics[width=0.6\textwidth]{trailer.png}
\caption{\label{trailer}\emph{electron positron annihilation}}
\end{figure}


\item The calculation we would like to perform is the differential cross section, that is the derivative of the cross section $\sigma$ with respect to the solid angle $\Omega$, $\frac{d \sigma}{d \Omega}$. This is a useful quantity as it is easily experimentally observed. In a particle collider, electrons and positrons are prepared in batches of length $l_A$ and $l_B$ and densities $\rho_A$ and $\rho_B$ respectively. When the two batches collide if the over lapping area of the head on collision is A, then the cross section is given by:
\end{itemize}

\begin{equation*}
  \sigma = \frac{\text{Number of events}}{\rho_A\rho_B l_A l_B A}
\end{equation*}

\begin{itemize}
  \item We now look at the dimensions of our quantities. Conveniently from our use of God given units, i.e. $\hbar =c =1$ we have that momentum, mass and energy have the same units as the energy mass equivalence becomes $E^2 = p^2 + m^2$. We also have from the Heisenberg's uncertainty principle that $\Delta p \Delta x \sim 1$. Thus the dimensions of mass and length are inversely related. 

  We can from this easily see that the dimensions of the quantity $[\rho_A\rho_B l_A l_B A]$ is $[m^2]$, which makes the dimensions of $\left[\frac{d\sigma}{d \Omega}\right] = \left[\frac{1}{m^2}\right]$ as angles are unitless. With this we can say that this quantity is also inversely proportional to the energy squared times some positive quantity that depends on the angle:


\begin{flalign}
\label{diff_cross}
  \frac{d\sigma}{d \Omega} \propto \frac{1}{E^2}|\mathcal{M}(\theta)|^2 
\end{flalign}
Here $\mathcal{M}$ is a dimensionless quantity, that is essentially the quantum mechanical amplitude for the process to occur. It does not depend on the energy $E$, as we are considering the limit where $E >> m_e,m_{\mu}$. This means we are unable to construct any dimensionless quantity like $E/m_e$ or $E/m_{\mu}$ as we have set $m_e/E = 0 = m_{\mu}/e$. Note that we could not take this limit and them $\mathcal{M}$ would depend on $E$, but it is simpler to consider the high energy limit. We will later calculate what the constant of proportionality in this equation is, and it turn out to be $1/64\pi^2$.
\end{itemize}
\begin{itemize}
  \item If we recall from Quantum mechanics, in perturbation theory we had that at first order the transition amplitude is related to the initial and final states along with the interaction Hamiltonian $H_i$, so:

\begin{equation*}
  \mathcal{M} \sim \bra{\text{final state}} H_I \ket{\text{initial state}}
\end{equation*}
But we know physically that the electrons do not interact with the muons, Instead what we know happens is that the electrons annihilate to form photons which in turn form the muons. This then means that $\bra{\mu^+ \mu^-} H_I \ket{e^+e^-} = 0 $ and instead we have that:

\begin{flalign}
\label{second order M}
  \mathcal{M} \sim \bra{\mu^+ \mu^-} H_I \ket{\gamma}^{\alpha}\bra{\gamma}H_I\ket{e^+e^-}_{\alpha}
\end{flalign}

This is a heuristic way of writing this second order contribution, but it makes sense physically as we have the electron positron pair interacting to become a photon ($\ket{\gamma}\bra{\gamma}$) and then the photon in turn becoming muons. Note the addition of vector indices ($\alpha$) as the photon is a vector particle (non-zero spin), so the photon created has 4 intermediate states, 3 for spin as it has spin-1 and one extra that comes from the fact that we are adding angular momenta in the four dimensional Lorentz group and must consider boosts also. (Remember the three spin components are generated by the three angular momentum operators and there is a corresponding generator for boosts). 
\end{itemize}

\begin{itemize}

  \item Since the photon must conserve angular momentum going to or from either of the two particles, the photon vector must be in the same direction as the axes of the particle pairs. We also know that the strength of the coupling between electrons (or muons) and photons is given by the electric charge $e$. This means for the case where electron has spin up along x axis, the positron has spin down:
\begin{equation*}
  \bra{\gamma} H_I \ket{e^+ e^-}^{\alpha} \propto e(0,1,i,0)
\end{equation*}
And if we have the same for muon and anti-muon:
\begin{equation*}
  \bra{\gamma} H_I \ket{\mu^+ \mu^-}^{\alpha} \propto e(0,\cos{\theta},i,-\sin(\theta))
\end{equation*}

The vectors here have the first component as the time component and the last three are part of the the polarization vector of the photon. See Jackson third edition page 299 for these vectors. 


\item When considering the experimental calculation of $\frac{d\sigma}{d \Omega}$ it also easier to account for all possible initial and final spin states, for which we need to take in to account conservation of angular momentum. This means we cant have two right polarized electron and positrons going to a left and right polarized muon and anti-muon. This condition then leaves 4 possible transitions which we can calculate the contributions. These calculations are done by dotting the two vectors we have above as in \ref{second order M}, making sure to take the complex conjugate of the first 4 vector and also remember to properly contract with the $(+---)$ metric. This results in:

\begin{flalign*}
& M(RL \rightarrow RL) \sim -e^2(1+\cos{\theta}) \\
& M(RL \rightarrow LR) \sim -e^2(1-\cos{\theta}) \\
& M(LR \rightarrow RL) \sim -e^2(1-\cos{\theta}) \\
& M(LR \rightarrow LR) \sim -e^2(1+\cos{\theta})
\end{flalign*}


There are no states that have 0 total angular momentum before and after as it turn out the intermediate photon (despite being a spin-1 boson) cant have spin 0 along an axis. Though in this case we would be requiring that the photon would have 0 total spin which is also not possible. 


\item We can finally take these 4 probabilities and square and sum them to get $|\mathcal{M}|^2$. Resulting in:

\begin{equation*}
  |\mathcal{M}|^2 \sim 4e^2(1+\cos^2{\theta})
\end{equation*}
So using \ref{diff_cross} and its proportionality constant $1/64\pi^2$ we can write:

\begin{equation*}
  \frac{d \sigma}{d \Omega} = \frac{e^2}{32E^2}(1+\cos^2{\theta})
\end{equation*}
Defining the constant $\alpha = e^2/4\pi \sim 137^{-1}$, we can then integrate this to get a expression for the total cross section as:

\begin{flalign*}
  \sigma = \frac{4\alpha^2\pi}{3E^2}
\end{flalign*}
This is the correct first order approximation!
\end{itemize}


\newpage  

\section{The need for Fields}
In this section we will see where regular quantum mechanics fails and what we need to do to fix it. 

\subsection{Non-relativistic free particle}

\begin{itemize}
  \item We can recall from QM that the probability of a particle at point $x$ at time $t$ propagating to $x'$ at time $t'$ is given by:
  \begin{equation}
  \label{propagator}
    \bra{x} e^{-iH(t-t')} \ket{x'}
  \end{equation}
  Here $H$ is the Hamiltonian. For a Non-relativistic free particle we have that $H = \hat{\textbf{P}}^2/2m$. We can then go about solving dor this propagator with this Hamiltonian in the usual way. This involves inserting the identity $\int \frac{d^3p}{(2\pi)^3}\ket{p}\bra{p} = \mathbb{I}$ into the above propagator \ref{propagator}before the $\ket{x'}$

  \begin{equation*}
      \bra{x} e^{-iH(t-t')} \ket{x'}  = \int \frac{d^3p}{(2\pi)^3}\bra{x}e^{-i\frac{\hat{\textbf{P}}^2}{2m}(t-t')}\ket{p}\braket{p|x'}
  \end{equation*}
  Then if we recall that $\braket{p|x'} = e^{-i\textbf{p}\cdot\textbf{x}'}$, $\braket{x|p} = e^{i\textbf{p}\cdot\textbf{x}}$ and $e^{-i\frac{\hat{\textbf{P}}^2}{2m}(t-t')}\ket{p} = e^{-i\frac{p^2}{2m}(t-t')}\ket{p}$. We get:

  \begin{flalign*}
    \bra{x} e^{-iH(t-t')} \ket{x'} & = \int \frac{d^3p}{(2\pi)^3}e^{-i\frac{p^2}{2m}(t-t')}e^{i\textbf{p}\cdot(\textbf{x}-\textbf{x}')} \\
    & = \left(\frac{m}{2\pi i(t-t')}\right)^{3/2}e^{im\frac{(\textbf{x}-\textbf{x}')^2}{2(t-t')}}
  \end{flalign*}
  \item What this is saying, is that for any two points $x$ and $x'$, no matter how far they are separated, have a non-zero probability of propagation from one to another. But this is direct contrast with what we know from special relativity! Two points separated by enough distance so that there space time interval, $\Delta s^2 = (t-t')^2 -|\textbf{x}-\textbf{x}'|^2$ is negative, i.e. space like. Which would require a propagation speed faster then light. 
\end{itemize}

\subsection*{Relativistic free particle}
\begin{itemize}
  \item But we can chalk this up to a mistake. We were not considering the Hamiltonian of a \textbf{relativistic} free particle. That is, with the Hamiltonian $H  = \sqrt{\textbf{P}^2 + m^2}$. The propagator in a similar manner to before then becomes: 
  \begin{equation*}
  \begin{split}
     \bra{x} e^{-i\sqrt{\textbf{P}^2 + m^2}(t-t')} \ket{x'} & = \int \braket{x|p}  e^{-i\sqrt{\textbf{P}^2 + m^2}(t-t')}  \braket{p|x'}\frac{d^3p}{(2\pi)^3} \\
     & = \int \frac{d^3p}{(2\pi)^3}  e^{-i\sqrt{p^2 + m^2}(t-t')+ i\textbf{p}\cdot(\textbf{x}-\textbf{x}')} 
  \end{split}
  \end{equation*}
  We can then parametrize the angle part of this integral by letting $\theta$ be the angle between $\textbf{p}$ and $(\textbf{x}-\textbf{x}')$, we can also further parametrize this with $\eta = \cos{\theta}$:
  \begin{equation*}
  \begin{split}
  \bra{x} e^{-i\sqrt{\hat{\textbf{P}}^2 + m^2}(t-t')} &\ket{x'}  = \int_{0}^{\infty} \frac{p^2dp}{(2\pi)^2}  \int_{-1}^{1} d \eta e^{-i\sqrt{p^2 + m^2}(t-t')+ ip|\textbf{x}-\textbf{x}'|\eta} \\
   = \int_{0}^{\infty} \frac{p^2dp}{(2\pi)^2}& e^{-i\sqrt{p^2 + m^2}(t-t')} \frac{1}{ip|\textbf{x}- \textbf{x}'|}\left(e^{ ip|\textbf{x}-\textbf{x}'|}-e^{- ip|\textbf{x}-\textbf{x}'|} \right)
  \end{split}
  \end{equation*}
  We can then turn this into an integral from $-\infty$ to $\infty$ with a factor of 2:
  \begin{equation}
  \label{Rel_propagator}
  \begin{split}
 = \frac{-i}{2\pi^2|\textbf{x}- \textbf{x}'|}\int_{0}^{\infty}pdp~e^{-i\sqrt{p^2 + m^2}(t-t')+ip|\textbf{x}-\textbf{x}'|}
  \end{split}
  \end{equation}
\end{itemize}
\subsubsection{Laplace steepest decent}
\begin{itemize}
  \item We can then use a useful approximation of integrals of this form by expanding around critical points. That is points $x_0$ where $f'(x_0)= 0$. The approximation is as follows, for a critical point $x_0$ of $f(x)$:
  \begin{flalign*}
\int_{-\infty}^{\infty}&h(x)e^{Af(x)}dx =  \int_{-\infty}^{\infty}h(x)e^{A\left[f(x_0) + \frac{1}{2}f''(x_0)(x-x_0)^2+\mathcal{O}((x-x_0)^3)\right]}dx \\ 
 \approx   h(x_0)e^{Af(x_0)}&\int_{-\infty}^{\infty}e^{-A\frac{1}{2}|f''(x_0)|(x-x_0)^2}dx = \sqrt{\frac{2\pi}{A|f''(x_0)|}}h(x_0)e^{Af(x_0)} 
  \end{flalign*}
  \item Where we have assumed $x_0$ is a global maxima so that $f''(x_0) \leq 0$. This approximation works well as the exponential ensures that any small deviation from $x_0$ contributes very little to the integral. It of course depends also on $h(x)$ being "well behaved".


  \item Back to the relevant integral \ref{Rel_propagator}, we can see that for us $f(p) = -i\sqrt{p^2 + m^2}(t-t')+ip|\textbf{x}-\textbf{x}'|$ and $h(p) = p$. So we find the stationary point of $f(p)$ via $f'(p_0)=0$: 
  \begin{equation*}
  \begin{split}
      f'(p_0) = \frac{-p_0(t-t')}{\sqrt{p_0^2+m^2}}+|\textbf{x}-\textbf{x}'| =0 \implies p_0^2 = \frac{m^2|\textbf{x}-\textbf{x}'|^2}{(t-t')^2-|\textbf{x}-\textbf{x}'|^2}
      \end{split}
    \end{equation*}  
    Then we can evaluate $f(p_0)$:
    \begin{equation*}
    \begin{split}
      f(p_0) & = -i\sqrt{\frac{m^2|\textbf{x}-\textbf{x}'|^2}{(t-t')^2-|\textbf{x}-\textbf{x}'|^2}+m^2}(t-t')+i\sqrt{\frac{m^2|\textbf{x}-\textbf{x}'|^2}{(t-t')^2-|\textbf{x}-\textbf{x}'|^2}} \\
      & = -\frac{m(|\textbf{x}-\textbf{x}'|^2-(t-t')^2)}{\sqrt{|\textbf{x}-\textbf{x}'|^2-(t-t')^2}} = -m\sqrt{|\textbf{x}-\textbf{x}'|^2-(t-t')^2}
    \end{split}
    \end{equation*}
    Where we have used the fact that we are considering probabilities outside the light-cone, i.e. where $|\textbf{x}-\textbf{x}'| > (t-t')$, so that the square root $ \sqrt{(t-t')^2-|\textbf{x}-\textbf{x}'|^2}=i\sqrt{|\textbf{x}-\textbf{x}'|^2-(t-t')^2}$ . 

    \item With this we can now use Laplace's steepest decent integral approximation to write:
    \begin{flalign*}
    \bra{x} e^{-i\sqrt{\hat{\textbf{P}}^2 + m^2}(t-t')} \ket{x'} \propto h(p_0)e^{f(p_0)} \propto e^{-\sqrt{|\textbf{x}-\textbf{x}'|^2-(t-t')^2}}
    \end{flalign*}
    This again does not solve our problem, we have that there is a non-zero probability of propagating to outside the light-cone. Some more radical approach is needed to solve this problem. 
\end{itemize}

\subsection{Field theory EoM}
\begin{itemize}
  \item The  Idea will be to go from dealing with particle, to waves. For this we need to generalize our idea of action being the integral of a Lagrangian over time to being the integral of a Lagrange density over time and space as a field is spread out, not localized. This means:
  \begin{equation*}
    S = \int L(q_i,\dot{q}_i,t)dt \rightarrow \int d^4x\mathcal{L}(\varphi(\textbf{x},t),\partial_{\mu} \varphi(\textbf{x},t))
  \end{equation*}
  We then have to vary this action to set $\delta S = 0$:
  \begin{equation*}
    \delta S = \int d^4x \left(\frac{\delta \mathcal{L}}{\delta \varphi}\delta \varphi  - \frac{\delta \mathcal{L}}{\delta \partial_{\mu} \varphi}\delta\partial_{\mu}\varphi\right) = \int d^4x \left(\frac{\delta \mathcal{L}}{\delta \varphi}  - \partial_{\mu}\left(\frac{\delta \mathcal{L}}{\delta \partial_{\mu} \varphi}\right)\right)\delta \varphi 
  \end{equation*}
  \begin{flalign}
  \label{EoM}
    \implies \frac{\delta \mathcal{L}}{\delta \varphi}  - \partial_{\mu}\frac{\delta \mathcal{L}}{\delta \partial_{\mu} \varphi} = 0 
  \end{flalign}
  Where we have integrated by parts the second term.
\end{itemize}

\subsection{Non-degenerate Lagrangian}
\begin{itemize}
  \item It is often the case, that when we have constructed our Lagrangian, that we would like to perform a Legendre transform from the variables $q$ and $\dot{q}$, to $q$ and $p$. This is gives us our Hamiltonian and takes the form $H =\sum_i p_i \dot{q}_i  - L(q,\dot{q})$. In doing this we are required to solve for $\dot{q}$ in terms of $q$ and $p$. But this is not always possible, for example given the Lagrangian $L \propto \dot{q} \implies p = \frac{\partial L}{\partial q} = 1$, so we are unable to solve for $\dot{q}(p,q)$. It turn out the condition for us to always be able to solve for $\dot{q}$ is as follows:

  If we denote the matrix $M_{ij} = \frac{\partial^2 L}{\partial\dot{q}^i\partial\dot{q}^j}$, then the condition becomes, we can solve for $\dot{q}_{i} \iff \text{det}(M) \neq 0$. The Lagrangian can then be written as:

  \begin{equation*}
  L = \sum_{i,j}M_{ij}\partial\dot{q}^i\partial\dot{q}^j, ~~~ \text{and}~~~ \dot{q}^i  = \sum_{j}M^{-1}_{ij}p_j
  \end{equation*}
  This is called a Non-degenerate Lagrangian. 
\end{itemize}

\subsection{Hamiltonian field theory} 
\begin{itemize}
  \item We wish to find a Hamiltonian for our field theory. The best way to go about extending our definition of $H =\sum_i p_i \dot{q}_i  - L(q,\dot{q})$, is to think of space as a discretized space, as we had the co-ordinates $q$ indexed by $i$. We can then calculate:

  \begin{equation*}
  p(\textbf{x}) = \frac{\partial L}{\partial \dot{\phi}} = \frac{\partial }{\partial \dot{\phi}(\textbf{x})}\sum_{\textbf{y}}\mathcal{L}(\phi(\textbf{y}),\dot{\phi}(\textbf{y}))d^3y = \pi(\textbf{x})d^3x. 
  \end{equation*}
  Where:
  \begin{flalign}
  \label{pi}
  \pi(\textbf{x}) = \frac{\partial \mathcal{L}}{\partial \dot{\phi}(\textbf{x})}
  \end{flalign}
  This is called the \emph{momentum density} conjugate to $\phi(\textbf{x})$. We can now write the Hamiltonian as:
  \begin{equation*}
    H = \sum_{\textbf{x}}p(\textbf{x})\dot{\phi}(\textbf{x}) - L
  \end{equation*}
  Which in the continuum limit becomes:
  \begin{flalign}
\label{Hamiltonian}
     H = \int d^3x[\pi(\textbf{x})\dot{\phi}(\textbf{x}) - \mathcal{L}] \equiv \int d^3x\mathcal{H} 
   \end{flalign} 
\end{itemize}


\subsection{Noether's Theorem}
\begin{itemize}
  \item We are used to Noether's Theorems relation between symmetries and conservation in the context of particles. We will now discuss this in the context of fields. If we have an infinitesimal transformation of fields taking the form:
  \begin{equation*}
     \phi(x) \rightarrow \phi'(x) = \phi(x) + \alpha\Delta\phi(x)
   \end{equation*} 
Here $\alpha$ is small. This transformation is a symmetry if it results in the same EoM, i.e. leaves the action unchanged. This means the Lagrangian must be the same up to the addition of a total derivative, which in this context is the 4-gradient:
\[
      \mathcal{L}(x) \rightarrow \mathcal{L}(x) + \alpha \partial_{\mu}\mathcal{J}^{\mu}(x)
\]
    Where $\mathcal{J}^{\mu}$ is some 4-vector. We can then find the expected form of $\Delta \mathcal{L}(\phi,\partial_{\mu}\phi )$:
    \begin{align*}
     \alpha \Delta \mathcal{L} & = \frac{\partial \mathcal{L}}{\partial \phi}(\alpha \Delta \phi) + \left(\frac{\partial \mathcal{L}}{\partial (\partial_{\mu}\phi)}\right)\partial_{\mu}(\alpha\Delta\phi) \\
    &  = \alpha \partial_{\mu}\left(\frac{\partial \mathcal{L}}{\partial(\partial_{\mu}\phi)}\Delta \phi\right)+\alpha\left[\frac{\partial\mathcal{L}}{\partial \phi} - \partial_{\mu}\left(\frac{\partial \mathcal{L}}{\partial(\partial_{\mu}\phi)}\right)\right]\Delta \phi 
    \end{align*}
    We can then recognize that the second term vanished via our EoM \ref{EoM}, so we can set the remaining term to $\alpha \partial_{\mu}\mathcal{J}^{\mu}(x)$, which is equivalent to:
\begin{flalign}
\label{Noether}
  \partial_{\mu}j^{\mu}(x) = 0, ~~~~\text{for} ~~j^{\mu}(x) = \frac{\partial \mathcal{L}}{\partial(\partial_{\mu}\phi)}\Delta \phi - \mathcal{J}^{\mu}
\end{flalign}
\item If the symmetry involves more then one field the first term above should be a sum of those terms, for the different fields. This conservation law also implies that there is a charge constant in time namely:
\begin{flalign*}
Q \equiv \int j^{0} d^3x
\end{flalign*}
\end{itemize}
\subsection{Stress-Energy Tensor}
\begin{itemize}
\item We can also consider transformations of space itself, for example, translations or rotations. This takes the for $x^{\mu} \rightarrow x^{\mu} - a^{\mu}$, which has the following affect on the fields (when infinitesimal): 

\[ 
  \phi(x) \rightarrow \phi(x+a) = \phi(x) + a^{\mu}\partial_{\mu}\phi(x)
 \]
The Lagrangian must also transform in the same way as it is also a scalar:
\[ 
  \mathcal{L}(x) \rightarrow \mathcal{L}(x+a) = \mathcal{L}(x) + a^{\nu}\partial_{\mu}(\delta^{\mu}_{\nu}\mathcal{L}(x))
 \]
 We can then recognize the $\delta^{\mu}_{\nu}\mathcal{L}(x)$ as $ \sim \mathcal{J}^{\mu}$ from \ref{Noether}, however since the change in the scalar field became a vector $\Delta \phi \rightarrow \partial_{\mu}\phi(x)$, then we must add a second index to our $\mathcal{J}^{\mu} \rightarrow \mathcal{J}^{\mu}_{~\nu}$. This is for the reason for the $\delta_{\nu}^{\mu}$ in the above expression and means our conserved vector $j^{\mu}$ becomes a conserved tensor of the form:

 \begin{flalign*}
      \partial_{\mu}T^{\mu}_{~\nu}(x) = 0, ~~~~\text{for} ~~T^{\mu}_{~\nu}(x) = \frac{\partial \mathcal{L}}{\partial(\partial_{\mu}\phi)}\partial_{\nu}\phi - \delta^{\mu}_{\nu}\mathcal{L}
  \end{flalign*} 
  This is the \emph{Stress-Energy Tensor} also called the \emph{Energy-Momentum Tensor}. We can then notice that the conserved charge associated with the time translations is nothing more then out Hamiltonian:
  \[
\int T^{00}d^3x = \int\left(\frac{\partial \mathcal{L}}{\partial\dot{\phi}}\dot{\phi} - \mathcal{L}\right)d^3x = \int \mathcal{H}d^3x = H  
  \] 
  Similarly for space translation the conserved charges are the physical momenta, not to be confused with the canonical momentum.
  \[  
  \int T^{0i}d^3x = \int \mathcal{P}_id^3x = P_i 
  \] 
\end{itemize}
\newpage
\section{The Klein-Gordan Field}
\begin{itemize}
  \item We ease ourselves into the concepts of quantum fields with the discussion of the Klein-Gordan field. This is one of the simplest types of fields and its use becomes obvious when we see the EoM, which is just the Schr\"odinger equation but made relativistic by replacing $\hat{\textbf{p}}^2/2m \rightarrow \sqrt{\hat{\textbf{p}}^2 + m^2}$. We will see more of this later.

     \item The Lagrange density for the Klein-Gordan field takes the form:
     \begin{equation}
       \label{KG field}
       \mathcal{L} = \frac{1}{2}\partial_{\mu} \phi \partial^{\mu}\phi -\frac{m^2}{2}\phi^2
     \end{equation}   
From this we can calculate the momentum density via \ref{pi} to get $\pi = \dot{\phi}$. And thus via \ref{Hamiltonian} the Hamiltonian density is:
\begin{equation}
\label{KG_Ham}
  \mathcal{H} = \frac{1}{2}\left(\pi^2 +(\nabla\phi)^2+m^2\phi^2\right) 
\end{equation}
\item The equations of motion can also be calculated easily via \ref{EoM} resulting in the \emph{Klein-Gordan equation}:
\begin{flalign}
\label{KG eq}
  (\partial_{\mu}\partial^{\mu}+m^2)\phi  = 0
\end{flalign} 
This $\partial_{\mu}\partial^{\mu}$ is often denoted $\Box$. As an aside, if we "make" the Schr\"odinger equation relativistic, i.e. $\textbf{p}^2 + m^2 = E^2$ and use the definition of the operators $ i\partial_t\psi= E\psi $ and $\hat{\textbf{p}} = -i\nabla$, so:  $-\nabla^2 + m^2 = -\partial^2_t$, recovering us the above Klein-Gordan equation.  
\end{itemize}

\subsection{Second Quantization}
\begin{itemize}
  \item We will now see if we can ``quantize'' this field. What we can do to help figure out how to do this is to take inspiration from the quantization of the single particle Harmonic oscillator in QM. This works nicely as we can see the Lagrangian density for the Klein-Gordan field \ref{KG field} resembles that of the harmonic oscillator. By quantize here we mean to reinterpret the dynamical variables as operators that obey canonical commutation relations, like $[p_i,q_j] = i\delta_{i,j},~[q_i,q_j] = 0 = [p_i,p_j]$. Instead however $\phi$ and $\pi$ will become our operators and we will require that $[\phi(\textbf{x}),\pi(\textbf{y})] = i \delta(\textbf{x}-\textbf{y}),~ [\phi(\textbf{x}),\phi(\textbf{y})] = 0 = [\pi(\textbf{x}),\pi(\textbf{y})]$. This procedure is often called \emph{Second Quantization} to distinguish it from the old one particle version. 
  
  \item Our goal is to now find the spectrum of the Hamiltonian. We start by taking our field $\phi(\textbf{x})$ and writing it in terms of its Fourier transform to momentum (p)-space
\[
  \phi(\textbf{x}) = \int \frac{d^3p}{(2\pi)^3}\phi(\textbf{p})e^{i\textbf{p}\cdot\textbf{x}}
\]
Here we also require that $\phi(\textbf{x},t)$ is real so $\phi^{\ast}(\textbf{p}) = \phi(-\textbf{p})$ Then using the Klein-Gordan equation \ref{KG eq} we get the condition:
\[
  \left[\partial_t^2 + p^2+m^2\right]\phi(\textbf{p}) = 0 
\]
Where we have acted the $-\partial^2_{i}$ from $\Box$ on the exponential to pull out the $p^2$ This equation of motion is the same as that of a harmonic oscillator with frequency $\omega_{\textbf{p}} = \sqrt{p^2+m^2}$. 

\item We now recall from QM that in order to solve the harmonic oscillator we introduced the creation ($\hat{a}^{\dagger}$) and annihilation ($\hat{a}$) operators, so that we could write $\hat{q} = \frac{1}{\sqrt{2 \omega}}(\hat{a}+\hat{a}^{\dagger})$ and $\hat{p} = -i\sqrt{\frac{\omega}{2}}(\hat{a}-\hat{a}^{\dagger})$. This then meant the Hamiltonian took the form $H = \frac{1}{2}(\hat{p}^2+\omega^2\hat{q}^2) = \omega(\hat{a}^{\dagger}\hat{a}+\frac{1}{2})$ and also that the commutators  were: $[H,\hat{a}^{\dagger}] = \omega \hat{a}^{\dagger}$ , $[H,\hat{a}] = -\omega \hat{a}$ and $[a,a^{\dagger}] =\mathcal{I}$. 
 
\item This can be done for our field $\phi$, except we have to consider that $\hat{q} = \frac{1}{\sqrt{2 \omega}}(\hat{a}+\hat{a}^{\dagger})$ is a function of momentum as for us we have $\omega_{\textbf{p}} = \sqrt{p^2+m^2}$. We can then recall that since we expect $\phi$ to be real we require that $\phi^{\dagger}(\textbf{p}) = \phi(-\textbf{p})$, this means we should have $\phi(\textbf{p}) \propto a_{\textbf{p}}+a^{\dagger}_{-\textbf{p}}$ as then $\phi(\textbf{p})^{\dagger} \propto a_{\textbf{p}}^{\dagger}+ a_{-\textbf{p}} \propto \phi(-\textbf{p})$. This means our corresponding expression for $\phi$ should take the form:
\begin{equation}
\label{phi_1}
  \phi(\textbf{x}) = \int \frac{d^3p}{(2\pi)^3}\frac{1}{\sqrt{2\omega_{\textbf{p}}}}\left(a_{\textbf{p}}+a^{\dagger}_{-\textbf{p}}\right)e^{i\textbf{p}\cdot\textbf{x}} 
\end{equation} 
\begin{equation}
\label{pi_1}
  \pi(\textbf{x}) = -i\int \frac{d^3p}{(2\pi)^3}\sqrt{\frac{\omega_{\textbf{p}}}{2}}\left(a_{\textbf{p}}-a^{\dagger}_{-\textbf{p}}\right)e^{i\textbf{p}\cdot\textbf{x}} 
\end{equation} 
We can also think of this expression as each Fourier mode of the field being treated as an independent oscillator with its own $a$ and $a^{\dagger}$ (which will also be a function of the momentum). We can then isolate the second term in this integral and change $\textbf{p}$ to $-\textbf{p}$, since we are integrating over all momentum space, this just results in the signs of the $p$'s in the term changing so we can write:

\begin{flalign}
\label{phi_2}
  \phi(\textbf{x}) = \int \frac{d^3p}{(2\pi)^3}\frac{1}{\sqrt{2\omega_{\textbf{p}}}}\left(a_{\textbf{p}}e^{i\textbf{p}\cdot\textbf{x}}+a^{\dagger}_{\textbf{p}}e^{-i\textbf{p}\cdot\textbf{x}}\right) 
\end{flalign}
In a similar manner the momentum density can be written as:

\begin{flalign}
\label{pi_2}
  \pi(\textbf{x}) = -i\int \frac{d^3p}{(2\pi)^3}\sqrt{\frac{\omega_{\textbf{p}}}{2}}\left(a_{\textbf{p}}e^{i\textbf{p}\cdot\textbf{x}}-a^{\dagger}_{\textbf{p}}e^{-i\textbf{p}\cdot\textbf{x}}\right) 
\end{flalign}
\end{itemize}
\subsubsection{Commutators}
\begin{itemize}
\item We can then go about checking the commutator $[\phi(\textbf{x}),\pi(\textbf{x}')]= i \delta(\textbf{x}-\textbf{y})$ which should be equivalent to the the new creation and annihilation operators satisfying $[a,a^{\dagger}] = (2 \pi)^{3}\delta(\textbf{p}-\textbf{p}')$. Using the above definitions \ref{phi_1} and \ref{pi_1} for this calculation is easier, resulting in:
\begin{equation}
\begin{split}
  [\phi(\textbf{x}),\pi(\textbf{x}')] & = -\frac{i}{2}\int \frac{d^3pd^3p'}{(2\pi)^6}\sqrt{\frac{\omega_{\textbf{p}'}}{\omega_{\textbf{p}}}}\left[-a_{\textbf{p}}a^{\dagger}_{\textbf{-p}'} + a^{\dagger}_{\textbf{-p}}a_{\textbf{p}'}-(-a^{\dagger}_{\textbf{p}'}a_{\textbf{-p}} + a_{\textbf{p}}a^{\dagger}_{\textbf{-p}'})\right]e^{i(\textbf{p}\cdot \textbf{x}+\textbf{p}'\cdot \textbf{x}')} \\ 
  &  = -\frac{i}{2}\int \frac{d^3pd^3p'}{(2\pi)^6}\sqrt{\frac{\omega_{\textbf{p}'}}{\omega_{\textbf{p}}}}\left[[a^{\dagger}_{\textbf{-p}},a_{\textbf{-p}'}]-[a_{\textbf{p}},a^{\dagger}_{\textbf{-p}'}]\right]e^{i(\textbf{p}\cdot \textbf{x}+\textbf{p}'\cdot \textbf{x}')}  \\ 
   & =  -\frac{i}{2}\int \frac{d^3pd^3p'}{(2\pi)^6}\sqrt{\frac{\omega_{\textbf{p}'}}{\omega_{\textbf{p}}}}(2\pi)^3\left[-\delta(\textbf{p}'+\textbf{p}) -\delta(\textbf{p}+\textbf{p}') \right]e^{i(\textbf{p}\cdot \textbf{x}+\textbf{p}'\cdot \textbf{x}')}  \\ 
   & = i\int \frac{d^3p}{(2\pi)^3}e^{i\textbf{p}\cdot( \textbf{x} -\textbf{x}')} = i\delta(\textbf{x} -\textbf{x}')  
  \end{split}
  \end{equation}  
  Where we have used the fact that $\omega_{\textbf{p}}=\omega_{\textbf{-p}}$ by definition. We can then also calculate the Hamiltonian via \ref{KG_Ham}:
  \begin{equation*}
    \begin{split}
      H  = \frac{1}{2}\int &d^3x\int\frac{d^3pd^3p'}{(2\pi)^6}\left[\frac{-\sqrt{\omega_{\textbf{p}}\omega_{\textbf{p}'}}}{2}\left(a_{\textbf{p}}-a^{\dagger}_{\textbf{-p}}\right)\left(a_{\textbf{p}'}-a^{\dagger}_{\textbf{-p}'}\right)\right. \\
       & \left. + \frac{-\textbf{p}\cdot \textbf{p}'+m^2}{2\sqrt{\omega_{\textbf{p}}\omega_{\textbf{p}'}}}\left(a_{\textbf{p}}+a^{\dagger}_{\textbf{-p}}\right)\left(a_{\textbf{p}'}+a^{\dagger}_{\textbf{-p}'}\right)\right]e^{i\textbf{x}\cdot(\textbf{p}+\textbf{p}')} \\
    \end{split}
  \end{equation*}
  We can then notice that the $\textbf{x}$ part of this integral creates a delta function $\int d^3xe^{i\textbf{x}\cdot(\textbf{p}+\textbf{p}')} = (2\pi)^{3}\delta(\textbf{p}+\textbf{p}')$, so $\textbf{p}= - \textbf{p}'$. This allows us to write:
  \begin{equation*}
    \begin{split}
    H  & = \frac{1}{4}\int\frac{d^3p}{(2\pi)^3}\omega_{\textbf{p}}\left[-\left(a_{\textbf{p}}-a^{\dagger}_{\textbf{-p}}\right)\left(a_{\textbf{-p}}-a^{\dagger}_{\textbf{p}}\right) + \left(a_{\textbf{p}}+a^{\dagger}_{\textbf{-p}}\right)\left(a_{\textbf{-p}}+a^{\dagger}_{\textbf{p}}\right) \right]  \\ 
    & = \frac{1}{2}\int\frac{d^3p}{(2\pi)^3}\omega_{\textbf{p}}\left[a_{\textbf{p}}a^{\dagger}_{\textbf{p}}+a^{\dagger}_{\textbf{-p}}a_{-\textbf{p}}\right]
 \end{split}
  \end{equation*}
  We can then for this second term do the same trick of changing $\textbf{p} \rightarrow -\textbf{p}$, which does not affect the integral's value as $\omega_{\textbf{p}} = \omega_{-\textbf{p}}$. Then also using the fact that $a^{\dagger}_{\textbf{p}}a_{\textbf{p}} = a_{\textbf{p}}a^{\dagger}_{\textbf{p}}-[a_{\textbf{p}},a^{\dagger}_{\textbf{p}}]$ we get the final result that:

  \begin{flalign}
  \label{Ham_SH}
  H = \int \frac{d^3p}{(2\pi)^3}\omega_{\textbf{p}}\left[a_{\textbf{p}}a^{\dagger}_{\textbf{p}}-\frac{1}{2}[a_{\textbf{p}},a^{\dagger}_{\textbf{p}}]\right]
  \end{flalign}
  If we stop to think about the second term in this expression we realize something strange, $[a_{\textbf{p}},a^{\dagger}_{\textbf{p}}] = (2\pi)^3\delta(0)$, which is and infinite constant (constant as in it wont affect any state we act $H$ on). But since this is constant we can never measure it experimentally as we only ever measure the energy shift between two different energies. We can also think of this as the zero point energy for each point in space, except that we made this space continuous resulting in infinite many oscillator ground states, so what did we expect really? This means we can ignore the second term in \ref{Ham_SH}.

   To make sure everything is consistent we also check the commutators, $[H,a_{\textbf{p}}^{\dagger}] = \omega a^{\dagger}_{\textbf{p}}$ , $[H,a_{\textbf{p}}] = -\omega a_{\textbf{p}}$:
   \begin{equation*}
      \begin{split}
        [H,a_{\textbf{p}}] & = \int \frac{d^3p'}{(2\pi)^3}\omega_{\textbf{p}'}\left[a_{\textbf{p}'}a^{\dagger}_{\textbf{p}'}a_{\textbf{p}}-a_{\textbf{p}}a^{\dagger}_{\textbf{p}'}a_{\textbf{p}'}\right] \\
        & = \int \frac{d^3p'}{(2\pi)^3}\omega_{\textbf{p}'}\left[a_{\textbf{p}'}a_{\textbf{p}}a^{\dagger}_{\textbf{p}'}-a_{\textbf{p}'}[a_{\textbf{p}},a^{\dagger}_{\textbf{p}'}]-a_{\textbf{p}}a_{\textbf{p}'}a^{\dagger}_{\textbf{p}'}\right] \\
        & = -\omega_{\textbf{p}}a_{\textbf{p}} 
      \end{split}
   \end{equation*}
   Here we have used the fact that $[a_{\textbf{p}},a_{\textbf{p}'}]= 0 = [a^{\dagger}_{\textbf{p}},a^{\dagger}_{\textbf{p}'}]$ to make the first and last term cancel. The commutator $[H,a^{\dagger}_{\textbf{p}}]=\omega_{\textbf{p}}a^{\dagger}_{\textbf{p}}$, can be calculated in the same manner. We then would like to write the states. 

   \item We can now start to think about the creation and annihilation operators as their name suggests. The ground state is $\ket{0}$ st. $a_{\textbf{p}}\ket{0} = 0$ and has $E=0$, where as acting with $a^{\dagger}_{\textbf{p}}$ increases the energy. These operators clearly depend on momentum $\textbf{p}$ so we can think of them as creating (and annihilating) momentum eigenstates. 
\end{itemize}

\subsection{Particle creation}
\begin{itemize}
\item The next step is to see if we can write out all the possible momentum states as powers of the creation operator $a^{\dagger}_{\textbf{p}}$. The question is then what is the choice of proportionality? We should of course have that the states are normalized, that is $\braket{\textbf{p}|\textbf{p}'} = \delta(\textbf{p}-\textbf{p}')$. This however, leads to a problem, namely that this quantity $\delta(\textbf{p}-\textbf{p}')$ is not Lorentz invariant. To see why, consider a boost along the $x_3$ direction. Using the property that $\delta(f(x)-f(x_0)) = \delta(x-x_0)/|f'(x_0)|$ we have that if we boost from momenta $\textbf{p},\textbf{p}' \rightarrow \tilde{\textbf{p}},\tilde{\textbf{p}}'$,so that $\tilde{p}_3 = \gamma(p_3+\beta E)$ and $\tilde{E} = \gamma(E + \beta p_3)$. Then we have that: 
\begin{equation*}
  \begin{split}
   &  \delta(\textbf{p}-\textbf{p}') = \delta(\tilde{\textbf{p}}-\tilde{\textbf{p}}') \frac{d \tilde{p}_3}{d p_3}  = \delta(\tilde{\textbf{p}}-\tilde{\textbf{p}}') \frac{d \gamma(p_3+\beta E)}{dp_3} \\
    &  =   \delta(\tilde{\textbf{p}}-\tilde{\textbf{p}}')\gamma \left(1 + \frac{d E}{dp_3}\right)  =  \delta(\tilde{\textbf{p}}-\tilde{\textbf{p}}')\frac{\gamma}{E} \left(E + \beta p_3\right) \\
    & =  \delta(\tilde{\textbf{p}}-\tilde{\textbf{p}}')\frac{\tilde{E}}{E} 
  \end{split}
\end{equation*}
Where we have used $E^2 = |\textbf{p}|^2+m^2$ to do the second last step. We can see from this that if we choose the normalization $\ket{\textbf{p}} = \sqrt{2E_{\textbf{p}}}(a^{\dagger}_{\textbf{p}}\ket{0}$ then $\braket{\textbf{p}|\textbf{p}'}$ is Lorentz invariant, as needed. 

\item With this we can consider the action of $\phi$ on the ground state $\ket{0}$. Via \ref{phi_2} this becomes: 

\[
  \phi(\textbf{x})\ket{0} = \int \frac{d^3p}{(2\pi)^3}\frac{1}{2E_{\textbf{p}}}e^{-i\textbf{p}\cdot \textbf{x}}\ket{\textbf{p}} 
\]
Where from here and now on we replace $\omega_{\textbf{p}}$ with $E_{\textbf{p}}$. This is similar to the Fourier expansion we have of $\ket{\textbf{x}}$ in regular QM, except for the factor of $1/E_{\textbf{p}}$, which is almost constant in the non-relativistic, so we can put forward the interpretation that the operator $\phi(\textbf{x})$ acting on the vacuum $\ket{0}$, creates a particle at position $\ket{\textbf{x}}$. 
\end{itemize}
\subsection{Time evolution}
\begin{itemize}
  \item We would now like to turn our heads to adding time evolution to our fields. So far we have been working in the Schr\"odinger picture and interpreted the resulting theory in terms of particles. We now switch to the Heisenberg picture where it is simpler to add time dependence. 

  We know the time evolution operator is a unitary operator $U(t)$, st. $\phi(x) = \phi(\textbf{x},t) = U^{\dagger}\phi(\textbf{x})U$
\end{itemize}


\end{document}
