\documentclass[11pt]{article}

\usepackage[letterpaper,top=2cm,bottom=2cm,left=2cm,right=2cm,marginparwidth=1.75cm]{geometry}
\usepackage{hyperref}
\usepackage{biblatex}
\addbibresource{Bib.bib}
\usepackage{mathtools}
\DeclarePairedDelimiterXPP\BigOSI[2]%
  {\mathcal{O}}{(}{)}{}%
  {\SI{#1}{#2}}
\usepackage{xcolor}
\usepackage{empheq}
\usepackage[most]{tcolorbox}
\usepackage{amsmath}
\usepackage{amssymb}
\usepackage{mathrsfs}
\usepackage[utf8]{inputenc}
\usepackage{graphicx}
\usepackage{float}
\usepackage{parskip}
\usepackage{comment}
\usepackage{mhchem}
 \usepackage{tabularx}
 \usepackage{titling}
 \usepackage{amsmath,environ}
 \usepackage[explicit]{titlesec}
\usepackage{fancyhdr}
\setlength{\droptitle}{3em} 

\title{Diferential Geometry}
\author{Thomas Brosnan}
\date{Notes taken in Professor Sergey Frolov's class, Michaelmas Term 2024}


\newtcbox{\mymath}[1][]{%
    nobeforeafter, math upper, tcbox raise base,
    enhanced, colframe=blue!30!black,
    colback=blue!30, boxrule=1pt,
    #1}
\tcbset{highlight math style={boxsep=2mm,,colback=blue!0!green!0!red!0!}}

\newenvironment{bux}{\empheq[box=\tcbhighmath]{align}}{\endempheq}
\newenvironment{bux*}{\empheq[box=\tcbhighmath]{align*}}{\endempheq}
\renewenvironment{flalign}{\empheq[box=\tcbhighmath]{align}}{\endempheq}
\newcommand{\hsp}{\hspace{8pt}}

\newcommand*{\sectionFont}{%
  \LARGE\bfseries
}

\newenvironment{eq}{\begin{equation}}{\end{eqfruation}}
    
\numberwithin{equation}{section}

\DeclareRobustCommand{\RR}{\mathbb{R}}

\makeatletter
\let\Title\@title % Copy the title to a new command
\makeatother

%change this RGB value to change the section background colour 
\definecolor{mycolor1}{RGB}{164, 196, 34}
\colorlet{SectionColour}{mycolor1}
%subsection background colour 
\definecolor{mycolor2}{gray}{0.8}
\colorlet{subSectionColour}{mycolor2}
%subsubsection background colour 
\definecolor{mycolor3}{RGB}{255,255,255}
\colorlet{subsubSectionColour}{mycolor3}


\begin{document}

\maketitle

\newpage
\topskip0pt
\vspace*{\fill}
\begin{center}
\Large ``hokay''
          -Sergey Frolov
\end{center}
\vspace*{\fill}
\newpage 
\tableofcontents
% For \section
 \titleformat{\section}[block]{\sectionFont}{}{0pt}{%
 \fcolorbox{black}{SectionColour}{\noindent\begin{minipage}{\dimexpr\textwidth-2\fboxsep-2\fboxrule\relax}\thesection  \hsp #1 {\strut} \end{minipage}}}
% For \subsection
 \titleformat{\subsection}[block]{\bfseries}{}{0pt}{%
 \fcolorbox{black}{subSectionColour}{\noindent\begin{minipage}{\dimexpr\textwidth-2\fboxsep-2\fboxrule\relax}\thesubsection  \hsp #1 {\strut} \end{minipage}}}
% For \section*
 \titleformat{name=\section, numberless}[block]{\sectionFont}{}{0pt}{%
 \fcolorbox{black}{SectionColour}{\noindent\begin{minipage}{\dimexpr\textwidth-2\fboxsep-2\fboxrule\relax} #1 {\strut} \end{minipage}}}
  % For \subsection*
 \titleformat{name=\subsection, numberless}[block]{\bfseries}{}{0pt}{%
 \fcolorbox{black}{subSectionColour}{\noindent\begin{minipage}{\dimexpr\textwidth-2\fboxsep-2\fboxrule\relax} #1 {\strut} \end{minipage}}}
 % For \subsubsection
 \titleformat{\subsubsection}[block]{\bfseries}{}{0pt}{%
 \fcolorbox{black}{subsubSectionColour}{\noindent\begin{minipage}{15cm}\thesubsubsection \hsp #1 {\strut} \end{minipage}}}
  % For \subsubsection*
 \titleformat{name=\subsubsection, numberless}[block]{\bfseries}{}{0pt}{%
 \fcolorbox{black}{subsubSectionColour}{\noindent\begin{minipage}{15cm} #1 {\strut} \end{minipage}}}
\newpage 
%header 
\pagestyle{fancy}
\fancyhf{} % Clear all header and footer fields
\fancyhead[L]{\Title}
\fancyhead[R]{\nouppercase{\leftmark}}
\fancyfoot[C]{-~\thepage~-}
\renewcommand{\headrulewidth}{1pt}





%starting document 
\normalsize
\newpage
\section{Definition of a Manifold}

\subsection{Regions}
\begin{itemize}
  \item A \emph{region}  (``open set'') is a set of $D$ points in $\RR^{n}$ such that together with each point $p_0$, D also contains all points sufficiently closer to $p_0$, i.e.: 
  \begin{align*}
  & \forall ~p_0  = (x_0^1,\ldots,x_0^n)\in D ~\exists ~\epsilon> 0, \\ 
   st: &p=(x^1,\ldots,x^n) \in D, ~\text{iff}~ |x^i-x^i_0| < \epsilon . 
  \end{align*}
  \item A \emph{region with out a boundary} is obtained fro ma region $D$ by adjoining all boundary points to D. The \emph{boundary} of a region is the set of all boundary points.  
\end{itemize}

\subsection{Differentiable Manifold}
\begin{itemize}
  \item A differentiable $n$-dimensional manifold is a set $M$ together with the following structure on it. The set $M$ is the union of a finite or countably infinite collection of subsets $U_{q}$ with the following properties: 
  \begin{itemize}
     \item  Each subset $U_{q}$ has defined on it co-ords $x^{\alpha}_{q}, \alpha = 1,\ldots,n$ called local co-ords by virtue of which $U_{q}$ is identifiable with a  region of Euclidean 
   \end{itemize} 
\end{itemize}
\end{document}
