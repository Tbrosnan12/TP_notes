\documentclass[11pt]{article}

\usepackage[letterpaper,top=2cm,bottom=2cm,left=2cm,right=2cm,marginparwidth=1.75cm]{geometry}
\usepackage{hyperref}
\usepackage{biblatex}
\addbibresource{Bib.bib}
\usepackage{mathtools}
\DeclarePairedDelimiterXPP\BigOSI[2]%
  {\mathcal{O}}{(}{)}{}%
  {\SI{#1}{#2}}
\usepackage{xcolor}
\usepackage{empheq}
\usepackage[most]{tcolorbox}
\usepackage{amsmath}
\usepackage{amssymb}
\usepackage{mathrsfs}
\usepackage[utf8]{inputenc}
\usepackage{graphicx}
\usepackage{float}
\usepackage{parskip}
\usepackage{comment}
%\usepackage{mhchem}
 \usepackage{tabularx}
 \usepackage{titling}
 \usepackage{amsmath,environ}
 \usepackage[explicit]{titlesec}
\usepackage{fancyhdr}
\setlength{\droptitle}{3em} 

\title{Differential Geometry}
\author{Thomas Brosnan}
\date{Notes taken in Professor Sergey Frolov's class, Michaelmas Term 2024}


\newtcbox{\mymath}[1][]{%
    nobeforeafter, math upper, tcbox raise base,
    enhanced, colframe=blue!30!black,
    colback=blue!30, boxrule=1pt,
    #1}
\tcbset{highlight math style={boxsep=2mm,,colback=blue!0!green!0!red!0!}}

\newenvironment{bux}{\empheq[box=\tcbhighmath]{align}}{\endempheq}
\newenvironment{bux*}{\empheq[box=\tcbhighmath]{align*}}{\endempheq}
\renewenvironment{flalign}{\vspace{-3mm}\empheq[box=\tcbhighmath]{align}}{\endempheq}
\renewenvironment{flalign*}{\vspace{-3mm}\empheq[box=\tcbhighmath]{align*}}{\endempheq}
%\renewenvironment{align}{\vspace{-5mm}\begin{align}}{\end{align}}
%\renewenvironment{align*}{\vspace{-5mm}\begin{align*}}{\end{align*}}
\renewenvironment{alignat}{\empheq{align*}}{\endempheq}
\newcommand{\hsp}{\hspace{8pt}}

\newcommand*{\sectionFont}{%
  \LARGE\bfseries
}

\newenvironment{eq}{\begin{equation}}{\end{eqfruation}}
    
\numberwithin{equation}{section}

\DeclareRobustCommand{\RR}{\mathbb{R}}

\makeatletter
\let\Title\@title % Copy the title to a new command
\makeatother

%change this RGB value to change the section background colour 
\definecolor{mycolor1}{RGB}{164, 196, 34}
\colorlet{SectionColour}{mycolor1}
%subsection background colour 
\definecolor{mycolor2}{gray}{0.8}
\colorlet{subSectionColour}{mycolor2}
%subsubsection background colour 
\definecolor{mycolor3}{RGB}{255,255,255}
\colorlet{subsubSectionColour}{mycolor3}


\begin{document}

\maketitle

\newpage
\topskip0pt
\vspace*{\fill}
\begin{center}
\Large ``hokay''
          -Sergey Frolov
\end{center}
\vspace*{\fill}
\newpage 
\tableofcontents
% For \section
 \titleformat{\section}[block]{\sectionFont}{}{0pt}{%
 \fcolorbox{black}{SectionColour}{\noindent\begin{minipage}{\dimexpr\textwidth-2\fboxsep-2\fboxrule\relax}\thesection  \hsp #1 {\strut} \end{minipage}}}
% For \subsection
 \titleformat{\subsection}[block]{\bfseries}{}{0pt}{%
 \fcolorbox{black}{subSectionColour}{\noindent\begin{minipage}{\dimexpr\textwidth-2\fboxsep-2\fboxrule\relax}\thesubsection  \hsp #1 {\strut} \end{minipage}}}
% For \section*
 \titleformat{name=\section, numberless}[block]{\sectionFont}{}{0pt}{%
 \fcolorbox{black}{SectionColour}{\noindent\begin{minipage}{\dimexpr\textwidth-2\fboxsep-2\fboxrule\relax} #1 {\strut} \end{minipage}}}
  % For \subsection*
 \titleformat{name=\subsection, numberless}[block]{\bfseries}{}{0pt}{%
 \fcolorbox{black}{subSectionColour}{\noindent\begin{minipage}{\dimexpr\textwidth-2\fboxsep-2\fboxrule\relax} #1 {\strut} \end{minipage}}}
 % For \subsubsection
 \titleformat{\subsubsection}[block]{\bfseries}{}{0pt}{%
 \fcolorbox{black}{subsubSectionColour}{\noindent\begin{minipage}{15cm}\thesubsubsection \hsp #1 {\strut} \end{minipage}}}
  % For \subsubsection*
 \titleformat{name=\subsubsection, numberless}[block]{\bfseries}{}{0pt}{%
 \fcolorbox{black}{subsubSectionColour}{\noindent\begin{minipage}{15cm} #1 {\strut} \end{minipage}}}
\newpage 
%header 
\pagestyle{fancy}
\fancyhf{} % Clear all header and footer fields
\fancyhead[L]{\Title}
\fancyhead[R]{\nouppercase{\leftmark}}
\fancyfoot[C]{-~\thepage~-}
\renewcommand{\headrulewidth}{1pt}





%starting document 
\normalsize
\newpage
\section{Definition of a Manifold}

\subsection{Regions}
\begin{itemize}
  \item A \emph{region}  (``open set'') is a set of $D$ points in $\RR^{n}$ such that together with each point $p_0$, D also contains all points sufficiently closer to $p_0$, i.e.: 
  \begin{align*}
  & \forall ~p_0  = (x_0^1,\ldots,x_0^n)\in D ~\exists ~\epsilon> 0, \\ 
   st: &p=(x^1,\ldots,x^n) \in D, ~\text{iff}~ |x^i-x^i_0| < \epsilon . 
  \end{align*}
  \item A \emph{region with out a boundary} is obtained fro ma region $D$ by adjoining all boundary points to D. The \emph{boundary} of a region is the set of all boundary points.  
\end{itemize}

\subsection{Differentiable Manifold}
\begin{itemize}
  \item A differentiable $n$-dimensional manifold is a set $M$ together with the following structure on it. The set $M$ is the union of a finite or countably infinite collection of subsets $U_{q}$ with the following properties: 
  \begin{itemize}
     \item  Each subset $U_{q}$ has defined on it co-ords $x_{q}^{\alpha}, \alpha = 1,\ldots,n$ called local co-ords by virtue of which $U_{q}$ is identifiable with a region of Euclidean $n$-space $\RR^n$ with Euclidean co-ords $x^{\alpha}_q$. The $U_{q}$ with their co-ord systems are called \emph{charts} or \emph{local coordinate neighborhoods}. 

\item Each non-empty intersection $U_{q}\cap U_{p}$ of a pair of charts thus has defined on it two co-ord systems, the restriction of $x^{\alpha}_{p}$ and $x^{\alpha}_{q}$. It is required that under each of these coordinatizations the intersection $U_{q}\cap U_{p}$ is identifiable with a region of $\RR^n$ and that each of these co-ordinate systems be expressible in terms of the other in a one to one differentiable manner. Thus, if a the \emph{transition} functions from $x^{\alpha}_{p}$ to $x^{\alpha}_{q}$ and back are given by:
 \begin{align*}
 & x^{\alpha}_{p} = x^{\alpha}_{p}( x^{1}_{q},\ldots, x^{n}_{q}),~~\alpha = 1,\ldots,n \\
 & x^{\alpha}_{q} = x^{\alpha}_{q}( x^{1}_{p},\ldots, x^{n}_{p}),~~\alpha = 1,\ldots,n
 \end{align*}
 Then the \emph{Jacobian} $J_{pq} = det(\partial x^{\alpha}_{p}/\partial x^{\alpha}_{q})$ is non-zero on $U_{p} \cap U_{q}$. 
  \end{itemize} 
\end{itemize}

\subsection{Abuse of notation}
\begin{itemize}
  \item Regular partial derivative do not have the same ``canceling'' that total derivative have ($dx * dy/dx = dy$) But we can restore this property through Einstein summation convention. That is that:
  \begin{flalign*}
  \sum_{\gamma=1}^{n}\frac{\partial x^{\alpha}_{p}}{\partial x^{\gamma}_{q}}\frac{\partial x^{\gamma}_{q}}{\partial x^{\beta}_{q}} = \frac{\partial x^{\alpha}_{p}}{\partial x^{\gamma}_{q}}\frac{\partial x^{\gamma}_{q}}{\partial x^{\beta}_{q}}  = \delta^{\alpha}_{\beta}
  \end{flalign*}

\end{itemize}

\newpage 
\section{Elements of Topology}
\subsection{Topological space}
\begin{itemize}
  \item A topological space is a set $X$ of points of which certain subsets called \emph{open sets} of the topological space, are distinguished, these open sets have to satisfy: 
  \begin{itemize}
    \item The intersection of any two (and hence of any finite collection) open sets should again be an open set.
    \item The union of any collection of open sets must again be open. 
    \item The empty set and the whole set $X$ must be open.
  \end{itemize}

  \item The compliment of any open  set is called a \emph{closed} set of the topological space. 

  In Euclidean space $\RR^n$ the ``Euclidean topology'' is the usual one where the open sets are the open regions. 
\end{itemize}
  \subsubsection{Induced topology} 
  \begin{itemize}
    \item Given any subset $A \in \RR^n$, the \emph{induced topology} on $A$ is that where the open sets are the intersections $A \cap U$, where $U$ ranges over all open sets of $\RR^{n}$. 
  \end{itemize}
\subsubsection{Continuity}
\begin{itemize}
  \item A map $f: X \rightarrow Y$ of one topological space to another is called \emph{continuous} if the complete inverse image $f^{-1}(U)$ of every open set $U \subset Y$ is open in X. 
\end{itemize}

\subsubsection{Homeomorphic}
\begin{itemize}
  \item Two topological space are \emph{topologically equivalent} or \emph{homeomorphic} if there is a one to one and onto map (bijective) between them, such that it and its inverse are continuous. 
\end{itemize}
\subsubsection{Topology on a manifold}
\begin{itemize}
  \item The topology on a manifold $M$ is given by the following specifications of the open sets. In every local co-ordinate neighborhood $U_q$ the open regions are to be open in the topology on M; the totality of open sets of M is then obtained by admitting as open, also arbitrary unions countable collections of such regions, i.e. by closing under countable unions. 
\end{itemize}
\subsection{Metric space}
\begin{itemize}
  \item A \emph{metric space} is a set which comes equipped with a ``distance function'' i.e. a real-valued function $\rho(x,y)$, defined on pairs $x, y$ of its elements and having the following properties:
  \begin{itemize}
    \item Symmetry: $\rho(x,y) = \rho(y,x)$.
    \item Positivity: $\rho(x,x) =0, ~~~ \rho(x,y) > 0 ~~\text{if}~~ x \neq y$. 
    \item The triangle inequality: $\rho(x,y) \leq \rho(x,z)+\rho(z,y)$.
  \end{itemize}
\end{itemize}
\subsubsection{Hausdorff}
\begin{itemize}
  \item A topological space is called \emph{Hausdorff} if any two points are contained in disjoint open sets. Any metric space is Hausdorff because the open balls of radius $\rho(x,y)/3$ with centers at $c,y$ do not intersect.   

  All topological spaces we consider will be Hausdorff. 
\end{itemize}

\subsubsection{Compact}
\begin{itemize}
  \item A topological space $X$ is said to be compact if every countable collection of open sets covering $X$ contains a finite sub-collection already covering $X$.

  If $X$ is a metric space the compactness is equivalent to the condition that from every sequence of points of $X$ a convergent sub-sequence can be selected.  
\end{itemize}
\subsubsection{Connected}
\begin{itemize}
  \item A topological space is connected if any two points can be joined by a continuous path. 
\end{itemize}

\subsection{Orientation}
\begin{itemize}
  \item A manifold $M$ is said to be \emph{orientated} of one can choose its atlas (collection of all the charts) so that for every pair $U_{p},U_{q}$ of intersecting co-ordinate neighborhoods the Jacobian of the transition functions is positive.  

  \item We say that the co-ordinate systems $x$ and $y$ define the \emph{same orientation} if $J>0$ and the \emph{opposite orientation} if $J<0$. 
\end{itemize}

\section{Mappings and Tensors on Manifolds}
\subsubsection{Manifold mappings}
\begin{itemize}
  \item A mapping $f: M \rightarrow N$ is said to be smooth of smoothness class $k$ if for all $p,q$ for which $f$ determines functions $y^{b}_{q}(x^{1}_{p},\ldots,x^{m}_{p}) = f(x^{1}_{p},..,x^m_p)^{b}_{p}$, these functions are, where defined, smooth of smoothness calss $k$ (i.e. all theire partial derivatives up to those of $k$-th order exist and are continuous).

  the smoothness class of $f$ cannot exceed the maximum class of the manifolds.  
\end{itemize}

\subsection{equivilent manifolds}
\begin{itemize}
  \item The manifolds $M$ and $N$ are said to be \emph{smoothly equivilent} or \emph{diffeomorphic} if there is a one to one and onto map $f$ such that both $f:M \rightarrow N$ and $f^{-1}:N \rightarrow M$ are smooth of some class $k\geq 1$. 

  Since $f^{-1}$ exits then the jacobian $J_{pq} \neq 0 $ wherever it is defined. 
\end{itemize}

\subsection{Tangent vector}
\begin{itemize}
  \item A \emph{tangent} vector to an $m$-dim manifold $M$ at an arbritrary point $x$ is represented in terms of local co-ords $x^{\alpha}_-p$ by an $m$ tuple $\xi^{\alpha}$ of components which are linked to the components in terms of any oher system $x^{\beta}_{q}$ of local co-ords by:
  \begin{align}
  \label{tangent}
    \xi^{\alpha}_{p} = \left(\frac{\partial x^{\alpha}_{p}}{\partial x^{\beta}_{q}}\right)_{x} \xi^{\beta}_{q},~~~\forall~ \alpha
  \end{align}

  \item The set of all tangent vectors to an $m$-dim manifold $M$ at a point $x$ forms an $m$-dm vector space $T_{x} = T_{x}M$, the \emph{tangent space} to $M$ at the point $x$. 

   \item Thus, the velocity at $x$ of any smooth curve $M$ through $x$ is a tangent vector to $M$ at $x$. 

    \item From this definition \ref{tangent} one sees that for any choice of local co-ords $x^{\alpha}$ in a neighbourhood of x
\end{itemize}
\end{document}
