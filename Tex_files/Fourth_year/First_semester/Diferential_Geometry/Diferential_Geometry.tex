\documentclass[11pt]{article}

\usepackage[letterpaper,top=2cm,bottom=2cm,left=2cm,right=2cm,marginparwidth=1.75cm]{geometry}
\usepackage{hyperref}
\usepackage{biblatex}
\addbibresource{Bib.bib}
\usepackage{mathtools}
\DeclarePairedDelimiterXPP\BigOSI[2]%
  {\mathcal{O}}{(}{)}{}%
  {\SI{#1}{#2}}
\usepackage{xcolor}
\usepackage{empheq}
\usepackage[most]{tcolorbox}
\usepackage{amsmath}
\usepackage{amssymb}
\usepackage{mathrsfs}
\usepackage[utf8]{inputenc}
\usepackage{graphicx}
\usepackage{float}
\usepackage{parskip}
\usepackage{comment}
%\usepackage{mhchem}
 \usepackage{tabularx}
 \usepackage{titling}
 \usepackage{amsmath,environ}
 \usepackage[explicit]{titlesec}
\usepackage{fancyhdr}
\usepackage{braket}
\setlength{\droptitle}{3em} 

\title{Differential Geometry}
\author{Thomas Brosnan}
\date{Notes taken in Professor Sergey Frolov's class, Michaelmas Term 2024}


\newtcbox{\mymath}[1][]{%
    nobeforeafter, math upper, tcbox raise base,
    enhanced, colframe=blue!30!black,
    colback=blue!30, boxrule=1pt,
    #1}
\tcbset{highlight math style={boxsep=2mm,,colback=blue!0!green!0!red!0!}}

\newenvironment{bux}{\empheq[box=\tcbhighmath]{align}}{\endempheq}
\newenvironment{bux*}{\empheq[box=\tcbhighmath]{align*}}{\endempheq}
\renewenvironment{flalign}{\vspace{-3mm}\empheq[box=\tcbhighmath]{align}}{\endempheq}
\renewenvironment{flalign*}{\vspace{-3mm}\empheq[box=\tcbhighmath]{align*}}{\endempheq}
%\renewenvironment{align}{\vspace{-5mm}\begin{align}}{\end{align}}
%\renewenvironment{align*}{\vspace{-5mm}\begin{align*}}{\end{align*}}
\renewenvironment{alignat}{\empheq{align*}}{\endempheq}
\newcommand{\hsp}{\hspace{8pt}}

\newcommand*{\sectionFont}{%
  \LARGE\bfseries
}

\newenvironment{eq}{\begin{equation}}{\end{eqfruation}}
    
\numberwithin{equation}{section}

\DeclareRobustCommand{\RR}{\mathbb{R}}

\makeatletter
\let\Title\@title % Copy the title to a new command
\makeatother

%change this RGB value to change the section background colour 
\definecolor{mycolor1}{RGB}{164, 196, 34}
\colorlet{SectionColour}{mycolor1}
%subsection background colour 
\definecolor{mycolor2}{gray}{0.8}
\colorlet{subSectionColour}{mycolor2}
%subsubsection background colour 
\definecolor{mycolor3}{RGB}{255,255,255}
\colorlet{subsubSectionColour}{mycolor3}


\begin{document}

\maketitle

\newpage
\topskip0pt
\vspace*{\fill}
\begin{center}
\Large ``hokay''
          -Sergey Frolov
\end{center}
\vspace*{\fill}
\newpage 
\tableofcontents
% For \section
 \titleformat{\section}[block]{\sectionFont}{}{0pt}{%
 \fcolorbox{black}{SectionColour}{\noindent\begin{minipage}{\dimexpr\textwidth-2\fboxsep-2\fboxrule\relax}\thesection  \hsp #1 {\strut} \end{minipage}}}
% For \subsection
 \titleformat{\subsection}[block]{\bfseries}{}{0pt}{%
 \fcolorbox{black}{subSectionColour}{\noindent\begin{minipage}{\dimexpr\textwidth-2\fboxsep-2\fboxrule\relax}\thesubsection  \hsp #1 {\strut} \end{minipage}}}
% For \section*
 \titleformat{name=\section, numberless}[block]{\sectionFont}{}{0pt}{%
 \fcolorbox{black}{SectionColour}{\noindent\begin{minipage}{\dimexpr\textwidth-2\fboxsep-2\fboxrule\relax} #1 {\strut} \end{minipage}}}
  % For \subsection*
 \titleformat{name=\subsection, numberless}[block]{\bfseries}{}{0pt}{%
 \fcolorbox{black}{subSectionColour}{\noindent\begin{minipage}{\dimexpr\textwidth-2\fboxsep-2\fboxrule\relax} #1 {\strut} \end{minipage}}}
 % For \subsubsection
 \titleformat{\subsubsection}[block]{\bfseries}{}{0pt}{%
 \fcolorbox{black}{subsubSectionColour}{\noindent\begin{minipage}{15cm}\thesubsubsection \hsp #1 {\strut} \end{minipage}}}
  % For \subsubsection*
 \titleformat{name=\subsubsection, numberless}[block]{\bfseries}{}{0pt}{%
 \fcolorbox{black}{subsubSectionColour}{\noindent\begin{minipage}{15cm} #1 {\strut} \end{minipage}}}
\newpage 
%header 
\pagestyle{fancy}
\fancyhf{} % Clear all header and footer fields
\fancyhead[L]{\Title}
\fancyhead[R]{\nouppercase{\leftmark}}
\fancyfoot[C]{-~\thepage~-}
\renewcommand{\headrulewidth}{1pt}



%starting document 
\normalsize
\newpage
\section{Definition of a Manifold}

\subsection{Regions}
\begin{itemize}
  \item A \emph{region}  (``open set'') is a set of $D$ points in $\RR^{n}$ such that together with each point $p_0$, D also contains all points sufficiently closer to $p_0$, i.e.: 
  \begin{align*}
  & \forall ~p_0  = (x_0^1,\ldots,x_0^n)\in D ~\exists ~\epsilon> 0, \\ 
   st: &p=(x^1,\ldots,x^n) \in D, ~\text{iff}~ |x^i-x^i_0| < \epsilon . 
  \end{align*}
  \item A \emph{region with out a boundary} is obtained fro ma region $D$ by adjoining all boundary points to D. The \emph{boundary} of a region is the set of all boundary points.  
\end{itemize}

\subsection{Differentiable Manifold}
\begin{itemize}
  \item A differentiable $n$-dimensional manifold is a set $M$ together with the following structure on it. The set $M$ is the union of a finite or countably infinite collection of subsets $U_{q}$ with the following properties: 
  \begin{itemize}
     \item  Each subset $U_{q}$ has defined on it co-ords $x_{q}^{\alpha}, \alpha = 1,\ldots,n$ called local co-ords by virtue of which $U_{q}$ is identifiable with a region of Euclidean $n$-space $\RR^n$ with Euclidean co-ords $x^{\alpha}_q$. The $U_{q}$ with their co-ord systems are called \emph{charts} or \emph{local coordinate neighborhoods}. 

\item Each non-empty intersection $U_{q}\cap U_{p}$ of a pair of charts thus has defined on it two co-ord systems, the restriction of $x^{\alpha}_{p}$ and $x^{\alpha}_{q}$. It is required that under each of these coordinatizations the intersection $U_{q}\cap U_{p}$ is identifiable with a region of $\RR^n$ and that each of these co-ordinate systems be expressible in terms of the other in a one to one differentiable manner. Thus, if a the \emph{transition} functions from $x^{\alpha}_{p}$ to $x^{\alpha}_{q}$ and back are given by:
 \begin{align*}
 & x^{\alpha}_{p} = x^{\alpha}_{p}( x^{1}_{q},\ldots, x^{n}_{q}),~~\alpha = 1,\ldots,n \\
 & x^{\alpha}_{q} = x^{\alpha}_{q}( x^{1}_{p},\ldots, x^{n}_{p}),~~\alpha = 1,\ldots,n
 \end{align*}
 Then the \emph{Jacobian} $J_{pq} = det(\partial x^{\alpha}_{p}/\partial x^{\alpha}_{q})$ is non-zero on $U_{p} \cap U_{q}$. 
  \end{itemize} 
\end{itemize}

\subsection{Abuse of notation}
\begin{itemize}
  \item Regular partial derivative do not have the same ``canceling'' that total derivative have ($dx * dy/dx = dy$) But we can restore this property through Einstein summation convention. That is that:
  \begin{flalign*}
  \sum_{\gamma=1}^{n}\frac{\partial x^{\alpha}_{p}}{\partial x^{\gamma}_{q}}\frac{\partial x^{\gamma}_{q}}{\partial x^{\beta}_{q}} = \frac{\partial x^{\alpha}_{p}}{\partial x^{\gamma}_{q}}\frac{\partial x^{\gamma}_{q}}{\partial x^{\beta}_{q}}  = \delta^{\alpha}_{\beta}
  \end{flalign*}

\end{itemize}

\newpage 
\section{Elements of Topology}
\subsection{Topological space}
\begin{itemize}
  \item A topological space is a set $X$ of points of which certain subsets called \emph{open sets} of the topological space, are distinguished, these open sets have to satisfy: 
  \begin{itemize}
    \item The intersection of any two (and hence of any finite collection) open sets should again be an open set.
    \item The union of any collection of open sets must again be open. 
    \item The empty set and the whole set $X$ must be open.
  \end{itemize}

  \item The compliment of any open  set is called a \emph{closed} set of the topological space. 

  In Euclidean space $\RR^n$ the ``Euclidean topology'' is the usual one where the open sets are the open regions. 
\end{itemize}
  \subsubsection{Induced topology} 
  \begin{itemize}
    \item Given any subset $A \in \RR^n$, the \emph{induced topology} on $A$ is that where the open sets are the intersections $A \cap U$, where $U$ ranges over all open sets of $\RR^{n}$. 
  \end{itemize}
\subsubsection{Continuity}
\begin{itemize}
  \item A map $f: X \rightarrow Y$ of one topological space to another is called \emph{continuous} if the complete inverse image $f^{-1}(U)$ of every open set $U \subset Y$ is open in X. 
\end{itemize}

\subsubsection{Homeomorphic}
\begin{itemize}
  \item Two topological space are \emph{topologically equivalent} or \emph{homeomorphic} if there is a one to one and onto map (bijective) between them, such that it and its inverse are continuous. 
\end{itemize}
\subsubsection{Topology on a manifold}
\begin{itemize}
  \item The topology on a manifold $M$ is given by the following specifications of the open sets. In every local co-ordinate neighborhood $U_q$ the open regions are to be open in the topology on M; the totality of open sets of M is then obtained by admitting as open, also arbitrary unions countable collections of such regions, i.e. by closing under countable unions. 
\end{itemize}
\subsection{Metric space}
\begin{itemize}
  \item A \emph{metric space} is a set which comes equipped with a ``distance function'' i.e. a real-valued function $\rho(x,y)$, defined on pairs $x, y$ of its elements and having the following properties:
  \begin{itemize}
    \item Symmetry: $\rho(x,y) = \rho(y,x)$.
    \item Positivity: $\rho(x,x) =0, ~~~ \rho(x,y) > 0 ~~\text{if}~~ x \neq y$. 
    \item The triangle inequality: $\rho(x,y) \leq \rho(x,z)+\rho(z,y)$.
  \end{itemize}
\end{itemize}
\subsubsection{Hausdorff}
\begin{itemize}
  \item A topological space is called \emph{Hausdorff} if any two points are contained in disjoint open sets. Any metric space is Hausdorff because the open balls of radius $\rho(x,y)/3$ with centers at $c,y$ do not intersect.   

  All topological spaces we consider will be Hausdorff. 
\end{itemize}

\subsubsection{Compact}
\begin{itemize}
  \item A topological space $X$ is said to be compact if every countable collection of open sets covering $X$ contains a finite sub-collection already covering $X$.

  If $X$ is a metric space the compactness is equivalent to the condition that from every sequence of points of $X$ a convergent sub-sequence can be selected.  
\end{itemize}
\subsubsection{Connected}
\begin{itemize}
  \item A topological space is connected if any two points can be joined by a continuous path. 
\end{itemize}

\subsection{Orientation}
\begin{itemize}
  \item A manifold $M$ is said to be \emph{orientated} of one can choose its atlas (collection of all the charts) so that for every pair $U_{p},U_{q}$ of intersecting co-ordinate neighborhoods the Jacobian of the transition functions is positive.  

  \item We say that the co-ordinate systems $x$ and $y$ define the \emph{same orientation} if $J>0$ and the \emph{opposite orientation} if $J<0$. 
\end{itemize}
\newpage
\section{Mappings on Manifolds}
\subsection{Manifold mappings}
\begin{itemize}
  \item A mapping $f: M \rightarrow N$ is said to be smooth of smoothness class $k$ if for all $p,q$ for which $f$ determines functions $y^{b}_{q}(x^{1}_{p},\ldots,x^{m}_{p}) = f(x^{1}_{p},..,x^m_p)^{b}_{p}$, these functions are, where defined, smooth of smoothness class $k$ (i.e. all their partial derivatives up to those of $k$-th order exist and are continuous).

  the smoothness class of $f$ cannot exceed the maximum class of the manifolds.  
\end{itemize}

\subsection{Equivalent manifolds}
\begin{itemize}
  \item The manifolds $M$ and $N$ are said to be \emph{smoothly equivilent} or \emph{diffeomorphic} if there is a one to one and onto map $f$ such that both $f:M \rightarrow N$ and $f^{-1}:N \rightarrow M$ are smooth of some class $k\geq 1$. 

  Since $f^{-1}$ exits then the Jacobian $J_{pq} \neq 0 $ wherever it is defined. 
\end{itemize}

\subsection{Tangent vector}
\begin{itemize}
  \item A \emph{tangent} vector to an $m$-dim manifold $M$ at an arbitrary point $x$ is represented in terms of local co-ords $x^{\alpha}_-p$ by an $m$ tuple $\xi^{\alpha}$ of components which are linked to the components in terms of any other system $x^{\beta}_{q}$ of local co-ords by:
  \begin{align}
  \label{tangent}
    \xi^{\alpha}_{p} = \left(\frac{\partial x^{\alpha}_{p}}{\partial x^{\beta}_{q}}\right)_{x} \xi^{\beta}_{q},~~~\forall~ \alpha
  \end{align}

  \item The set of all tangent vectors to an $m$-dim manifold $M$ at a point $x$ forms an $m$-dm vector space $T_{x} = T_{x}M$, the \emph{tangent space} to $M$ at the point $x$. 

   \item Thus, the velocity at $x$ of any smooth curve $M$ through $x$ is a tangent vector to $M$ at $x$. 
\end{itemize}

\subsection{Push forward}
\begin{itemize}
  \item A smooth map $f$ from $M$ to $N$ gives rise for each $x$ to a \emph{push forward} or an \emph{induced linear} map to tangent spaces:
  \begin{align*}
      f_{\ast}: T_{x}M \rightarrow T_{f(x)}N
    \end{align*}  
    defined as sending the velocity at $x$ of any smooth curve $x =x(\tau)$ on $M$ to the velocity vector at $f(x)$ of the curve $f(x(\tau))$ on $N$. If the map $f$ is given by: $y^b = f^b(x^{1},\ldots,x^m)$ for $x \in M$ and $y \in N$, then the push forward map $f_{\ast}$ is:
    \begin{align*}
      \xi^{\alpha} \rightarrow \eta^{b} = \frac{\partial f^{b}}{\partial x^{\alpha}}\xi^{\alpha}. 
    \end{align*}

    \item For a real valued function $f: M -> \RR$, the push-forward map $f_{\ast}$ corresponding to each $x\in M$ is a real valued linear function on the tangent space to $M$ at $x$: 
    \begin{align*}
      \xi^{a} \rightarrow \eta = \frac{\partial f}{\partial x^{\alpha}}\xi^{\alpha}
    \end{align*}
    and it is represented by the gradiant of $f$ at $x$, and is a co-vector or one form. Thus $f_{\ast}$ can be identified with the differential $df$, in particular:
    \begin{align*}
      dx^{\alpha}_{p}: \xi^{\alpha}\rightarrow\eta =\xi^{\alpha}_{p}
    \end{align*}
\end{itemize}

\subsection{Directional derivative}
\begin{itemize}
  \item We can associate with each vector $\xi = (\xi^{i})$ a linear differential operator as follows: Since the gradient $\frac{\partial f}{\partial x^i}$ of a function $f$ is a co-vector, the quantity:
  \begin{align*}
    \partial_{\xi}f = \xi^{i}\frac{\partial f}{\partial x^{i}}
  \end{align*}
  is a scalar called the directional derivative of $f$ in the direction of $\xi$. 

  \item Thus an arbitrary vector $\xi$ corresponds to the operator:
  \begin{align*}
    \partial_{\xi} = \xi^{i}\frac{\partial }{\partial x^i}
  \end{align*}
  So we can identify $\frac{\partial}{\partial x^i}\equiv e_{i}$ as the \emph{Canonical basis of the tangent space}. 
\end{itemize} 

\subsection{Riemann metric}
\begin{itemize}
  \item A \emph{Riemann metric} on a manifold $M$ is a point-dependent, positive-definite quadratic form on the tangent vectors at each point, depending smoothly on the local co-ords of the points. 

  Thus at each point $x = (x^{1}_{p},\ldots,x^{m}_p)$ of each chart $U_{p}$, the metric is given by a symmetric metric $g_{\alpha\beta}(x^{1}_{p},...,x^{m}_p)$, and determines a symmetric scalar product of pairs of tangent vectors at the point $x$. 
  \begin{align*}
     \braket{\xi, \eta}  = g_{\alpha\beta}^{(p)}\xi^{\alpha}_{p}\eta^{\beta}_{p} = \braket{\eta,\xi}, ~~~ |\xi|^2 = \braket{\xi,\xi}
   \end{align*} 
   This scalar product is to be co-ordinate independent:
   \begin{align*}
     g_{\alpha\beta}^{(p)}\xi^{\alpha}_{p}\eta^{\beta}_{p} = g_{\alpha\beta}^{(q)}\xi^{\alpha}_{q}\eta^{\beta}_{q}
   \end{align*}
   And therefor the coefficients $g_{\alpha\beta}^{(p)}$ of the quadratic form transform as:
   \begin{align}
   \label{g}
     g_{\gamma\delta}^{(q)}  = \frac{\partial x^{\alpha}_{p}}{\partial x^{\gamma}_{q}}\frac{\partial x^{\beta}_p}{\partial x^{\delta}_{q}}g_{\alpha\beta}^{(p)} 
   \end{align}
   For a \emph{pseudo-Riemann} metric $M$ one just requires the quadratic form to be \emph{nondegenerate}. Note that \ref{g} can be re-written as:
   \begin{align*}
     ds^2 = g_{\alpha\beta}^{(p)}dx^{\alpha}_{p}dx^{\beta}_{p} =g_{\alpha\beta}^{(q)}dx^{\alpha}_{q}dx^{\beta}_{q}  
   \end{align*}
   Where $ds$ is called a line element, and it is chart-independent. $ds$ is used to measure the distance between two infinitesimally close points. 
\end{itemize}

\newpage 
\section{Tensors}
\subsection{Tensor def}
\begin{itemize}
  \item A \emph{tensor of type} $(k,l)$ and rank $k+l$ on an $m$-dim manifold $M$ is given each local co-ord system $(x^i_p)$ by a family of functions:
  \begin{align*}
    ^{(p)}T^{i_1,\ldots,i_k}_{j_1,\ldots,j_l}(x) ~~\text{of the point}~x.
  \end{align*}
  In other local co-ord $(x^i_q)$ the components $^{(p)}T^{i_1,\ldots,i_k}_{j_1,\ldots,j_l}(x)$  of the same tensor are:
  \begin{align*}
    ^{(p)}T^{s_1,\ldots,s_k}_{t_1,\ldots,t_l}(x) = \frac{\partial x^{s_1}_q}{\partial x^{i_1}_p}\cdot \cdot \cdot\frac{\partial x^{s_k}_q}{\partial x^{i_k}_p}\frac{\partial x^{j_1}_p}{\partial x^{t_1}_q}  \cdot \cdot \cdot \frac{\partial x^{j_l}_p}{\partial x^{t_l}_q} \cdot ~^{(p)}T^{i_1,\ldots,i_k}_{j_1,\ldots,j_l}(x)
  \end{align*}
\end{itemize}
\subsection{Operations on Tensors}
\subsubsection{Permutation of indices}
\begin{itemize}
  \item Let $\sigma$ be some permutation of $1,2,\ldots,l$. $\sigma$ acrs on the ordered tuple $(j_1,\ldots,j_l)$ as $\sigma(j_1,\ldots,j_l) = (j_{\sigma_1},\ldots,j_{\sigma_l})$. We say that a tensor $\tilde{T}^{i_1,\ldots,i_k}_{j_1,\ldots,j_l}(x)$ =is obtained from a tensor $T^{i_1,\ldots,i_k}_{j_1,\ldots,j_l}(x)$ by means of a permutation $\sigma$ of the lower indices if at each point of $M$:
  \begin{align*}
    \tilde{T}^{i_1,\ldots,i_k}_{j_1,\ldots,j_l}(x) = T^{i_1,\ldots,i_k}_{\sigma(j_1,\ldots,j_l)}(x)
  \end{align*}
  Permutation of upper indicies is defined similarly. 
\end{itemize}
\subsubsection{Contraction of indicies}
\begin{itemize}
  \item By the contraction of a tensor $T^{i_1,\ldots,i_k}_{j_1,\ldots,j_l}(x)$ of type $(k,l)$ with respect to the indcies $i_a,j_a$ we mean the tensor (summation over $n$):
  \begin{align*}
    T^{i_1,\ldots,i_{k-1}}_{j_1,\ldots,j_{l-1}}(x)=T^{i_1,\ldots i_{a-1},n,i_{a+1},\ldots,i_k}_{j_1,\ldots ,j_{a-1},n,j_{a+1},\ldots,j_l}(x)
  \end{align*}
  Of type $(k-1,l-1)$
\end{itemize}
\subsubsection{Product of Tensors}
\begin{itemize}
  \item Given two tensors $T = \left(T^{i_1,\ldots,i_k}_{j_1,\ldots,j_l}\right)$ of type $(k,l)$ and $P = \left(P^{i_1,\ldots,i_p}_{j_1,\ldots,j_q}\right)$ of type $(p,q)$, we define their product to be the tensor product $S = T \otimes P$ of type $(k+p,l+q)$ with components:
  \begin{align*}
    S^{i_{1,\ldots,i_{k+p}}}_{j_1,\ldots,j_{l+q}} = T^{i_1,\ldots,i_k}_{j_1,\ldots,j_l}P^{i_{k+1},\ldots,i_p}_{j_{l+1},\ldots,j_q}
  \end{align*}
  This multiplication is \emph{not commutative} but it is associative. 

\item The result of applying the above three operations to tensors are again tensors. 
\end{itemize}

\subsection{Co-Vectors}
\begin{itemize}
  \item Recall that the differential of a function $f$ of $x^{1},\ldots,x^n$ corresponding to the increments $dx^i$ in the $x^i$ is:
  \begin{align*}
    df = \frac{\partial f}{\partial x^i}dx^i
  \end{align*}
  Since $dx^i$ is a vector $df$ has the same value in any co-ord system. 
  In general, given any co-vector $(T_i)$, the differential form $T_idx^i$ is invariant under change of chart. We can thus identify $dx^i \equiv e^{i}$ as the \emph{canonical basis of co-vectors or cotangent space}. 
\end{itemize}

\subsection{Skew-Symmetric Tensor}
\begin{itemize}
  \item A \emph{skew-symmetric tensor} of type $(0,k)$ is a tensor $T_{i_1,\ldots,i_k}$ satisfying:
  \begin{align*}
     T_{\sigma(i_1,\ldots,i_k)} = \mathfrak{s}(\sigma)T_{i_1,\ldots,i_k}
   \end{align*} 
   where for all permutations $\mathfrak{s}(\sigma)$ is the sign function. i.e. $\mathfrak{s}(\sigma) = +1(-1)$ for even(odd) permutation. If two indices of $T_{i_1,\ldots,i_k}$ are the same then the corresponding component of $T_{i_1,\ldots,i_k}$ is $0$. This means if $k>n$ the tensor is automatically $0$. 

     \item The standard basis at a given point is:  
     \begin{align*}
       dx^{i_1}\wedge\cdot \cdot \cdot \wedge dx^{i_k},~~~i_{1}<i_2<\cdot\cdot\cdot<i_k
     \end{align*}
     Where:
     \begin{align*}
       dx^{i_1}\wedge\cdot \cdot \cdot \wedge dx^{i_k} = \sum_{\sigma\in S_{k}}\mathfrak{s}(\sigma)e^{i_{\sigma_1}}\otimes\cdot \cdot \cdot \otimes e^{i_{\sigma_k}}
     \end{align*}
     Here $S_k$ is the symmetric group. i.e. the group of all permutations of $k$ elements. 

     \item The differential form of the skew-symmetric tensor $\left(T_{i_1,\ldots,i_k}\right)$ is:

     \begin{flalign*}
    T_{i_1,\ldots,i_k}e^{i_1}\otimes \cdot \cdot \cdot \otimes e^{i_k} &= \sum_{i_{1}<i_2<\cdot\cdot\cdot<i_k}    T_{i_1,\ldots,i_k} dx^{i_1}\wedge\cdot \cdot \cdot \wedge dx^{i_k}   \\
    & = \frac{1}{k!}T_{i_1,\ldots,i_k}dx^{i_1}\wedge\cdot \cdot \cdot \wedge dx^{i_k}
     \end{flalign*}
     Where the last step can be made as both $dx^{i_1}\wedge\cdot \cdot \cdot \wedge dx^{i_k}$ and $T_{i_1,\ldots,i_k}$ are anti-symmetric. 
\end{itemize}

\subsection{Volume element}
\begin{itemize}
  \item A metric $g_{ij}$ on a manifold is a tensor of type $(0,2)$ and on an oriented manifold of $dim(M)=n$ such a metric gives rise to a \emph{volume element}: 
  \begin{align*}
    T_{i_1,\ldots,i_n} = \sqrt{|g|}\epsilon{i_1,\ldots,i_n},~~~g=det(g_{ij})
  \end{align*}
  It is convenient to write the volume element in the notation of differential forms:
  \begin{align*}
    \Omega = \sqrt{|g|}dx^{1}\wedge \cdot \cdot \cdot \wedge dx^{n}
  \end{align*}
  If $g_{ij}$ is Riemann then the \emph{volume} $V$ of $M$ is:
  \begin{flalign*}
    V = \int_M \Omega  = \int_{M}\sqrt{|g|}dx^{1}\wedge \cdot \cdot \cdot \wedge dx^{n}
  \end{flalign*}
\end{itemize}

\subsection{Generalized push forward}
\begin{itemize}
  \item We can generalize the push froward map we had on vectors earlier to the space of tensors $(k,0)$:
  \begin{flalign*}
    f_{\ast}: \xi^{i_1,\ldots,i_k} \rightarrow \eta^{a_1,\ldots,a_k} = \frac{\partial f^{a_1}}{\partial x^{i_1}} \cdot \cdot \cdot \frac{\partial f^{a_k}}{\partial x^{i_k}} \xi^{i_1,\ldots,i_k}
  \end{flalign*}

\end{itemize}
\subsection{Pull back}
\begin{itemize}
  \item Let $T_x^{(0,k)}M$ denote the space of tensors of type $(0,k)$ at $x \in M$. Let $f$ be a smooth map from $M$ to $N$. It gives rise to a map:
  \begin{align*}
    f^{\ast}: T_{f(x)}^{(0,k)}N \rightarrow T_x^{(0,k)}M
  \end{align*}

  which in terms of $x^i \in U \subset M$, and $y^a \in V \subset N$ is written as:
  \begin{flalign*}
     f^{\ast}: \eta_{a_1,\ldots,a_k} \rightarrow \xi_{i_1,\ldots,i_k} = \frac{\partial f^{a_1}}{\partial x^{i_1}} \cdot \cdot \cdot \frac{\partial f^{a_k}}{\partial x^{i_k}}\eta_{a_1,\ldots,a_k} 
   \end{flalign*} 
   The map $f^{\ast}$ is called the \emph{pullback}. 

   \item We can then note the following relationship between pullbacks and push forwards. Let us denote the action of a vector on another vector as follows:
   \begin{align*}
     \zeta(\theta) \equiv \zeta_{i_1,..,i_k}\theta^{i_1,\ldots,i_k}
   \end{align*}
   Then we can write that:
   \begin{align*}
     (f^{\ast}\eta)(\xi) & = \frac{\partial f^{a_1}}{\partial x^{i_1}} \cdot \cdot \cdot \frac{\partial f^{a_k}}{\partial x^{i_k}}\eta_{a_1,\ldots,a_k}\xi^{i_1,\ldots,i_k} = \eta(f_{\ast}\xi)
   \end{align*}
\end{itemize}

\newpage

\section{Manifolds and surfaces}
\subsection{Immersion}
\begin{itemize}
  \item A manifold $M$ of dim $m$ is said to be immersed in a manifold $N$ of dim $n \geq m$ if $\exists$ a smooth map $f:M \rightarrow N$  such that the push forward map $f_{\ast}$ is at each point a one to one map of the tangent space. 

  The map $f$ is called the \emph{immersion} of $M$ to $N$.

  Since $f_{\ast}$ is at each a point one to one map of the tangent space, in terms of local co-ords the Jacobian matrix of $f$ at each point has rank equal to $m = $ dim $M$. 
\end{itemize}

\subsubsection{Embedding}
\begin{itemize}
  \item An immersion of $M$ to $N$ is called an $embedding$ if it one to one. Then $M$ is called a \emph{sub-manifold} of $N$. 

  \item To see the difference between these two definitions note that a Klein bottle is immersed in $\RR^3$ but not embedded as its tangent spaces are distinct (intersecting points can have different tangent spaces) but the map of points is not one- to one as there are cross overs. 
\end{itemize}

\subsection{Manifold with boundary}
\begin{itemize}
  \item A closed region $A$ of a manifold $M$ defined by an inequality:
  \begin{align*}
    f(x) \leq  0, ~~~(\text{or} f(x) \geq 0)
  \end{align*}
  where $f$ is a real-valued function on $M$. This region is a \emph{Manifold with boundary}. It is assumed that the boundary $\partial A$ given by $f(x)=0$ is a non-singular sub-manifold of $M$ i.e. $\nabla f \neq 0 $ on $\partial A$. 
\end{itemize}
\subsubsection{Closed manifold}
\begin{itemize}
  \item A compact manifold without a boundary is called \emph{closed}. 
\end{itemize}

\subsection{Surfaces as Manifolds}
\begin{itemize}
  \item A \emph{Non-singular surface} $M$ of dimension $k$ in $n$-dim Euclidean space is given by a set of $n-k$ equations:
  \begin{align*}
  f_i(x^1,..,x^n) = 0,~~~i=1,\ldots,n-k
  \end{align*}
  where $\forall x$ the matrix $\left(\frac{\partial f_i}{\partial x^{\alpha}}\right)$ has rank $n-k$. 
\end{itemize}

\subsection{Orientation of surfaces }
\subsubsection{Orientation class}
\begin{itemize}
  \item Consider a frame $\tau_1 = (e_1^{(1)},\ldots,e_n^{(1)})$ called an ordered basis and another frame $\tau_1 = (e_1^{(2)},\ldots,e_n^{(2)})$ then we say that they lie in the \emph{same orientation class} if $det A>0$ and the \emph{opposite orientation class} if $det A<0$. Where $A$ is defined as:
  \begin{align*}
  A: e^{(1)}_k: \rightarrow e^{(2)}_k
  \end{align*}
\end{itemize}
\subsubsection{Orientability} 
\begin{itemize}
   \item A manifold is said to be \emph{orientable} if it is possible to choose at every point of it a single orientation class depending continuously on the points.

    A particular choice of such an orientation class for each point is called an orientation of the manifold, and a manifold equipped with a particular orientation is said to be \emph{oriented}.

    If no orientation exists the manifold is said to be \emph{non-orientable}
 \end{itemize} 

 \subsection{Two-sided hyper-surface}
 \begin{itemize}
   \item A connected $(n-1)$-dim sub-manifold of $\RR^n$ is called two sided if a single valued continuous field of unit normals can be defined on it.  

   such a sub-manifold is called a \emph{two-sided hyper-surface}. 
 \end{itemize}

\newpage

 \section{Lie Groups}
 \subsection{Group}
 \begin{itemize}
   \item A \emph{group} is a non-empty set $G$ on which there is defined a binary operation $(a,b) \rightarrow ab$ satisfying the following properties:
   \begin{itemize}
     \item Closure: If $a$ and $b$ belong to $G$, then $ab \in G$.
     \item Associativity: $\forall a,b,c \in G,~~~a(bc) = (ab)c$.
     \item Identity: $\exists$ an element $1\in G$ st: $a1=1a =a,~~\forall a \in G$
     \item Inverse: If $a\in G$ then $\exists~a^{-1}\in G $ st: $aa^{-1} = a^{-1}a = 1$. 
   \end{itemize}
    \end{itemize}

    \subsection{Lie Group}
    \begin{itemize}
      \item A manifold $G$ is called a \emph{Lie Group} if it has given on it a group operation with the properties that the maps $\varphi:G \rightarrow G$, defined by $\varphi(g) = g^{-1}$ and $\psi:G \times G \rightarrow G$ defined by $\psi(g,h) = gh$, are smooth maps. 
    \end{itemize}

    \subsection{Example of Lie groups}
    \subsubsection{General Linear group}
    \begin{itemize}
      \item This is $GL(n,\RR)$ consisting of all $n \times n$ real matrices with non zero determinant in a region $\RR^{n^2}$. dim $GL(n,\RR) = n^2$.
    \end{itemize}


    \subsubsection{Special Linear group}
    \begin{itemize}
      \item This is $SL(n,\RR)$ consisting of all $n \times n$ real matrices with determinant equal to $1$. It is a hyper-surface in $\RR^{n^2}$.  
\begin{align*}
det A =1,~~~A \in Mat(n,\RR)
\end{align*}
      dim $ SL(n,\RR) = n^2-1$.
    \end{itemize}


       \subsubsection{Orthogonal group}
    \begin{itemize}
      \item This is $O(n,\RR)$ consisting of all $n \times n$ real matrices Satisfying:  
      \begin{align*}
A^T \cdot A = \mathbb{I},~~~A \in Mat(n,\RR)
\end{align*}
dim $O(n,\RR) = \frac{1}{2}n(n-1)$.
    \end{itemize}

\subsubsection{Orthogonal group}
    \begin{itemize}
      \item This is $O(n,\RR)$ consisting of all $n \times n$ real matrices Satisfying:  
      \begin{align*}
A^T \cdot A = \mathbb{I},~~~A \in Mat(n,\RR)
\end{align*}
dim $O(n,\RR) = \frac{1}{2}n(n-1)$.
    \end{itemize}
\end{document}
 