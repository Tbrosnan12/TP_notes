\documentclass[11pt]{article}

\usepackage[letterpaper,top=2cm,bottom=2cm,left=2cm,right=2cm,marginparwidth=1.75cm]{geometry}
\usepackage{hyperref}
\usepackage{biblatex}
\addbibresource{Bib.bib}
\usepackage{mathtools}
\DeclarePairedDelimiterXPP\BigOSI[2]%
  {\mathcal{O}}{(}{)}{}%
  {\SI{#1}{#2}}
\usepackage{xcolor}
\usepackage{empheq}
\usepackage[most]{tcolorbox}
\usepackage{amsmath}
\usepackage{amssymb}
\usepackage{mathrsfs}
\usepackage[utf8]{inputenc}
\usepackage{graphicx}
\usepackage{float}
\usepackage{parskip}
\usepackage{comment}
%\usepackage{mhchem}
 \usepackage{tabularx}
 \usepackage{titling}
 \usepackage{amsmath,environ}
 \usepackage[explicit]{titlesec}
\usepackage{fancyhdr}
\usepackage{braket}
\usepackage{bm}
\setlength{\droptitle}{3em} 

\title{Differential Geometry}
\author{Thomas Brosnan}
\date{Notes taken in Professor Sergey Frolov's class, Michaelmas Term 2024}


\newtcbox{\mymath}[1][]{%
    nobeforeafter, math upper, tcbox raise base,
    enhanced, colframe=blue!30!black,
    colback=blue!30, boxrule=1pt,
    #1}
\tcbset{highlight math style={boxsep=2mm,,colback=blue!0!green!0!red!0!}}

\newenvironment{bux}{\empheq[box=\tcbhighmath]{align}}{\endempheq}
\newenvironment{bux*}{\empheq[box=\tcbhighmath]{align*}}{\endempheq}
\renewenvironment{flalign}{\vspace{-3mm}\empheq[box=\tcbhighmath]{align}}{\endempheq}
\renewenvironment{flalign*}{\vspace{-3mm}\empheq[box=\tcbhighmath]{align*}}{\endempheq}
%\renewenvironment{align}{\vspace{-5mm}\begin{align}}{\end{align}}
%\renewenvironment{align*}{\vspace{-5mm}\begin{align*}}{\end{align*}}
\renewenvironment{alignat}{\empheq{align*}}{\endempheq}
\newcommand{\hsp}{\hspace{8pt}}

\newcommand*{\sectionFont}{%
  \LARGE\bfseries
}

   
\numberwithin{equation}{section}


\DeclareRobustCommand{\RR}{\mathbb{R}}
\DeclareRobustCommand{\CC}{\mathbb{C}}

\makeatletter
\let\Title\@title % Copy the title to a new command
\makeatother

%change this RGB value to change the section background colour 
\definecolor{mycolor1}{RGB}{164, 196, 34}
\colorlet{SectionColour}{mycolor1}
%subsection background colour 
\definecolor{mycolor2}{gray}{0.8}
\colorlet{subSectionColour}{mycolor2}
%subsubsection background colour 
\definecolor{mycolor3}{RGB}{255,255,255}
\colorlet{subsubSectionColour}{mycolor3}

\newcommand{\I}[1]{\emph{#1}}

\begin{document}

\maketitle

\newpage
\topskip0pt
\vspace*{\fill}
\begin{center}
\Large ``hokay''
          -Sergey Frolov
\end{center}
\vspace*{\fill}
\newpage 
\tableofcontents
% For \section
 \titleformat{\section}[block]{\sectionFont}{}{0pt}{%
 \fcolorbox{black}{SectionColour}{\noindent\begin{minipage}{\dimexpr\textwidth-2\fboxsep-2\fboxrule\relax}\thesection  \hsp #1 {\strut} \end{minipage}}}
% For \subsection
 \titleformat{\subsection}[block]{\bfseries}{}{0pt}{%
 \fcolorbox{black}{subSectionColour}{\noindent\begin{minipage}{\dimexpr\textwidth-2\fboxsep-2\fboxrule\relax}\thesubsection  \hsp #1 {\strut} \end{minipage}}}
% For \section*
 \titleformat{name=\section, numberless}[block]{\sectionFont}{}{0pt}{%
 \fcolorbox{black}{SectionColour}{\noindent\begin{minipage}{\dimexpr\textwidth-2\fboxsep-2\fboxrule\relax} #1 {\strut} \end{minipage}}}
  % For \subsection*
 \titleformat{name=\subsection, numberless}[block]{\bfseries}{}{0pt}{%
 \fcolorbox{black}{subSectionColour}{\noindent\begin{minipage}{\dimexpr\textwidth-2\fboxsep-2\fboxrule\relax} #1 {\strut} \end{minipage}}}
 % For \subsubsection
 \titleformat{\subsubsection}[block]{\bfseries}{}{0pt}{%
 \fcolorbox{black}{subsubSectionColour}{\noindent\begin{minipage}{15cm}\thesubsubsection \hsp #1 {\strut} \end{minipage}}}
  % For \subsubsection*
 \titleformat{name=\subsubsection, numberless}[block]{\bfseries}{}{0pt}{%
 \fcolorbox{black}{subsubSectionColour}{\noindent\begin{minipage}{15cm} #1 {\strut} \end{minipage}}}
\newpage 
%header 
\pagestyle{fancy}
\fancyhf{} % Clear all header and footer fields
\fancyhead[L]{\Title}
\fancyhead[R]{\nouppercase{\leftmark}}
\fancyfoot[C]{-~\thepage~-}
\renewcommand{\headrulewidth}{1pt}



%starting document 
\normalsize
\newpage
\section{Definition of a Manifold}

\subsection{Regions}
\begin{itemize}
  \item A \emph{region}  (``open set'') is a set of $D$ points in $\RR^{n}$ such that together with each point $p_0$, D also contains all points sufficiently closer to $p_0$, i.e.: 
  \begin{align*}
  & \forall ~p_0  = (x_0^1,\ldots,x_0^n)\in D ~\exists ~\epsilon> 0, \\ 
   st: &p=(x^1,\ldots,x^n) \in D, ~\text{iff}~ |x^i-x^i_0| < \epsilon . 
  \end{align*}
  \item A \emph{region with out a boundary} is obtained fro ma region $D$ by adjoining all boundary points to D. The \emph{boundary} of a region is the set of all boundary points.  
\end{itemize}

\subsection{Differentiable Manifold}
\begin{itemize}
  \item A differentiable $n$-dimensional manifold is a set $M$ together with the following structure on it. The set $M$ is the union of a finite or countably infinite collection of subsets $U_{q}$ with the following properties: 
  \begin{itemize}
     \item  Each subset $U_{q}$ has defined on it co-ords $x_{q}^{\alpha}, \alpha = 1,\ldots,n$ called local co-ords by virtue of which $U_{q}$ is identifiable with a region of Euclidean $n$-space $\RR^n$ with Euclidean co-ords $x^{\alpha}_q$. The $U_{q}$ with their co-ord systems are called \emph{charts} or \emph{local coordinate neighborhoods}. 

\item Each non-empty intersection $U_{q}\cap U_{p}$ of a pair of charts thus has defined on it two co-ord systems, the restriction of $x^{\alpha}_{p}$ and $x^{\alpha}_{q}$. It is required that under each of these coordinatizations the intersection $U_{q}\cap U_{p}$ is identifiable with a region of $\RR^n$ and that each of these co-ordinate systems be expressible in terms of the other in a one to one differentiable manner. Thus, if a the \emph{transition} functions from $x^{\alpha}_{p}$ to $x^{\alpha}_{q}$ and back are given by:
 \begin{align*}
 & x^{\alpha}_{p} = x^{\alpha}_{p}( x^{1}_{q},\ldots, x^{n}_{q}),~~\alpha = 1,\ldots,n \\
 & x^{\alpha}_{q} = x^{\alpha}_{q}( x^{1}_{p},\ldots, x^{n}_{p}),~~\alpha = 1,\ldots,n
 \end{align*}
 Then the \emph{Jacobian} $J_{pq} = det(\partial x^{\alpha}_{p}/\partial x^{\alpha}_{q})$ is non-zero on $U_{p} \cap U_{q}$. 
  \end{itemize} 
\end{itemize}

\subsection{Abuse of notation}
\begin{itemize}
  \item Regular partial derivative do not have the same ``canceling'' that total derivative have ($dx * dy/dx = dy$) But we can restore this property through Einstein summation convention. That is that:
  \begin{flalign*}
  \sum_{\gamma=1}^{n}\frac{\partial x^{\alpha}_{p}}{\partial x^{\gamma}_{q}}\frac{\partial x^{\gamma}_{q}}{\partial x^{\beta}_{q}} = \frac{\partial x^{\alpha}_{p}}{\partial x^{\gamma}_{q}}\frac{\partial x^{\gamma}_{q}}{\partial x^{\beta}_{q}}  = \delta^{\alpha}_{\beta}
  \end{flalign*}

\end{itemize}

\newpage 
\section{Elements of Topology}
\subsection{Topological space}
\begin{itemize}
  \item A topological space is a set $X$ of points of which certain subsets called \emph{open sets} of the topological space, are distinguished, these open sets have to satisfy: 
  \begin{itemize}
    \item The intersection of any two (and hence of any finite collection) open sets should again be an open set.
    \item The union of any collection of open sets must again be open. 
    \item The empty set and the whole set $X$ must be open.
  \end{itemize}

  \item The compliment of any open  set is called a \emph{closed} set of the topological space. 

  In Euclidean space $\RR^n$ the ``Euclidean topology'' is the usual one where the open sets are the open regions. 
\end{itemize}
  \subsubsection{Induced topology} 
  \begin{itemize}
    \item Given any subset $A \in \RR^n$, the \emph{induced topology} on $A$ is that where the open sets are the intersections $A \cap U$, where $U$ ranges over all open sets of $\RR^{n}$. 
  \end{itemize}
\subsubsection{Continuity}
\begin{itemize}
  \item A map $f: X \rightarrow Y$ of one topological space to another is called \emph{continuous} if the complete inverse image $f^{-1}(U)$ of every open set $U \subset Y$ is open in X. 
\end{itemize}

\subsubsection{Homeomorphic}
\begin{itemize}
  \item Two topological space are \emph{topologically equivalent} or \emph{homeomorphic} if there is a one to one and onto map (bijective) between them, such that it and its inverse are continuous. 
\end{itemize}
\subsubsection{Topology on a manifold}
\begin{itemize}
  \item The topology on a manifold $M$ is given by the following specifications of the open sets. In every local co-ordinate neighborhood $U_q$ the open regions are to be open in the topology on M; the totality of open sets of M is then obtained by admitting as open, also arbitrary unions countable collections of such regions, i.e. by closing under countable unions. 
\end{itemize}
\subsection{Metric space}
\begin{itemize}
  \item A \emph{metric space} is a set which comes equipped with a ``distance function'' i.e. a real-valued function $\rho(x,y)$, defined on pairs $x, y$ of its elements and having the following properties:
  \begin{itemize}
    \item Symmetry: $\rho(x,y) = \rho(y,x)$.
    \item Positivity: $\rho(x,x) =0, ~~~ \rho(x,y) > 0 ~~\text{if}~~ x \neq y$. 
    \item The triangle inequality: $\rho(x,y) \leq \rho(x,z)+\rho(z,y)$.
  \end{itemize}
\end{itemize}
\subsubsection{Hausdorff}
\begin{itemize}
  \item A topological space is called \emph{Hausdorff} if any two points are contained in disjoint open sets. Any metric space is Hausdorff because the open balls of radius $\rho(x,y)/3$ with centers at $c,y$ do not intersect.   

  All topological spaces we consider will be Hausdorff. 
\end{itemize}

\subsubsection{Compact}
\begin{itemize}
  \item A topological space $X$ is said to be compact if every countable collection of open sets covering $X$ contains a finite sub-collection already covering $X$.

  If $X$ is a metric space the compactness is equivalent to the condition that from every sequence of points of $X$ a convergent sub-sequence can be selected.  
\end{itemize}
\subsubsection{Connected}
\begin{itemize}
  \item A topological space is connected if any two points can be joined by a continuous path. 
\end{itemize}

\subsection{Orientation}
\begin{itemize}
  \item A manifold $M$ is said to be \emph{orientated} of one can choose its atlas (collection of all the charts) so that for every pair $U_{p},U_{q}$ of intersecting co-ordinate neighborhoods the Jacobian of the transition functions is positive.  

  \item We say that the co-ordinate systems $x$ and $y$ define the \emph{same orientation} if $J>0$ and the \emph{opposite orientation} if $J<0$. 
\end{itemize}
\newpage
\section{Mappings on Manifolds}
\subsection{Manifold mappings}
\begin{itemize}
  \item A mapping $f: M \rightarrow N$ is said to be smooth of smoothness class $k$ if for all $p,q$ for which $f$ determines functions $y^{b}_{q}(x^{1}_{p},\ldots,x^{m}_{p}) = f(x^{1}_{p},..,x^m_p)^{b}_{p}$, these functions are, where defined, smooth of smoothness class $k$ (i.e. all their partial derivatives up to those of $k$-th order exist and are continuous).

  the smoothness class of $f$ cannot exceed the maximum class of the manifolds.  
\end{itemize}

\subsection{Equivalent manifolds}
\begin{itemize}
  \item The manifolds $M$ and $N$ are said to be \emph{smoothly equivilent} or \emph{diffeomorphic} if there is a one to one and onto map $f$ such that both $f:M \rightarrow N$ and $f^{-1}:N \rightarrow M$ are smooth of some class $k\geq 1$. 

  Since $f^{-1}$ exits then the Jacobian $J_{pq} \neq 0 $ wherever it is defined. 
\end{itemize}

\subsection{Tangent vector}
\begin{itemize}
  \item A \emph{tangent} vector to an $m$-dim manifold $M$ at an arbitrary point $x$ is represented in terms of local co-ords $x^{\alpha}_-p$ by an $m$ tuple $\xi^{\alpha}$ of components which are linked to the components in terms of any other system $x^{\beta}_{q}$ of local co-ords by:
  \begin{align}
  \label{tangent}
    \xi^{\alpha}_{p} = \left(\frac{\partial x^{\alpha}_{p}}{\partial x^{\beta}_{q}}\right)_{x} \xi^{\beta}_{q},~~~\forall~ \alpha
  \end{align}

  \item The set of all tangent vectors to an $m$-dim manifold $M$ at a point $x$ forms an $m$-dm vector space $T_{x} = T_{x}M$, the \emph{tangent space} to $M$ at the point $x$. 

   \item Thus, the velocity at $x$ of any smooth curve $M$ through $x$ is a tangent vector to $M$ at $x$. 
\end{itemize}

\subsection{Push forward}
\begin{itemize}
  \item A smooth map $f$ from $M$ to $N$ gives rise for each $x$ to a \emph{push forward} or an \emph{induced linear} map to tangent spaces:
  \begin{align*}
      f_{\ast}: T_{x}M \rightarrow T_{f(x)}N
    \end{align*}  
    defined as sending the velocity at $x$ of any smooth curve $x =x(\tau)$ on $M$ to the velocity vector at $f(x)$ of the curve $f(x(\tau))$ on $N$. If the map $f$ is given by: $y^b = f^b(x^{1},\ldots,x^m)$ for $x \in M$ and $y \in N$, then the push forward map $f_{\ast}$ is:
    \begin{align*}
      \xi^{\alpha} \rightarrow \eta^{b} = \frac{\partial f^{b}}{\partial x^{\alpha}}\xi^{\alpha}. 
    \end{align*}

    \item For a real valued function $f: M -> \RR$, the push-forward map $f_{\ast}$ corresponding to each $x\in M$ is a real valued linear function on the tangent space to $M$ at $x$: 
    \begin{align*}
      \xi^{a} \rightarrow \eta = \frac{\partial f}{\partial x^{\alpha}}\xi^{\alpha}
    \end{align*}
    and it is represented by the gradiant of $f$ at $x$, and is a co-vector or one form. Thus $f_{\ast}$ can be identified with the differential $df$, in particular:
    \begin{align*}
      dx^{\alpha}_{p}: \xi^{\alpha}\rightarrow\eta =\xi^{\alpha}_{p}
    \end{align*}
\end{itemize}

\subsection{Directional derivative}
\begin{itemize}
  \item We can associate with each vector $\xi = (\xi^{i})$ a linear differential operator as follows: Since the gradient $\frac{\partial f}{\partial x^i}$ of a function $f$ is a co-vector, the quantity:
  \begin{align*}
    \partial_{\xi}f = \xi^{i}\frac{\partial f}{\partial x^{i}}
  \end{align*}
  is a scalar called the directional derivative of $f$ in the direction of $\xi$. 

  \item Thus an arbitrary vector $\xi$ corresponds to the operator:
  \begin{align*}
    \partial_{\xi} = \xi^{i}\frac{\partial }{\partial x^i}
  \end{align*}
  So we can identify $\frac{\partial}{\partial x^i}\equiv e_{i}$ as the \emph{Canonical basis of the tangent space}. 
\end{itemize} 

\subsection{Riemann metric}
\begin{itemize}
  \item A \emph{Riemann metric} on a manifold $M$ is a point-dependent, positive-definite quadratic form on the tangent vectors at each point, depending smoothly on the local co-ords of the points. 

  Thus at each point $x = (x^{1}_{p},\ldots,x^{m}_p)$ of each chart $U_{p}$, the metric is given by a symmetric metric $g_{\alpha\beta}(x^{1}_{p},...,x^{m}_p)$, and determines a symmetric scalar product of pairs of tangent vectors at the point $x$. 
  \begin{align*}
     \braket{\xi, \eta}  = g_{\alpha\beta}^{(p)}\xi^{\alpha}_{p}\eta^{\beta}_{p} = \braket{\eta,\xi}, ~~~ |\xi|^2 = \braket{\xi,\xi}
   \end{align*} 
   This scalar product is to be co-ordinate independent:
   \begin{align*}
     g_{\alpha\beta}^{(p)}\xi^{\alpha}_{p}\eta^{\beta}_{p} = g_{\alpha\beta}^{(q)}\xi^{\alpha}_{q}\eta^{\beta}_{q}
   \end{align*}
   And therefor the coefficients $g_{\alpha\beta}^{(p)}$ of the quadratic form transform as:
   \begin{align}
   \label{g}
     g_{\gamma\delta}^{(q)}  = \frac{\partial x^{\alpha}_{p}}{\partial x^{\gamma}_{q}}\frac{\partial x^{\beta}_p}{\partial x^{\delta}_{q}}g_{\alpha\beta}^{(p)} 
   \end{align}
   For a \emph{pseudo-Riemann} metric $M$ one just requires the quadratic form to be \emph{nondegenerate},(i.e. the determinant of $g$ is not 0). Note that \ref{g} can be re-written as:
   \begin{align*}
     ds^2 = g_{\alpha\beta}^{(p)}dx^{\alpha}_{p}dx^{\beta}_{p} =g_{\alpha\beta}^{(q)}dx^{\alpha}_{q}dx^{\beta}_{q}  
   \end{align*}
   Where $ds$ is called a line element, and it is chart-independent. $ds$ is used to measure the distance between two infinitesimally close points. 
\end{itemize}

\newpage 
\section{Tensors}
\subsection{Tensor def}
\begin{itemize}
  \item A \emph{tensor of type} $(k,l)$ and rank $k+l$ on an $m$-dim manifold $M$ is given each local co-ord system $(x^i_p)$ by a family of functions:
  \begin{align*}
    ^{(p)}T^{i_1,\ldots,i_k}_{j_1,\ldots,j_l}(x) ~~\text{of the point}~x.
  \end{align*}
  In other local co-ord $(x^i_q)$ the components $^{(p)}T^{i_1,\ldots,i_k}_{j_1,\ldots,j_l}(x)$  of the same tensor are:
  \begin{align*}
    ^{(p)}T^{s_1,\ldots,s_k}_{t_1,\ldots,t_l}(x) = \frac{\partial x^{s_1}_q}{\partial x^{i_1}_p}\cdot \cdot \cdot\frac{\partial x^{s_k}_q}{\partial x^{i_k}_p}\frac{\partial x^{j_1}_p}{\partial x^{t_1}_q}  \cdot \cdot \cdot \frac{\partial x^{j_l}_p}{\partial x^{t_l}_q} \cdot ~^{(p)}T^{i_1,\ldots,i_k}_{j_1,\ldots,j_l}(x)
  \end{align*}
\end{itemize}
\subsection{Operations on Tensors}
\subsubsection{Permutation of indices}
\begin{itemize}
  \item Let $\sigma$ be some permutation of $1,2,\ldots,l$. $\sigma$ acrs on the ordered tuple $(j_1,\ldots,j_l)$ as $\sigma(j_1,\ldots,j_l) = (j_{\sigma_1},\ldots,j_{\sigma_l})$. We say that a tensor $\tilde{T}^{i_1,\ldots,i_k}_{j_1,\ldots,j_l}(x)$ =is obtained from a tensor $T^{i_1,\ldots,i_k}_{j_1,\ldots,j_l}(x)$ by means of a permutation $\sigma$ of the lower indices if at each point of $M$:
  \begin{align*}
    \tilde{T}^{i_1,\ldots,i_k}_{j_1,\ldots,j_l}(x) = T^{i_1,\ldots,i_k}_{\sigma(j_1,\ldots,j_l)}(x)
  \end{align*}
  Permutation of upper indicies is defined similarly. 
\end{itemize}
\subsubsection{Contraction of indicies}
\begin{itemize}
  \item By the contraction of a tensor $T^{i_1,\ldots,i_k}_{j_1,\ldots,j_l}(x)$ of type $(k,l)$ with respect to the indcies $i_a,j_a$ we mean the tensor (summation over $n$):
  \begin{align*}
    T^{i_1,\ldots,i_{k-1}}_{j_1,\ldots,j_{l-1}}(x)=T^{i_1,\ldots i_{a-1},n,i_{a+1},\ldots,i_k}_{j_1,\ldots ,j_{a-1},n,j_{a+1},\ldots,j_l}(x)
  \end{align*}
  Of type $(k-1,l-1)$
\end{itemize}
\subsubsection{Product of Tensors}
\begin{itemize}
  \item Given two tensors $T = \left(T^{i_1,\ldots,i_k}_{j_1,\ldots,j_l}\right)$ of type $(k,l)$ and $P = \left(P^{i_1,\ldots,i_p}_{j_1,\ldots,j_q}\right)$ of type $(p,q)$, we define their product to be the tensor product $S = T \otimes P$ of type $(k+p,l+q)$ with components:
  \begin{align*}
    S^{i_{1,\ldots,i_{k+p}}}_{j_1,\ldots,j_{l+q}} = T^{i_1,\ldots,i_k}_{j_1,\ldots,j_l}P^{i_{k+1},\ldots,i_p}_{j_{l+1},\ldots,j_q}
  \end{align*}
  This multiplication is \emph{not commutative} but it is associative. 

\item The result of applying the above three operations to tensors are again tensors. 
\end{itemize}

\subsection{Co-Vectors}
\begin{itemize}
  \item Recall that the differential of a function $f$ of $x^{1},\ldots,x^n$ corresponding to the increments $dx^i$ in the $x^i$ is:
  \begin{align*}
    df = \frac{\partial f}{\partial x^i}dx^i
  \end{align*}
  Since $dx^i$ is a vector $df$ has the same value in any co-ord system. 
  In general, given any co-vector $(T_i)$, the differential form $T_idx^i$ is invariant under change of chart. We can thus identify $dx^i \equiv e^{i}$ as the \emph{canonical basis of co-vectors or cotangent space}. 
\end{itemize}

\subsection{Skew-Symmetric Tensor}
\begin{itemize}
  \item A \emph{skew-symmetric tensor} of type $(0,k)$ is a tensor $T_{i_1,\ldots,i_k}$ satisfying:
  \begin{align*}
     T_{\sigma(i_1,\ldots,i_k)} = \mathfrak{s}(\sigma)T_{i_1,\ldots,i_k}
   \end{align*} 
   where for all permutations $\mathfrak{s}(\sigma)$ is the sign function. i.e. $\mathfrak{s}(\sigma) = +1(-1)$ for even(odd) permutation. If two indices of $T_{i_1,\ldots,i_k}$ are the same then the corresponding component of $T_{i_1,\ldots,i_k}$ is $0$. This means if $k>n$ the tensor is automatically $0$. 

     \item The standard basis at a given point is:  
     \begin{align*}
       dx^{i_1}\wedge\cdot \cdot \cdot \wedge dx^{i_k},~~~i_{1}<i_2<\cdot\cdot\cdot<i_k
     \end{align*}
     Where:
     \begin{align*}
       dx^{i_1}\wedge\cdot \cdot \cdot \wedge dx^{i_k} = \sum_{\sigma\in S_{k}}\mathfrak{s}(\sigma)e^{i_{\sigma_1}}\otimes\cdot \cdot \cdot \otimes e^{i_{\sigma_k}}
     \end{align*}
     Here $S_k$ is the symmetric group. i.e. the group of all permutations of $k$ elements. 

     \item The differential form of the skew-symmetric tensor $\left(T_{i_1,\ldots,i_k}\right)$ is:

     \begin{flalign*}
    T_{i_1,\ldots,i_k}e^{i_1}\otimes \cdot \cdot \cdot \otimes e^{i_k} &= \sum_{i_{1}<i_2<\cdot\cdot\cdot<i_k}    T_{i_1,\ldots,i_k} dx^{i_1}\wedge\cdot \cdot \cdot \wedge dx^{i_k}   \\
    & = \frac{1}{k!}T_{i_1,\ldots,i_k}dx^{i_1}\wedge\cdot \cdot \cdot \wedge dx^{i_k}
     \end{flalign*}
     Where the last step can be made as both $dx^{i_1}\wedge\cdot \cdot \cdot \wedge dx^{i_k}$ and $T_{i_1,\ldots,i_k}$ are anti-symmetric. 
\end{itemize}

\subsection{Volume element}
\begin{itemize}
  \item A metric $g_{ij}$ on a manifold is a tensor of type $(0,2)$ and on an oriented manifold of $dim(M)=n$ such a metric gives rise to a \emph{volume element}: 
  \begin{align*}
    T_{i_1,\ldots,i_n} = \sqrt{|g|}\epsilon{i_1,\ldots,i_n},~~~g=det(g_{ij})
  \end{align*}
  It is convenient to write the volume element in the notation of differential forms:
  \begin{align*}
    \Omega = \sqrt{|g|}dx^{1}\wedge \cdot \cdot \cdot \wedge dx^{n}
  \end{align*}
  If $g_{ij}$ is Riemann then the \emph{volume} $V$ of $M$ is:
  \begin{flalign*}
    V = \int_M \Omega  = \int_{M}\sqrt{|g|}dx^{1}\wedge \cdot \cdot \cdot \wedge dx^{n}
  \end{flalign*}
\end{itemize}

\subsection{Generalized push forward}
\begin{itemize}
  \item We can generalize the push froward map we had on vectors earlier to the space of tensors $(k,0)$:
  \begin{flalign*}
    f_{\ast}: \xi^{i_1,\ldots,i_k} \rightarrow \eta^{a_1,\ldots,a_k} = \frac{\partial f^{a_1}}{\partial x^{i_1}} \cdot \cdot \cdot \frac{\partial f^{a_k}}{\partial x^{i_k}} \xi^{i_1,\ldots,i_k}
  \end{flalign*}

\end{itemize}
\subsection{Pull back}
\begin{itemize}
  \item Let $T_x^{(0,k)}M$ denote the space of tensors of type $(0,k)$ at $x \in M$. Let $f$ be a smooth map from $M$ to $N$. It gives rise to a map:
  \begin{align*}
    f^{\ast}: T_{f(x)}^{(0,k)}N \rightarrow T_x^{(0,k)}M
  \end{align*}

  which in terms of $x^i \in U \subset M$, and $y^a \in V \subset N$ is written as:
  \begin{flalign*}
     f^{\ast}: \eta_{a_1,\ldots,a_k} \rightarrow \xi_{i_1,\ldots,i_k} = \frac{\partial f^{a_1}}{\partial x^{i_1}} \cdot \cdot \cdot \frac{\partial f^{a_k}}{\partial x^{i_k}}\eta_{a_1,\ldots,a_k} 
   \end{flalign*} 
   The map $f^{\ast}$ is called the \emph{pullback}. 

   \item We can then note the following relationship between pullbacks and push forwards. Let us denote the action of a vector on another vector as follows:
   \begin{align*}
     \zeta(\theta) \equiv \zeta_{i_1,..,i_k}\theta^{i_1,\ldots,i_k}
   \end{align*}
   Then we can write that:
   \begin{align*}
     (f^{\ast}\eta)(\xi) & = \frac{\partial f^{a_1}}{\partial x^{i_1}} \cdot \cdot \cdot \frac{\partial f^{a_k}}{\partial x^{i_k}}\eta_{a_1,\ldots,a_k}\xi^{i_1,\ldots,i_k} = \eta(f_{\ast}\xi)
   \end{align*}
\end{itemize}

\newpage

\section{Manifolds and surfaces}
\subsection{Immersion}
\begin{itemize}
  \item A manifold $M$ of dim $m$ is said to be immersed in a manifold $N$ of dim $n \geq m$ if $\exists$ a smooth map $f:M \rightarrow N$  such that the push forward map $f_{\ast}$ is at each point a one to one map of the tangent space. 

  The map $f$ is called the \emph{immersion} of $M$ to $N$.

  Since $f_{\ast}$ is at each a point one to one map of the tangent space, in terms of local co-ords the Jacobian matrix of $f$ at each point has rank equal to $m = $ dim $M$. 
\end{itemize}

\subsubsection{Embedding}
\begin{itemize}
  \item An immersion of $M$ to $N$ is called an $embedding$ if it one to one. Then $M$ is called a \emph{sub-manifold} of $N$. 

  \item To see the difference between these two definitions note that a Klein bottle is immersed in $\RR^3$ but not embedded as its tangent spaces are distinct (intersecting points can have different tangent spaces) but the map of points is not one- to one as there are cross overs. 
\end{itemize}

\subsection{Manifold with boundary}
\begin{itemize}
  \item A closed region $A$ of a manifold $M$ defined by an inequality:
  \begin{align*}
    f(x) \leq  0, ~~~(\text{or} f(x) \geq 0)
  \end{align*}
  where $f$ is a real-valued function on $M$. This region is a \emph{Manifold with boundary}. It is assumed that the boundary $\partial A$ given by $f(x)=0$ is a non-singular sub-manifold of $M$ i.e. $\nabla f \neq 0 $ on $\partial A$. 
\end{itemize}
\subsubsection{Closed manifold}
\begin{itemize}
  \item A compact manifold without a boundary is called \emph{closed}. 
\end{itemize}

\subsection{Surfaces as Manifolds}
\begin{itemize}
  \item A \emph{Non-singular surface} $M$ of dimension $k$ in $n$-dim Euclidean space is given by a set of $n-k$ equations:
  \begin{align*}
  f_i(x^1,..,x^n) = 0,~~~i=1,\ldots,n-k
  \end{align*}
  where $\forall x$ the matrix $\left(\frac{\partial f_i}{\partial x^{\alpha}}\right)$ has rank $n-k$. 
\end{itemize}

\subsection{Orientation of surfaces }
\subsubsection{Orientation class}
\begin{itemize}
  \item Consider a frame $\tau_1 = (e_1^{(1)},\ldots,e_n^{(1)})$ called an ordered basis and another frame $\tau_1 = (e_1^{(2)},\ldots,e_n^{(2)})$ then we say that they lie in the \emph{same orientation class} if $det A>0$ and the \emph{opposite orientation class} if $det A<0$. Where $A$ is defined as:
  \begin{align*}
  A: e^{(1)}_k: \rightarrow e^{(2)}_k
  \end{align*}
\end{itemize}
\subsubsection{Orientability} 
\begin{itemize}
   \item A manifold is said to be \emph{orientable} if it is possible to choose at every point of it a single orientation class depending continuously on the points.

    A particular choice of such an orientation class for each point is called an orientation of the manifold, and a manifold equipped with a particular orientation is said to be \emph{oriented}.

    If no orientation exists the manifold is said to be \emph{non-orientable}
 \end{itemize} 

 \subsection{Two-sided hyper-surface}
 \begin{itemize}
   \item A connected $(n-1)$-dim sub-manifold of $\RR^n$ is called two sided if a single valued continuous field of unit normals can be defined on it.  

   such a sub-manifold is called a \emph{two-sided hyper-surface}. 
 \end{itemize}

\newpage

 \section{Lie Groups}
 \subsection{Group}
 \begin{itemize}
   \item A \emph{group} is a non-empty set $G$ on which there is defined a binary operation $(a,b) \rightarrow ab$ satisfying the following properties:
   \begin{itemize}
     \item Closure: If $a$ and $b$ belong to $G$, then $ab \in G$.
     \item Associativity: $\forall a,b,c \in G,~~~a(bc) = (ab)c$.
     \item Identity: $\exists$ an element $1\in G$ st: $a1=1a =a,~~\forall a \in G$
     \item Inverse: If $a\in G$ then $\exists~a^{-1}\in G $ st: $aa^{-1} = a^{-1}a = 1$. 
   \end{itemize}
    \end{itemize}

    \subsection{Lie Group}
    \begin{itemize}
      \item A manifold $G$ is called a \emph{Lie Group} if it has given on it a group operation with the properties that the maps $\varphi:G \rightarrow G$, defined by $\varphi(g) = g^{-1}$ and $\psi:G \times G \rightarrow G$ defined by $\psi(g,h) = gh$, are smooth maps. 
    \end{itemize}

    \subsection{Example of Lie groups}
    \subsubsection{General Linear group}
    \begin{itemize}
      \item This is $GL(n,\RR)$ consisting of all $n \times n$ real matrices with non zero determinant in a region $\RR^{n^2}$. dim $GL(n,\RR) = n^2$.
    \end{itemize}


    \subsubsection{Special Linear group}
    \begin{itemize}
      \item This is $SL(n,\RR)$ consisting of all $n \times n$ real matrices with determinant equal to $1$. It is a hyper-surface in $\RR^{n^2}$.  
\begin{align*}
det A =1,~~~A \in Mat(n,\RR)
\end{align*}
      dim $ SL(n,\RR) = n^2-1$.
    \end{itemize}


       \subsubsection{Orthogonal group}
    \begin{itemize}
      \item This is $O(n,\RR)$ consisting of all $n \times n$ real matrices Satisfying:  
      \begin{align*}
A^T \cdot A = \mathbb{I},~~~A \in Mat(n,\RR)
\end{align*}
dim $O(n,\RR) = \frac{1}{2}n(n-1)$.
    \end{itemize}

\subsubsection{Special Orthogonal group}
    \begin{itemize}
      \item This is $SO(n,\RR)$ consisting of all $n \times n$ real matrices Satisfying:  
      \begin{align*}
A^T \cdot A = \mathbb{I},~~~det(A)=1,~~~A \in Mat(n,\RR)
\end{align*}
dim $SO(n,\RR) = \frac{1}{2}n(n-1)$.
    \end{itemize}

  \subsubsection{Pseudo Orthogonal group}
    \begin{itemize}
      \item This is $O(p,q,n)$ consisting of all $n \times n$ real matrices Satisfying:  
      \begin{align*}
A^T \cdot \eta \cdot A = \eta,~~~\text{det}(A)=1,~~~\eta = \text{diag}\{\underbrace{1,\ldots,1}_{p},\underbrace{-1,\ldots,-1}_{q}\}
\end{align*}
dim $O(p,q,n) = \frac{1}{2}n(n-1)$.
    \end{itemize}

\subsubsection{Unitary group}
    \begin{itemize}
      \item This is $U(n)$ consisting of all $n \times n$ complex matrices Satisfying:  
      \begin{align*}
A^{\dagger} \cdot A = \mathbb{I},~~~A \in Mat(n,\mathbb{C})
\end{align*}
dim $U(n) =n^2$.
\end{itemize}

\subsubsection{Special Unitary group}
    \begin{itemize}
      \item This is $SU(n)$ consisting of all $n \times n$ complex matrices Satisfying:  
      \begin{align*}
A^{\dagger} \cdot A = \mathbb{I},~~~\text{det}(A)=1,~~~A \in Mat(n,\mathbb{C})
\end{align*}
dim $U(n) =n^2-1$.
\end{itemize}

\newpage
\section{Projective spaces}
\subsection{Real protective space}
\begin{itemize}
  \item The \emph{real Projective space} $\RR P^{n}$ is the set of all straight lines in $\RR^{n+1}$ passing through the origin. Equivalently it is the set of equivalence classes of non-zero vectors in $\RR^{n+1}$ where two non-zero vectors are equivalent if they are scalar multiples of one another. 

   \item We may think of $\RR P^n$ as obtained from $S^{n}$ by gluing, that is identifying diametrically opposite points. This means we have the isomorphism $\RR P^n \simeq S^{n}/Z_2$.   
\end{itemize}


\subsection{Quaternions} 
\begin{itemize}
  \item The set $\mathbb{H}$ of \emph{Quaternions} consists of all linear combinations:
  \begin{align*}
    q \in \mathbb{H}, ~~~q = a \boldsymbol{1} + b\boldsymbol{i} + c\boldsymbol{j} + d\boldsymbol{k},~~~a,b,c,d\in \RR
  \end{align*}
  Where $\boldsymbol{1},\boldsymbol{i},\boldsymbol{j},\boldsymbol{k}$ are linearly independent. Where these bases satisfy the following multiplications:
\begin{align*}
    \bm{i} \cdot \bm{j} &= \bm{k} = -\bm{j} \cdot \bm{i}, & \bm{j} \cdot \bm{k} &= \bm{i} = -\bm{k} \cdot \bm{j}, & \bm{k} \cdot \bm{i} &= \bm{j} = -\bm{i} \cdot \bm{k}, \\
    \bm{i} \cdot \bm{i} &\equiv \bm{i}^2 = -1, & \bm{j} \cdot \bm{j} &\equiv \bm{j}^2 = -1, & \bm{k} \cdot \bm{k} &\equiv \bm{k}^2 = -1, \\
    \bm{i} \cdot \bm{1} &= \bm{i} = \bm{1} \cdot \bm{i}, & \bm{j} \cdot \bm{1} &= \bm{j} = \bm{1} \cdot \bm{j}, & \bm{k} \cdot \bm{1} &= \bm{k} = \bm{1} \cdot \bm{k}, & \bm{1} \cdot \bm{1} &= \bm{1}.
\end{align*}
This makes $\mathbb{H}$ an associative algebra over the field of real numbers. 

\end{itemize}

\subsection{Complex Projective spaces}
\begin{itemize}
  \item The \emph{complex projective space} $\mathbb{CP^n}$ is the set of equivalence classes of non-zero vectors in $\mathbb{C^{n+1}}$ where two nonzero vectors are equivalent if they are scalar multiples of one another. 

  \item In a similar manner to the real projective space we can identify the isomorphism: $\mathbb{C} \simeq S^{2n+1}/U(1)$. 
\end{itemize}

\newpage 
\section{Lie Algebras}
\subsection{Neighborhood of identity element}
\begin{itemize}
  \item Let $G$ be a Lie group. let the point $g_0\equiv 1 \in G$ be the identity element of $G$, and let $T = T_{(1)}$ be the tangent space at the identity element. We can now express the group operations on $G$ in a chart $U_0$ containing $g_0$ in terms of local co-ords. We choose co-ords in $U_0$ so that the identity element is the origin. $g_0 \equiv 1 = (0,\ldots,0)$. then if we let:
  \begin{align*}
    g_1=(x^1,\ldots,x^n),~~~g_2=(y^1,\ldots,y^n),~~~g_3=(z^1,\ldots z^n)
  \end{align*}
  Which allows us to define the product of two elements:
  \begin{align*}
    g_1g_2 = (\psi^{1}(x,y),\ldots,\psi^{n}(x,y)) = (\psi^i(x,y)) \in U_0
  \end{align*}
  An inverse as:
  \begin{align*}
    g_1^{-1} = (\varphi^1(x),..,\varphi^n(x)) = (\varphi^i(x)) \in U_0
  \end{align*}
  These functions $\varphi(x),\psi(x)$ satisfy:
  \begin{align*}
    & \psi^i(x,0) = \psi^i(0,x) = x^i \\
    & \psi^i(x,\varphi(x)) \\ 
    & \psi^i(x,\psi(y,z)) = \psi^i(\psi(x,y),z)
  \end{align*}
  \end{itemize}
  \subsubsection{Taylor expansion}
  \begin{itemize}
    \item Let $\psi^i(x,y)$ be sufficiently smooth and for $x,y,z \sim \epsilon $:
    \begin{align*}
      \psi^i(x,y) =&  x^i + y^i + b^{i}_{jk}x^jy^k + \mathcal{O}(\epsilon^3) \\
      & b^i_{jk} = \frac{\partial^2\psi^i}{\partial x^j\partial y^k}\bigg\vert_{x=y=0}
    \end{align*}
  \end{itemize}


  \subsection{Commutator}
  \begin{itemize}
    \item Let $\xi,\eta ~\in T$, and their components in terms of $x^i$ are $\xi^i$ and $\eta^i$. Then we can define the \emph{commutator} $[\xi,\eta] \in T$ is defined by:
    \begin{flalign*}
      [\xi,\eta]^{i} = c^{i}_{jk}\xi^{j}\eta^k,~~~c^{i}_{jk}\equiv b^{i}_{jk}-b^{i}_{kj}
    \end{flalign*}
    \item It has three basic quantities:
       \begin{itemize}
     \item It is \emph{bi-linear} operation on the $n$-dim vector space $T$. 
     \item Skew-symmetry: $[\xi,\eta] = -[\eta,\xi]$.
     \item Jacoby identity: $[[\xi,\eta],\zeta]+[[\zeta,\xi],\eta]+[[\eta,\zeta],\xi]$
   \end{itemize}
  \end{itemize}

  \subsection{Lie Algebra}
  \begin{itemize}
    \item A \emph{Lie algebra} is a vector space $\mathcal{G}$ over a field $F$ with a bi-linear operation $[\cdot ,\cdot ]: \mathcal{G} \times \mathcal{G} \rightarrow \mathcal{G}$ which is called a commutator or a lie bracket, such that the three axioms above are satisfied. 

  \item This means we can identify the tangent space of a Lie Group at the identity is with respect to the commutator operation of a Lie algebra called the \emph{Lie algebra of the Lie group G}.  

   \item If we choose $\xi = e_j, \eta= e_k$, then combined with the fact that $(e_m) = \delta_{m}^{n}$, then we have:
   \begin{flalign*}
      [e_j,e_k]^i = c^i_{jk}e_i
    \end{flalign*} 
  \end{itemize}

  \subsubsection{Structure Constants}
  \begin{itemize}
    \item The constants $c^{i}_{jk}$ which determine the commutation operation on a Lie algebra, and which are skew-symmetric in $j,k$ are called the \emph{structure constants} of the Lie algebra.  
  \end{itemize} 

\newpage 
\section{One parameter subgroups} 
\begin{itemize}
  \item A \emph{One parameter subgroup} of a lie group $G$ is defined to be a parametric curve $F(t)$ on the manifold $G$ such that:
  \begin{align*}
    F(0) = 1,~~F(t_1+t_2) = F(t_{1})F(t_2),~~~F(-t)= F^{-1}(t)
  \end{align*}
  The velocity vector at $F(t)$ is:
  \begin{align*}
    \frac{dF}{dt} = \frac{dF(t+\epsilon)}{dt}\bigg\vert_{\epsilon=0} = \frac{d}{d\epsilon}(F(t)F(\epsilon))\bigg\vert_{\epsilon=0} = F(t)\frac{dF(\epsilon)}{d\epsilon}\bigg\vert_{\epsilon=0}
  \end{align*}
  Hence:
  \begin{align*}
    \dot{F}(t) = F(t)\dot{F}(0) ~~~\text{or}~~~F^{-1}(t)\dot{F}(t) = \dot{F}(0)
  \end{align*}
  i.e. the induced action of left multiplication by $F^{-1}(t)$ sends $\dot{F}(t)$ to $\dot{F}(0) = const \in T $. 

  \item conversely, $\forall ~ A \in T$ the equation $F^{-1}(t)\dot{F}(t) = A$ is satisfied by a unique one-parameter subgroup $F(t)$ of $G$. If $G$ is a matrix group then $F(t) - exp (At)$.  
\end{itemize}

\subsection{Co-ords of the first kind}
\begin{itemize}
  \item One parameter subgroups can be used to define so called \emph{canonical} in a neighborhood of the identity of a Lie group $G$. 

  \item Let $A_1,\ldots,A_n$ form a basis for the Lie algebra $T$. Then $\forall~ A = \sum_iA_ix^i ~ \in ~ T ~\exists $ a one parameter group $F(t) = exp(At)$. To the point $F(1)= exp(A)$ we assign as co-ords co-officiants $x^1,\ldots,x^n$, which gives us a system of co-ords in a sufficiently small neighborhood of $g_0 =1 \in G$. These are called the \emph{canonical co-ords of the first kind}. 
\end{itemize}

\subsection{Co-ords of the second kind}
\begin{itemize}
  \item Another system of co-ords is obtained by introducing $F_i(t)= exp (At)$ and representing a point $g$ sufficiently close to $g_0$ as:
  \begin{align*}
    g = F_1(t_1)F_2(t_2)\cdot \cdot \cdot F_n(t_n)
  \end{align*}
  for small $t_1,...,t_n$. Assigning co-ords $x^1=t_1,...,x^n=t_n$ to the point $g$, we get the \emph{canonical co-ords of the second kind}.  
\end{itemize}


\newpage
\section{Linear Representations}
\subsection{Representations}
\begin{itemize}
  \item A \emph{Linear representation} of a group $G$ of $dimG = n$ is a homomorphism:
  \begin{align*}
    \rho:G \rightarrow GL(r,\RR), ~~~\text{or}~~~\rho: G\rightarrow GL(r,\CC)
  \end{align*}

\item Given a representation $\rho$ of $G$ the map: 
\begin{align*}
  \chi_{\rho}: G \rightarrow \RR, ~~~\text{or}~~~G \rightarrow \CC
\end{align*}
defined by:
\begin{align*}
\chi_{\rho}(g) = \text{tr}(\rho(g))
\end{align*}
is called the \emph{character} of the representation $\rho$. 

\item A representation $\rho$ of $G$ is said to be \emph{irreducible} if the vector space $\RR^r$ contains no proper subspace invariant under the matrix group $\rho(G)$. 
\end{itemize}

\subsubsection{Matrix Invariance}
\begin{itemize}
  \item A subspace $W$ of the representation space $\RR^r$ is called \emph{invariant under the matrix group} $\rho(G)$ (or simply $G$ invariant) if:
  \begin{align*}
    \rho(G)W \subset W,~~~\forall~g \in G
  \end{align*}
  Then we can restrict $\rho$ to $W$ and get a \emph{subrepresentation}.
\end{itemize}

\subsection{Schur's Lemma}
\begin{itemize}
  \item Let $\rho_{i}: G \rightarrow GL(r_i,\RR),~~i=1,2$ be two irreducible representations (irreps) of a group $G$. If $A:R^{r_1} \rightarrow \RR^{r_2}$ is a linear transformation changing $\rho_1$ to $\rho_2$, i.e. stratifying:
  \begin{align*}
    A \rho_1(g) = \rho_2A,~~~\forall~g\in G
  \end{align*}
  Then either $A$ is the zero transformation or else a bijection, in which case $r_1=r-2$. 
\end{itemize}

\subsection{Push Forward Representation}
\begin{itemize}
  \item If $G$ is a Lie group and a representation $\rho: G \rightarrow GL(r,\RR)$ is a smooth map, then the push-froward map $\rho_{\ast}$ is a linear map from the Lie algebra $\mathfrak{g} = T_{(1)}$ to the space of all $r\times r$ matrices:
  \begin{align*}
    \rho_{\ast}: \mathfrak{g}\rightarrow Mat(r,\RR) 
  \end{align*}
  It can then be shown that this means $\rho_{\ast}$ is a \emph{representation} of the Lie algebra $\mathfrak{g}$, i.e. that it is a Lie algebra homomorphism. Meaning it is linear and preserves the commutators $\rho_{\ast}[\xi,\eta] = [\rho_{\ast}\xi,\rho_{\ast}\eta]$.    
\end{itemize}


\subsection{Faithful}
\begin{itemize}
  \item A representation $\rho: G \rightarrow GL(r,\RR)$ is called \emph{faithful} if it is one to one i.e. if its Kernel is trivial. So $\rho(g) \neq \mathbb{I}$ unless $g=g_0$. 

   \item If a Lie group has a faithful representation then it can be realized as a matrix Lie group.   
\end{itemize}

\subsection{Inner automorphism}
\begin{itemize}
  \item For each $h \in G$ the transformation $G \rightarrow G$ defined by $g \rightarrow   hgh^{-1}$ is called the \emph{inner automorphism}. of $G$ determined by $h$. 

  \item Any inner automorphism does not move the identity element. i.e. $g_0 = hg_0h^{-1}$ and therefor the push forward (induced linear) map of the tangent space $T$ to $G$ at $g_0$  is a linear transformation of $T$ denoted by:
  \begin{align*}
     Ad_h: T \rightarrow T
   \end{align*} 
   it satisfies the following:
   \begin{itemize}
     \item $Ad_{g_0}= id$, where $id$ is the identity transformation of $T$. 
     \item $Ad_{h_1}Ad_{h_2} = Ad_{h_1h_2}$ for all $h_1,h_2 \in G$. because $h_1h_2gh_2^{-1}h_1^{-1} = (h_1h_2)g(h_1h_2)^{-1}$. 
     \item Choosing $h_1 = h, h_2 = h^{-1}$, we get that $Ad_{h^{-1}} = Ad_{h}^{-1}$
   \end{itemize}

   \item This means that the map $h \rightarrow Ad_{h}$ is a \emph{linear representation} of the group $G$. i.e. a homomorphism to a group of linear transformations, $Ad: G \rightarrow GL(n,\RR), h\rightarrow Ad_{h} = Ad(h)$. This representation of $G$ is called \emph{Adjoint}. 
\end{itemize}

\subsection{One Parameter Adjoint}
\begin{itemize}
  \item Let $F(t) = e^{At}$ be a one parameter subgroup of a Lie  group $G$. Then $Ad_{F(t)}$ is a one parameter subgroup of $GL(n,\RR)$. 

  The vector $\frac{d}{dt}Ad_{F(t)}\Big\vert_{t=0}$ lies in the Lie algebra $\mathfrak{g} \sim Mat(n,\RR)$ of the Group $GL(n,\RR)$ and can be regarded as a linear operator. 

   \item This operator is denoted $ad_{A}$ and is given by:
   \begin{flalign*}
      ad_A:\RR \rightarrow \RR, ~~~ B \rightarrow [A,B],~~~B \in T \simeq \RR^n
    \end{flalign*} 
\end{itemize}


\newpage 
\section{Simple Lie Algebras and Forms}

\subsection{Simple \& Semi-Simple}
\begin{itemize}
  \item A Lie algebra $\mathfrak{g} = \{\RR^n,c^{i}_{jk}\}$ is said to be \emph{simple} if it is \emph{non-commutative} and has \emph{no proper ideals}, i.e. subspaces $\mathcal{I} \neq \mathfrak{g},0$ for which $[\mathcal{I},\mathfrak{g}] \subset \mathcal{I}$.  


  \item It is instead called \emph{semi-simple} if we can write $\mathfrak{g} = \mathcal{I}_1 \otimes \mathcal{I}_2 \otimes \cdot \cdot \cdot \otimes \mathcal{I}_k$ Where the $\mathcal{I}_j$ are ideals which are simple as Lie algebras. These ideals are pairwise commuting $[\mathcal{I}_i,\mathcal{I}_j] = 0,~~i\neq j$. 

  A Lie group is defined to be simple or semi-simple according to its Lie algebra. 

  \item A theorem that can be proven is that if the Lie algebra $\mathfrak{g}$ of a Lie group $G$ is simple, then the linear representation $Ad:G\rightarrow GL(n,\RR)$ is \emph{irreducible}, i.e. $\mathfrak{g}$ has no proper invariant sub-spaces under the group of inner automorphisms $Ad_{G}$.  
\end{itemize}

\subsection{Killing Form}
\begin{itemize}
  \item The \emph{Killing form} on an arbitrary Lie algebra $\mathfrak{g}$ is defined (up to a sign) by:
  \begin{align*}
     \braket{A,B} = -\text{tr}(ad_{A}ad_{B})
   \end{align*} 
   \item If the Killing form of a Lie algebra is positive definite then the Lie algebra is semi-simple. 

   \item We also have that a Lie algebra is semi-simple if and only if its Killing form is non-degenerate. 
\end{itemize}

\newpage 
\section{Group Actions}
\subsection{Left and Right actions} % (fold)
\label{sub:left_and_right_actions}
\begin{itemize}
  \item We say that a Lie group $G$ is represented as a \emph{group of transformations} of a manifold $M$, or has a \emph{left action} on $M$ if:
  \begin{itemize}
    \item There is associated with each of its elements $g$ a diffeomorphism from $M$ to itself. $x \mapsto \mathcal{T}_{g}(x),~~~x \in M$. Such that $\mathcal{T}_{g}\mathcal{T}_{h} = \mathcal{T}_{gh},~\forall~g,h \in G$. 
    \item $\mathcal{T}_{g}(x)$ depends smoothly on the arguments $g,x$ i.e. the map $(g,x) \mapsto \mathcal{T}_{g}(x)$ is a smooth map from $G \times M \rightarrow M$. 
  \end{itemize}

  \item The Lie group is said to have \emph{Right action} on $M$ if the above definition is valid with $\mathcal{T}_{g}\mathcal{T}_{h} = \mathcal{T}_{hg}$.  
\end{itemize}

% subsubsection left_and_right_actions (end)


\subsection{Transitivity} % (fold)
\label{sub:transitivity}
\begin{itemize}
  \item The action of a group $G$ on $M$ is said to be \emph{transitive} if for every two points $x,y \in M$ there exists an element of $G$ such that $\mathcal{T}_{g}(x) =y$. 

  To show that an action of a group on a manifold is transitive it is sufficient to choose any point of $M$ as a reference point $x_0$, and to prove that for any point $y\in M$ there exists an element $g \in G$ such that $y = \mathcal{T}_{g}(x_0)$. 
\end{itemize}
% subsection transitive (end)

\subsubsection{Homogeneity} % (fold)
\label{ssub:Homogeneity}
\begin{itemize}
  \item A manifold on which a Lie group acts transitively is called a \emph{homogeneous space} of the Lie group. 

  \item In particular, $G$ is a homogeneous space for itself, e.g. as $h \rightarrow \mathcal{T}_{g}(h)=gh,~h \in G$. $G$ is called the \emph{principle} homogeneous space. 
\end{itemize}
\subsubsection{Isotropy group} % (fold)
\label{ssub:isotropy_group}
\begin{itemize}
  \item Let $x$ be any point of a homogeneous space $M$ of a Lie group $G$. The \emph{isotropy} group (or \emph{stationary} group) $H_{x}$ of the point $x$ is the stabilizer of $x$ under the action of $G$:
\begin{align*}
   H_x = \{ h | \mathcal{T}_{h}(x) = x\}
 \end{align*} 
 \item All isotropy groups $H_x$ of points $x$ of a homogeneous space are isomorphic. 

 \item There is a one to one correspondence between the points of a homogeneous space $M$ of a group $G$, and the left cosets $gH$ of $H$ in $G$, where $H$ is the isotropy group and $G$ acts on the left. Thus we can write $M \simeq G/H$, i.e. $M$ is a diffeomorphic to the quotient space $G/H$.    
\end{itemize}
% subsubsection isotropy_group (end)
% subsection homogeneous (end)

\subsection{Examples of Homogeneous spaces} % (fold)
\label{sub:examples_of_homogeneous_spaces}
\subsubsection{Stiefel manifolds} % (fold)
\label{ssub:stiefel_manifolds}
\begin{itemize}
  \item For each $n,k$ the Stiefel manifold $V_{n,k}$ has as its points \emph{all orthonormal} $k$-frames $x=(e_1,\ldots,e_k)$ of $k$ vectors $e_{a}$ in $\RR^n$. 

  \item The dimension of $V_{n,k}$ is $nk-\frac{1}{2}k(k+1)$ and $V_{n,k} \simeq O(n)/O(n-k) \simeq SO(n)/SO(n-k)$.
\end{itemize}
% subsubsection stiefel_manifolds (end)
\subsubsection{Real Grassmanian manifolds} % (fold)
\label{ssub:real_grassmanian_manifolds}
\begin{itemize}
  \item The points of $G_{n,k}$ are the $k$ dimensional planes passing through the origin of $\RR^{n}$. 

  \item It can be shown that $G_{n,k} \simeq O(n)/(O(k)\times O(n-k)) \simeq G_{n,n-k}$. The dimension of $G_{n,k}$ is $(n-k)k$ 
\end{itemize}
% subsubsection real_grassmanian_manifolds (end)

% subsection examples_of_homogeneous_spaces (end)


\newpage 
\section{Vector Bundles} % (fold)
\label{sec:vector_bundles}
\subsection{Tangent Bundle} % (fold)
\label{sub:tangent_bundle}
\begin{itemize}
  \item The \emph{tangent bundle} $T(M)$ of an $n$ dimensional manifold $M$ is a $2n$ dimensional manifold defined as follows:
  \begin{itemize}
    \item The points of $T(M)$ are the pairs $(x,\xi), x \in M,~\xi \in T_{x}M$. 
    \item Given a chart $U_{q}$ of $M$ with the local co-ords $(x^i_{q})$, the corresponding chart $U^T_q$ of $T(M)$ is the set of all pairs $(x,\xi)$ where:
    \begin{align*}
       x = (x^{1}_1,\ldots,x^n_{q}) \in U_{q},~~~\xi = \xi^{i}_q\frac{\partial}{\partial x^i_q} \in T_xM
     \end{align*} 
     with local co-ords $(y^1_q,\ldots,y^{2n}_q) = (x^1_q,\ldots,x^n_q,\xi^1_q,\ldots,\xi^n_q) = (x^i_q,\xi^i_q)$.
  \end{itemize}
  \item This tangent bundle is a smooth oriented manifold. 
\end{itemize}
% subsection tangent_bundle (end)

\subsection{Cotangent Bundle} % (fold)
\label{sub:cotangent_bundle}
\begin{itemize}
  \item The \emph{cotangent bundle} $T^{\ast}(M)$ of an $n$ dimensional manifold $M$ is a $2n$ dim manifold defined as follows:
  \begin{itemize}
    \item The points $T^{\ast}(M)$ are the pairs $(x,p), x \in M$ and $p$ a co-vector at the point $x$, so $p\in T^{\ast}_xM$. 
    \item Given a chart $U_{q} $ of $M$ with the local co-ords $(x^i_q)$, the corresponding chart $U^{T^{\ast}}_x$ of $T^{\ast}M$ is the set of all pairs $(x,p)$, where:
    \begin{align*}
    x = (x^{1}_1,\ldots,x^n_{q}) \in U_{q},~~~p = p_{qi}dx^i_q \in T^{\ast}_xM
    \end{align*}
    with local co-ords $(y^1_q,\ldots,y^{2n}_q) = (x^1_q,\ldots,x^n_q,p_{q1},\ldots,p_{qn}) = (x^i_q,p_{qi})$.
  \end{itemize}
  \item This cotangent bundle is a smooth oriented manifold. 
\end{itemize}
% subsubsection cotangent_bundle (end)

\subsection{Symplectic Manifold} % (fold)
\label{sub:symplectic_manifold}
\begin{itemize}
    \item The existence of a metric on $M$ gives rise to a map:
  \begin{align*}
   T(M) \rightarrow T^{\ast}(M): (x^i,\xi^i)\mapsto (x^i,g_{ij}\xi^i)
   \end{align*} 
   \item Since $\omega = p_idx^i$, a differential one-form on $M$, is invariant under a change of co-ords of $T^{\ast}(M)$, it is a differential form on $T^{\ast}(M)$. 

   Its differential $\Omega = d\omega = dp_i \wedge dx^i$ is a \emph{non-degenerate closed},($d\Omega =0$), 2-form on $T^{\ast}(M)$. 

   \item Thus $T^{\ast}(M)$ is a $symplectic$ manifold, i.e. it is equipped with a closed non-degenerate 2-form. 
 \end{itemize}
% subsection symplectic_manifold (end)
% section vector_bundles (end)

\newpage

\section{Vector and Tensor Fields} % (fold)
\label{sec:vector_and_tensor_fields}
\subsection{Vector Field} % (fold)
\label{sub:vector_field}
\begin{itemize}
  \item A \emph{vector field}  is a map that specifies a unique vector at each point $x$ of the manifold $M$:
  \begin{align*}
    \xi: M \rightarrow T(M), ~~~x\mapsto \xi_x \in T_xM
  \end{align*}
  A vector field intersects each tangent space of $T(M)$ at one and only one point, i.e. a vector field is a curve which is no-where parallel to a tangent space. It is a \emph{cross~section} of $T(M)$. 

    \item A vector field can be understood as a differential operator that maps a scalar function to a scalar function on $M$:
    \begin{align*}
        \xi(f) = \xi^i\frac{\partial f}{\partial x^i}. 
      \end{align*}  
      \item These maps are linear and satisfy the Leibniz rule. This means they are \I{derivations}. 
\end{itemize}
% subsubsection vector_field (end)

\subsection{Tensor Field} % (fold)
\label{sub:tensor_field}
\begin{itemize}
  \item A \I{Tensor field} of type $(r,s)$ assigns a unique tensor of type $(r,s)$ to each point $x$ of the manifold $M$:
  \begin{align*}
    ^{(r,s)}\xi:M \rightarrow T^{(r,s)}(M),~~~x \mapsto ^{(r,s)}\xi_x \in T^{(r,s)}_xM
  \end{align*}
.   It is a cross section of $T^{(r,s)}_xM$. 
\end{itemize}
% subsection tensor_field (end)
% section vector_and_tensor_fields (end)

\subsection{Commutator or Lie Bracket} % (fold)
\label{sub:commutator_or_lie_bracket}
\begin{itemize}
  \item Consider the composition $\xi(\eta(f)) = \xi^{i}\frac{\partial}{\partial x^i}\left(\eta^{i}\frac{\partial f}{\partial x^j}\right) = \xi^i\frac{\partial \eta^j}{\partial x^i}\frac{\partial f}{\partial x^j}+\xi^i\eta^j\frac{\partial^2 f}{\partial x^i\partial x^j}$. The second term makes this not a vector field.  

  \item This motivates us to define the \I{commutator} or \I{Lie Bracket} define by:
  \begin{align*}
  [\xi,\eta](f) \equiv \xi(\eta(f))  - \eta(\xi(f)) = \left(\xi^i\frac{\partial \eta^j}{\partial x^i}-\eta^i\frac{\partial \xi^j}{\partial x^i}\right)\frac{\partial f}{\partial x^j}
  \end{align*}
  This is a vector field that satisfies:
\begin{align*}
[\xi, \eta] &= -[\eta, \xi] \\
[\xi, \eta + \zeta] &= [\xi, \eta] + [\xi, \zeta] \\
[\xi, f\eta] &= f[\xi, \eta] + \xi(f)\eta \\
[\xi, [\eta, \zeta]] &+ [\eta, [\zeta, \xi]] + [\zeta, [\xi, \eta]] = 0
\end{align*}
Thus thee vector space equipped with the commutator operation is an \I{infinite dimensional Lie algebra}.
\end{itemize}
% subsection commutator_or_lie_bracket (end)


\subsection{Integral curves} % (fold)
\label{sub:integral_curves}
\begin{itemize}
  \item let $\xi^i(x)$ bee a vector fie;d on $M$. Consider the autonomous (meaning the equations have no explicit dependence on $t$) system of differential equations:
  \begin{align*}
  \dot{x}^i \equiv \frac{dx^i}{dt} = \xi^i(x^1(t),\ldots,x^n(t)),~~~i=1,\ldots,n
  \end{align*}
  \item The solutions $x^i(t)$ to this system are called the \I{integral curves} of the vector field $\xi^i$. The vector field $\xi^i(x)$ is comprised of tangent vector to the integral curves.   
\end{itemize}
% subsection integral_curves (end)

\subsubsection{Flows and Velocity Fields} % (fold)
\label{ssub:flow}
\begin{itemize}
  \item A local abelian one parameter subgroup of diffeomorphisms $F_t$ is called the \I{flow} generated by the vector field $\xi^i$. Where $F^i_t(x^1_0,\ldots,x^n_0) = x^i(t,x^1_0,\ldots,x^n_0)$.

\item This can be reversed to say that given a one parameter local group of diffeomorphisms we can define its \I{velocity fields} to be thee vector field:
\begin{align*}
\xi^i = \left(\frac{d}{dt}F^i_t\right)
\end{align*}
\item Note that in general two flows do not commute and in fact the commutator measures thee discrepancy between the points obtained by following the integral curves of two different vector fields in different orders. 
\item The vectors comprising a co-ord induced basis commute because all the partial derivatives do, The converse is also true, if all the elements of a basis for vector fields commute then the basis is co-ordinate induced.
\end{itemize}
% subsubsection flow (end)

\subsection{Exponential function of Vector Fields} % (fold)
\label{sub:exponential_function_of_vector_fields}
\begin{itemize}
  \item A one parameter subgroup of diffeomorphisms $F_t(x)$ with associated vector field $\xi(x)$ is defined to \I{act on smooth functions} $f=f(x)$ as follows:
  \begin{align*}
    (F_tf)(x) = f(F_t(x)). 
    \end{align*}  
    \item The \I{exponential function} of a vector field $\xi$ is the operator:
    \begin{align*}
    \exp(t\partial_{\xi}) = 1+ \partial_{\xi}+\frac{t^2}{2}(\partial_{\xi}^2)+ \ldots
    \end{align*}
    Where $\partial_{\xi}$ is the directional derivation operator in the direction of $\xi$. 

    \item The action of $e^{t\partial_{\xi}}$ on functions $f(x)$ is defined as:
    \begin{align*}
       \exp(t\partial_{\xi})f = f+ \partial_{\xi}f+\frac{t^2}{2}(\partial_{\xi}^2)f+ \ldots
       \end{align*}   `
      $\forall~t$ for which this series converges. 

       \item  For analytic vector fields $\xi(x)$ and analytic functions $f(x)$ the exponential function of $\xi(x)$ i.e. $e^{t\partial_{\xi}}$, coincides for sufficiently small $t$ with the action of $F_t$ on $f$:
       \begin{align*}
        e^{\partial_{\xi}}f = f(F_t(x))
        \end{align*} 
\end{itemize}
% subsection exponential_function_of_ (end)

\newpage 
\section{The Lie Derivative} % (fold)
\label{sec:the_lie_derivative}
\begin{itemize}
  \item The motivation for this is that we want to define the action of a flow $F_t$ generated by $\xi$ on Tensors. The problem with this is that we have no way of comparing tensors at \I{different points} on the manifold. To fix this we switch to active transformations where the points $x^i$ stay fixed and we instead change the co-ordinate system by acting on the basis with $F_t^{-1} = F_{-t}$. 
\end{itemize}

\subsection{Action of Flows on Tensors} % (fold)
\label{sub:action_of_flows_on_tensors}
\begin{itemize}
  \item A one parameter subgroup of diffeomorphisms $F_t(x)$ with associated vector field $\xi(x)$ is defined to \I{act on smooth tensors} $T = (T^{i_1,..,i_p}_{j_1,\ldots,j_q})$ of type $(p,q)$ as follows:
  \begin{align*}
    (F_tT)^{i_1,..,i_p}_{j_1,\ldots,j_q}(x)  = T^{k_1,..,k_p}_{l_1,\ldots,l_q}(y)\frac{\partial y^{l_1}}{\partial x^{j_1}}\cdot \cdot \cdot \frac{\partial y^{l_1}}{\partial x^{j_q}}\frac{\partial x^{i_1}}{\partial y^{k_1}}\cdot \cdot \cdot \frac{\partial x^{i_p}}{\partial y^{k_p}}
    \end{align*}  
    Where $y^i = F_t^i(x)$.
\end{itemize}
% subsection action_of_flows_on_tensors (end)

\subsection{Lie Derivative} % (fold)
\label{sub:lie_derivative}
\begin{itemize}
  \item The Lie derivative of a tensor $T=(T^{i_1,..,i_p}_{j_1,\ldots,j_q})$ along a vector field $\xi$ is the tensor $L_{\xi}T$ given by:
  \begin{flalign*}
  L_{\xi}T^{i_1,..,i_p}_{j_1,\ldots,j_q} = \left[\frac{d}{dt}(F_tT)^{i_1,..,i_p}_{j_1,\ldots,j_q}\right]_{t=0}
  \end{flalign*}
\end{itemize}
% subsection lie_derivative (end)
% section the_lie_derivative (end)
\end{document}
 